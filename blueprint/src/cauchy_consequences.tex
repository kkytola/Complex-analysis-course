\section{Analyticity of derivatives}

\begin{lemma}[\Palka{Lemma~V.1.6}]
  \label{lem:derivative_auxiliary_integral}
  \uses{def:contour_integral, def:analytic_function}
  % \lean{}
  % \leanok
  Let $\gamma$ be a contour in $\bC$, and let $h \colon \gamma \to \bC$
  be a continuous function on the contour (we slightly abuse the notation
  here to identify the contour as a subset $\gamma \subset \bC$).
  Let $k \in \bN$ be a positive integer.
  Define $H \colon \bC \setminus \gamma \to \bC$ by
  \begin{align*}
    H(z) = \int_\gamma \frac{h(\zeta)}{(\zeta - z)^{k}} \, \ud \zeta .
  \end{align*}
  Then $H$ is analytic on $\bC \setminus \gamma$, and its derivative
  at $z \in \bC \setminus \gamma$ is given by
  \begin{align*}
    H'(z) = k \, \int_\gamma \frac{h(\zeta)}{(\zeta - z)^{k+1}} \, \ud \zeta .
  \end{align*}
\end{lemma}
\begin{proof}
  % \uses{}
  % \leanok
  \ldots
\end{proof}

\begin{lemma}[Analyticity of derivatives \Palka{Thm~V.3.1}]
  \label{lem:derivative_analytic}
  \uses{def:analytic_function}
  %\lean{}
  % \leanok
  If a function $f \colon U \to \bC$ is analytic on an open set $U \subset \bC$,
  then its derivative $f'$ is also analytic on $U$.
  In particular, then $f$ is continuously differentiable,
  $f \in \ContDiff{1}(U)$.
\end{lemma}
\begin{proof}
  \uses{thm:CAUCHY_FORMULA, lem:derivative_auxiliary_integral}%\uses{lem:lemma_1_6}
  % \leanok
  \ldots
\end{proof}

\begin{corollary}[Analyticity of higher derivatives \Palka{Cor~V.3.2}]
  \label{lem:higher_derivatives_analytic}
  \uses{def:analytic_function}
  %\lean{}
  % \leanok
  If a function $f \colon U \to \bC$ is analytic on an open set $U \subset \bC$,
  then all its derivatives $f', f'', \ldots, f^{(k)}, \ldots$ are also analytic on $U$.
  In particular, then $f$ is infinitely differentiable,
  $f \in \Smooth(U)$.
\end{corollary}
\begin{proof}
  \uses{lem:derivative_analytic}
  % \leanok
  Straightforward induction using Lemma~\ref{lem:derivative_analytic}.
\end{proof}

\begin{theorem}[Morera's theorem \Palka{Thm~V.3.3}]
  \label{thm:morera}
  \uses{def:analytic_function}
  %\lean{}
  % \leanok
  Let $f \colon U \to \bC$ be a continuous function on an open set $U \subset \bC$.
  If $f$ has the property that
  \begin{align*}
    \oint_{\bdry \Tri} f(z) \, \ud z = 0
  \end{align*}
  for any closed triangle $\Tri \subset U$, then $f$ is analytic on $U$.
\end{theorem}
\begin{proof}
  \uses{def:primitive, lem:derivative_analytic}
  % \leanok
  \ldots
\end{proof}

\begin{theorem}[Cauchy's integral formula for derivatives]
  \label{thm:cauchy_formula_derivative}
  \uses{def:analytic_function}
  %\lean{}
  % \leanok
  Suppose that a function $f \colon U \to \bC$ is analytic on an open
  set $U \subset \bC$ containing the disk $\DiskRC{r}{z} \subset U$.
  Then for any closed contour $\gamma$ in $\DiskRC{r}{z}$
  and $n \in \bN$ we have
  \begin{align*}
    \windN{\gamma}{z} \, f^{(n)}(z)
      = \frac{n!}{2\pi \ii} \oint_\gamma \frac{f(\zeta)}{(\zeta - z)^{n+1}} \, \ud \zeta .
  \end{align*}
\end{theorem}
\begin{proof}
  \uses{thm:CAUCHY_FORMULA}
  % \leanok
  \ldots
\end{proof}

\begin{lemma}[Cauchy's estimate for derivatives \Palka{Thm~V.3.6}]
  \label{lem:cauchy_derivative_estimate}
  \uses{def:analytic_function}
  %\lean{}
  % \leanok
  Suppose that a function $f \colon U \to \bC$ is analytic on an open
  set $U \subset \bC$ containing the disk $\DiskRC{r}{z} \subset U$,
  and suppose that there exists a constant $M>0$ such that
  $|f(z)| \le M$ for all $z \in \DiskRC{r}{z}$.
  Then for any $n \in \bN$ we have
  \begin{align*}
    \big| f^{(n)}(z) \big| \le n! \, M \, r^{-n} .
  \end{align*}
\end{lemma}
\begin{proof}
  \uses{thm:cauchy_formula_derivative, lem:integral_triangle_inequality}
  % \leanok
  \ldots
\end{proof}



\section{Liouville's theorem}

\begin{theorem}[Liouville's theorem \Palka{Thm~V.3.7}]
  \label{thm:liouville}
  \uses{def:analytic_function}
  %\lean{}
  % \leanok
  If a function $f \colon \bC \to \bC$ on the entire complex plane
  is analytic and bounded, then $f$ is a constant function.
\end{theorem}
\begin{proof}
  \uses{lem:cauchy_derivative_estimate}
  % \leanok
  \ldots
\end{proof}



\section{The fundamental theorem of algebra}

\begin{theorem}[Fundamental theorem of algebra \Palka{Thm~V.3.8}]
  \label{thm:FUNDAMENTAL_THEOREM_OF_ALGEBRA}
  \uses{def:polynomial}
  %\lean{}
  % \leanok
  Every non-constant polynomial function $p \colon \bC \to \bC$
  has a root, i.e., there exists a $z_0 \in \bC$ such that $p(z_0) = 0$.
\end{theorem}
\begin{proof}
  \uses{thm:liouville, lem:polynomial_analytic}
  % \leanok
  \ldots
\end{proof}

\begin{corollary}[Fundamental theorem of algebra \Palka{Thm~V.3.9}]
  \label{cor:complex_polynomial_factorization}
  \uses{def:polynomial}
  %\lean{}
  % \leanok
  A complex-coefficient
  polynomial $p(z) = a_0 + a_1 z + a_2 z^2 + \cdots + a_n z^n$
  of degree $n \in \bN$ can be factored as
  \begin{align*}
    p(z) = c \, (z - z_1) \, (z - z_2) \, \cdots \, (z - z_n)
  \end{align*}
  where $c = a_n \ne 0$, and $z_1 , \ldots, z_n \in \bC$
  are the roots of $p$ (with repetition according to the
  multiplicities of the roots).
\end{corollary}
\begin{proof}
  \uses{thm:FUNDAMENTAL_THEOREM_OF_ALGEBRA}
  % \leanok
  This follows from Theorem~\ref{thm:fundamental_theorem_of_algebra}
  by induction on the degree of the polynomial,
  using the polynomial division (Euclidean algorithm in the
  ring of univariate polynomials, see \textbf{MS-C1081 Abstract Algebra}).
\end{proof}



\section{Maximum principle}

\begin{theorem}[Maximum principle for analytic functions \Palka{Thm~V.3.10}]
  \label{thm:maximum_principle}
  \uses{def:analytic_function, def:connected}
  %\lean{}
  % \leanok
  Let $f \colon \Domain \to \bC$ be an analytic function on a
  connected open set $\Domain \subset \bC$.
  Suppose that there exists a point $z_0 \in \Domain$
  such that
  \begin{align*}
    |f(z)| \le |f(z_0)| \qquad \text{ for all } z \in \Domain .
  \end{align*}
  Then $f$ is a constant function.
\end{theorem}
\begin{proof}
  \uses{thm:CAUCHY_FORMULA}
  % \leanok
  \ldots
\end{proof}

\begin{corollary}[Maximum principle for analytic functions continuous up to the boundary \Palka{Cor~V.3.11}]
  \label{cor:maximum_modulus_on_boundary}
  \uses{def:analytic_function, def:connected}
  %\lean{}
  % \leanok
  Let $\Domain \subset \bC$ be a bounded connected open set.
  Let $f \colon \closOf{\Domain} \to \bC$ be a continuous function
  on its closure which is analytic in $\Domain$.
  Then $z \mapsto |f(z)|$ attains its maximum in $\closOf{\Domain}$
  at some point of the boundary $\bdry \Domain$.
\end{corollary}
\begin{proof}
  \uses{thm:maximum_principle}
  % \leanok
  \ldots
\end{proof}

\begin{lemma}[Schwarz's lemma \Palka{Thm~V.3.14}]
  \label{lem:schwarz_lemma}
  \uses{def:analytic_function}
  %\lean{}
  % \leanok
  Let $f \colon \DiskRC{1}{0} \to \bC$ be an analytic function
  on the open unit disk such that $|f(z)| \le 1$
  for all $z \in \DiskRC{1}{0}$ and $f(0) = 0$.
  Then we have
  \begin{align*}
    |f'(0)| \le 1
    \qquad \text{ and } \qquad
    |f(z)| \le |z| \quad \text{ for all } z \in \DiskRC{1}{0} .
  \end{align*}
  Furthermore, unless $f$ is of the form $f(z) = \lambda z$ for
  some $\lambda \in \bC$ with $|\lambda| = 1$, then we have
  \begin{align*}
    |f'(0)| < 1
    \qquad \text{ and } \qquad
    |f(z)| < |z| \quad \text{ for all } z \in \DiskRC{1}{0} \setminus \set{0}.
  \end{align*}
\end{lemma}
\begin{proof}
  \uses{cor:maximum_modulus_on_boundary}
  % \leanok
  \ldots
\end{proof}



\section{The mean value property}

\begin{theorem}[Mean value property for analytic functions]
  \label{thm:mean_value_property}
  \uses{def:analytic_function}
  %\lean{}
  % \leanok
  Suppose that a function $f \colon U \to \bC$ is analytic on an open
  set $U \subset \bC$ containing the closed disk $\CDiskRC{r}{z} \subset U$.
  Then we have
  \begin{align*}
    f(z) = \frac{1}{2 \pi r} \oint_{\bdry \DiskRC{r}{z}}
            \frac{f(\zeta)}{\zeta - z} \, |\ud \zeta| .
  \end{align*}
\end{theorem}
\begin{proof}
  \uses{thm:CAUCHY_FORMULA}
  % \leanok
  \ldots
\end{proof}
