% In this file you should put the actual content of the blueprint.
% It will be used both by the web and the print version.
% It should *not* include the \begin{document}
%
% If you want to split the blueprint content into several files then
% the current file can be a simple sequence of \input. Otherwise It
% can start with a \section or \chapter for instance.

This document is a skeleton of the theoretical content of
the Fall 2024 course MS-C1300 Complex Analysis at Aalto University.
The skeleton includes statements of results and their
logical interdependencies, but no proofs, no examples, and no
visualizations.

Instead, the hand-written lecture notes on the course home page correspond
to the contents of the lectures. They include discussions of examples and
informal ideas as well as the main theorems and their proofs.
This theory skeleton is meant to complement the lectures as
an organizational document: to indicate the flow of the theory
development, and to clarify the main goals of the course and the
purposes of the intermediate results.

An attempt is made to provide references to corresponding parts of
the textbook: \emph{An Introduction to Complex Function Theory}
(Undergraduate Texts in Mathematics) by Bruce Palka.

This skeleton is work in progress. It is guaranteed to contain mistakes,
the number and severity of which will be reduced with updates
during the course. Please inform the lecturer Kalle Kytölä of
errors that you notice!

\chapter{The complex number system}
\section{The field of complex numbers}

\begin{definition}[Complex numbers and their arithmetic operations \Palka{Sec. I.1.1}]
  \label{def:complex_numbers}
  %\uses{}
  %\lean{}
  % \leanok

  The set of \term{complex numbers} is $\bC = \bR \times \bR$, i.e.,
  the set of pairs $(x,y)$ of real numbers $x,y \in \bR$.

  The operations of addition and multiplication on $\bC$ are defined by the formulas
  \begin{align*}
    (x_1,y_2) + (x_2,y_2) = \; & (x_1+x_2, y_1+y_2) \\
    (x_1,y_1) \cdot (x_2,y_2) = \; & (x_1 x_2 - y_1 y_2 , \, x_1 y_2 + y_1 x_2) .
  \end{align*}
  %(x_1,y_2) + (x_2,y_2) = (x_1+x_2, y_1+y_2)
  %$(x_1,y_1) \cdot (x_2,y_2) = (x_1 x_2 - y_1 y_2 , \, x_1 y_2 + y_1 x_2)$.

  Denote
  $0 = (0,0) \in \bC$ and $1 = (1,0) \in \bC$.

  For $z = (x,y) \in \bC$, denote
  $-z %= -x - \ii \, y
    = (-x,-y) \in \bC$ and if $z \ne 0$ then denote
  $z^{-1} %= \frac{x}{x^2 + y^2} + \ii \frac{-y}{x^2 + y^2}
            = \Big(\frac{x}{x^2 + y^2} , \frac{-y}{x^2 + y^2} \Big) \in \bC$.

  We write a complex number $(x,y)$ as $x + \ii \, y$.
  The compex number $\ii = (0,1) \in \bC$ is called the \term{imaginary unit}.
  %Denote
  %$0 = 0 + 0 \, \ii = (0,0) \in \bC$ and $1 = 1 + 0 \, \ii = (1,0) \in \bC$.
  \end{definition}

Typically used variable names for complex number are $z,w,\zeta \in \bC$ etc.

\begin{theorem}[The field of complex numbers \Palka{Sec. I.1.1}]
  \label{thm:complex_number_field}
  \uses{def:complex_numbers}
  %\lean{}
  %\leanok

  The set $\bC$ of compex numbers with its operations of addition and multiplication,
  is a \term{field}, i.e., the following properties hold for
  all $z, w, z_1, z_2, z_3 \in \bC$:
  \begin{itemize}
    \item $z + w = w + z$ \term{(commutativity of addition)}
    \item $z w = w z$ \term{(commutativity of multiplication)}
    \item $z_1 + (z_2 + z_3) = (z_1 + z_2) + z_3$ \term{(associativity of addition)}
    \item $z_1 (z_2 z_3) = (z_1 z_2) z_3$ \term{(associativity of multiplication)}
    \item $0 = 0 + 0 \, \ii = (0,0) \in \bC$ satisfies $z + 0 = z$
          \term{(neutral element for addition)}
    \item $1 = 1 + 0 \, \ii = (1,0) \in \bC$ satisfies $z \cdot 1= z$
          \term{(neutral element for multiplication)}
    \item %for any $z = x + \ii y$, the element $-z %= -x - \ii \, y
          %= (-x,-y) \in \bC$ satisfies
          $z + (-z) = 0$ for any $z \in \bC$ \term{(opposite element / additive inverse)}
    \item %for any $z = x + \ii y \ne 0$, the element
          %  $z^{-1} %= \frac{x}{x^2 + y^2} + \ii \frac{-y}{x^2 + y^2}
          %  = \Big(\frac{x}{x^2 + y^2} , \frac{-y}{x^2 + y^2} \Big) \in \bC$ satisfies
          $z \, z^{-1} = 1$ for any $z \in \bC \setminus \set{0}$ \term{(multiplicative inverse)}
    \item $(z_1 + z_2) w = z_1 w + z_2 w$ \term{(distributivity)}.
  \end{itemize}
\end{theorem}
\begin{proof}
  Straightforward calculations using the definitions of the operations
  (\defref{def:complex_numbers}).
  %\uses{}
  % \leanok
\end{proof}

\section{The algebra and geometry of complex numbers}


\subsection{The field of complex numbers}


\subsection{Conjugate, modulus, and argument}

\begin{definition}[Complex conjugate \Palka{Sec.~I.1.2}]
  \label{def:complex_conjugate}
  \uses{def:complex_numbers}
  %\lean{}
  % \leanok
  The \term{complex conjugate} of a complex number $z = x + \ii y$
  (where $x,y \in \bR$) is the complex number $\cconj{z} = x - \ii y$.
\end{definition}


\begin{lemma}[Properties of complex conjugate \Palka{Sec.~I.1.2~(1.1)}]
  \label{lem:complex_conjugate_properties}
  \uses{def:complex_conjugate}
  %\lean{}
  % \leanok
  For any $z , w \in \bC$, we have
  \begin{align*}
    \cconj{\cconj{z}} = z , \qquad
    \cconj{z + w} = \cconj{z} + \cconj{w} , \qquad
    \cconj{z w} = \cconj{z} \, \cconj{w} ,
  \end{align*}
  \begin{align*}
    \re (z) = \frac{z + \cconj{z}}{2} , \qquad
    \im (z) = \frac{z - \cconj{z}}{2 \ii} .
  \end{align*}
\end{lemma}
\begin{proof}
  %\uses{}
  % \leanok
  Direct calculation.
\end{proof}

\begin{definition}[Absolute value (modulus) \Palka{Sec.~I.1.2}]
  \label{def:absolute_value}
  \uses{def:complex_numbers}
  %\lean{}
  % \leanok
  The \term{absolute value} (or \term{modulus}) of a
  complex number $z = x + \ii y$ (where $x,y \in \bR$)
  is the nonnegative real number $|z| = \sqrt{x^2 + y^2}$.
\end{definition}

\begin{lemma}[Properties of absolute value \Palka{Sec.~I.1.2~(1.2)}]
  \label{lem:absolute_value_properties}
  \uses{def:absolute_value, def:complex_conjugate}
  %\lean{}
  % \leanok
  For any $z, w \in \bC$, we have
  \begin{align*}
    |z|^2 = z \, \cconj{z} , \qquad
    |z w| = |z| \, |w| ,
  \end{align*}
  \begin{align*}
    \re(z) \le |z| , \qquad
    \im(z) \le |z| ,
  \end{align*}
  \begin{align*}
    |z + w| \le |z| + |w| , \qquad
    |z + w| \ge \big| |z| - |w| \big| .
  \end{align*}
  Also, if $z \ne 0$, then
  \begin{align*}
    z^{-1} = \frac{\cconj{z}}{|z|^2} , \qquad
    \qquad
    \Big| \frac{w}{z} \Big| = \frac{|w|}{|z|} .
  \end{align*}
\end{lemma}
\begin{proof}
  %\uses{}
  % \leanok
  Straightforward.
\end{proof}

\begin{definition}[Argument \Palka{Sec.~I.1.2}]
  \label{def:argument}
  \uses{def:complex_numbers, def:absolute_value}
  %\lean{}
  % \leanok
  A real number $\theta \in \bR$ is an \term{argument} of a complex
  number $z \in \bC$ if
  \begin{align*}
    z = |z| \, \Big( \cos (\theta) + \ii \, \sin (\theta) \Big) .
  \end{align*}

  (Note: the argument of a nonzero complex number is defined only
  modulo addition of integer multiples of $2 \pi$.)

  The \term{principal argument} of a nonzero complex number $z \in \bC$
  is its unique argument on the interval $(-\pi, \pi]$, and it is denoted
  by $\Arg(z)$.
\end{definition}

\begin{lemma}[Multiplication in polar form \Palka{Sec.~I.1.2~(1.6)}]
  \label{lem:multiplication_polar}
  \uses{thm:complex_number_field, def:argument, def:absolute_value}
  %\lean{}
  % \leanok
  For any $z,w \in \bC$, we have
  \begin{align*}
    |zw| = |z| \, |w|
    \qquad \text{ and } \qquad
    \Arg(zw) = \Arg(z) + \Arg(w) .
  \end{align*}
\end{lemma}

\begin{lemma}[De Moivre's formula \Palka{Sec.~I.1.2~(1.7)}]
  \label{lem:de_moivre}
  \uses{thm:complex_number_field}
  %\lean{}
  % \leanok
  For any $\theta \in \bR$ and $n \in \bZ$, we have
  \begin{align*}
    \big( \cos \theta + i \sin \theta \big)^n = \cos(n \theta) + i \sin(n \theta) .
  \end{align*}
\end{lemma}
\begin{proof}
  \uses{lem:multiplication_polar}
  % \leanok
  Straightforward induction using Lemma~\ref{lem:multiplication_polar}.
\end{proof}

%\begin{itemize}
%  \item $n$th roots of unity / of a complex number
%\end{itemize}



\section{Functions of a complex variable}


\subsection{Exponentials and logarithms}

\begin{definition}[Complex exponential]
  \label{def:complex_exp}
  % \uses{}
  % \lean{}
  % \leanok

  foo.
\end{definition}


\subsection{Polynomials and rational functions}

\begin{definition}[Polynomial]
  \label{def:polynomial}
  \uses{thm:complex_number_field}
  %\lean{}
  % \leanok
  Polynomial functions are functions $p : \bC \to \bC$ of the form
  \begin{align*}
    p(z) = a_n z^n + a_{n-1} z^{n-1} + \cdots + a_1 z + a_0
  \end{align*}
  where $a_0,a_1,\ldots,a_{n-1},a_n \in \bC$ are coefficients.
\end{definition}

\begin{definition}[Rational function]
  \label{def:rational_function}
  \uses{def:polynomial}
  %\lean{}
  % \leanok
  Rational functions are functions $f : D \to \bC$ which can be written as
  ratios $f(z) = \frac{p(z)}{q(z)}$ of two polynomials
  $p, q \colon \bC \to \bC$ on a domain $D \subset \bC$
  where the denominator polynomial $q$ has no zeroes.
\end{definition}


\chapter{Complex derivatives and analytic functions}
\section{Real linear maps versus complex linear maps}

The right abstract way of understanding the differential of a function is as a linear
approximation to a function locally. The key difference between real analysis and
complex analysis is whether one uses real linear maps or complex linear maps.

\begin{definition}[Linear map]
  \label{def:linear}
  \uses{}
  %\lean{}
  % \leanok
  Let $\bK$ be a field (for example $\bK = \bR$ or $\bK = \bC$), and let
  $V$ and $W$ be vector spaces over $\bK$. A function $L : V \to W$ is
  said to be \term{$\bK$-linear} if
  \begin{align*}
    L(v_1 + v_2) = \; & L(v_1) + L(v_2) & & \text{ for all } v_1, v_2 \in V, \\
    L(c v) = \; & c \, L(v) & & \text{ for all } v \in V, c \in \bK .
  \end{align*}
  Such a function $L$ is also called a \term{$\bK$-linear map} (or a
  \term{$\bK$-linear transformation}) between the spaces $V$ and $W$.
\end{definition}

The complex plane $\bC \cong \bR^2$ can be seen either as a 2-dimensional real vector space
or as a $1$-dimensional complex vector space. In particular, it makes sense to talk about
both $\bR$-linear maps $\bC \to \bC$ and $\bC$-linear maps $\bC \to \bC$.

More generally, any complex vector space can be seen as a real vector space
(of twice the same dimension), and any complex linear map becomes a real linear map.
The converse is not true! Let us elaborate on this in a simple example which
will soon be seen to pertain to the difference of complex differentiability
and real differentiability.

\textbf{Remark:} Identifying $\bC = \bR^2$ (and choosing basis vectors $1, \ii \in \bC$
for $\bC$ seen as a 2-dimensional vector space), we see that an
$\bR$-linear map $L : \bC \to \bC$ can be encoded in a $2 \times 2$ matrix with real entries,
\begin{align*}
  M = \left[ \begin{array}{cc} a & b \\ c & d \end{array} \right] \in \bR^{2 \times 2}
\end{align*}
in such a way that
\begin{align*}
  L (x + \ii y) = (a x + b y) + \ii (c x + d y) .
\end{align*}

\textbf{Remark:} A $\bC$-linear map $L : \bC \to \bC$ can be encoded in a single complex number
$\lambda \in \bC$ (or more pedantically, in a $1 \times 1$ matrix $[\lambda] \in \bC^{1 \times 1}$),
in such a way that
\begin{align*}
  L z = \lambda z .
\end{align*}
We can write $\lambda = \alpha + \ii \beta$, with
$\alpha = \re(\lambda), \beta = \im(\lambda) \in \bR$. We can also write $z = x + \ii y$
and obtain the expression
\begin{align*}
  L (x + \ii y) = (\alpha + \ii \beta) (x + \ii y)
    = (\alpha x - \beta y) + \ii (\beta x + \alpha y) .
\end{align*}
In other words, seen as a real-linear map, the complex multiplication by $\lambda$
corresponds to the matrix
\begin{align*}
  M = \left[ \begin{array}{cc} \alpha & -\beta \\ \beta & \alpha \end{array} \right] .
\end{align*}
This clearly shows that not every real-linear map $\bC \to \bC$ is complex linear.
It also gives an explicit set of equations for the entries of
the matrix of a real-linear map characterizing complex-linearity, which turn out to be
very closely related to the Cauchy-Riemann equations.

\begin{lemma}[Complex linear versus real linear maps of $\bC$]
  \label{lem:complex_linear_vs_real_linear}
  \uses{def:linear, thm:complex_number_field}
  %\lean{}
  % \leanok
  Let $L : \bC \to \bC$ be a $\bR$-linear map represented in the basis $1,\ii$ by
  the matrix $M = \left[ \begin{array}{cc} a & b \\ c & d \end{array} \right] \in \bR^{2 \times 2}$.
  Then the following are equivalent:
  \begin{itemize}
    \item $L$ is $\bC$-linear;
    \item $b = - c$ and $a = d$.
  \end{itemize}
\end{lemma}
\begin{proof}
  %\uses{}
  % \leanok
  Clear from the above discussion.
\end{proof}



\section{Complex derivative}

\begin{definition}[Complex derivative \Palka{Sec.~III.1.1}]
  \label{def:complex_derivative}
  \uses{thm:complex_number_field, def:limit}
  %\lean{}
  % \leanok
  Let $f \colon A \to \bC$ be a complex-valued function defined on
  a subset $A \subset \bC$ of the complex plane, and let $z_0 \in A$
  be an interior point of the subset.

  The $f$ is said to have
  a \term{complex derivative}
  \begin{align*}
    f'(z_0) := \lim_{z \to z_0} \frac{f(z) - f(z_0)}{z - z_0}
  \end{align*}
  at $z_0$, if the limit on the right hand side above exists.

  (In complex analysis we often drop the epithet ``complex'' above,
  and simply call $f'(z_0)$ the \term{derivative} of $f$ at $z_0$.)
\end{definition}

\begin{lemma}[Local linear approximation]
  \label{lem:local_linear_approximation}
  \uses{def:complex_derivative}
  %\lean{}
  % \leanok
  If a function $f \colon A \to \bC$ has complex derivative
  $f'(z_0) = \lambda \in \bC$ at a point $z_0 \in A$, then
  we can write a linear approximation
  \begin{align*}
    f(z) = f(z_0) + (z - z_0) \, \lambda + \epsilon(z) ,
  \end{align*}
  where the error term $\epsilon$ is small near $z_0$ in the sense that
  $\lim_{z \to z_0} \frac{\epsilon(z)}{|z-z_0|} = 0$.
\end{lemma}
\begin{proof}
  %\uses{}
  % \leanok
  \ldots
\end{proof}

\begin{lemma}[Complex differentiability implies continuity \Palka{Sec.~III.1.1}]
  \label{lem:complex_derivative_implies_continuous}
  \uses{def:continuous, def:complex_derivative}
  %\lean{}
  % \leanok
  If a function $f \colon A \to \bC$ has a complex derivative $f'(z_0)$
  at a point $z_0 \in A$, then it is continuous at $z_0$.
\end{lemma}
\begin{proof}
  \uses{lem:differentiable_implies_continuous}
  % \leanok
  \ldots
\end{proof}



\section{Cauchy-Riemann equations}

\begin{lemma}[Complex derivative implies differentiability]
  \label{lem:complex_derivative_implies_differentiable}
  \uses{def:complex_derivative, def:differential, lem:complex_linear_vs_real_linear}
  %\lean{}
  % \leanok
  Let $f \colon A \to \bC$ be a function defined on a set $A \subset \bC$,
  and let $u \colon U \to \bR$ and $v \colon U \to \bR$ be its
  real and imaginary parts, viewed as real-valued functions of two real variables,
  $u(x,y) = \re \big( f (x + \ii y) \big)$ and
  $v(x,y) = \im \big( f (x + \ii y) \big)$,
  so that $f = u + \ii \, v$.
  If $f$ has a complex derivative $f'(z_0)$ at an interior
  point $z_0 = x_0 + \ii y_0 \in A$, then $u$ and $v$ are differentiable
  at $(x_0,y_0)$ and their partial derivatives satisfy the
  \term{Cauchy-Riemann equations}
  \begin{align*}
    %\pder{x} u(x,y) = \pder{y} v(x,y)
    %\pderof{u}{x} (x,y) = \pderof{v}{y} (x,y)
    \pderof{u}{x}(x_0,y_0) = \pderof{v}{y}(x_0,y_0)
    \quad \text{and} \quad
    %\pder{y} u(x,y) = -\pder{x} v(x,y) .
    %\pderof{u}{y} (x,y) = - \pderof{v}{x} (x,y) .
    \pderof{u}{y}(x_0,y_0) = - \pderof{v}{x}(x_0,y_0).
  \end{align*}

  (These equations are equivalent to the differential
  $\ud f (x_0, y_0) \colon \bR^2 \to \bR^2$
  being $\bC$-linear when we identify $\bR^2 = \bC$.)

  We can then write the derivative at $z_0$ in any of the following ways:
  \begin{align*}
    f'(z_0) \, = \; & \pderof{u}{x}(x_0,y_0) + \ii \, \pderof{v}{x}(x_0,y_0)
    = \pderof{v}{y}(x_0,y_0) - \ii \, \pderof{u}{y}(x_0,y_0) \\
    = \; & \pderof{u}{x}(x_0,y_0) - \ii \, \pderof{u}{y}(x_0,y_0)
    = \pderof{v}{y}(x_0,y_0) + \ii \, \pderof{v}{x}(x_0,y_0) .
  \end{align*}

\end{lemma}
\begin{proof}
  \uses{lem:local_linear_approximation, lem:jacobian_matrix}
  % \leanok
  \ldots
\end{proof}



\subsection{Differentiation rules}

\begin{lemma}[Linearity of the derivative \Palka{Sec.~III.1.2~(3.4)}]
  \label{lem:derivative_linearity}
  \uses{def:complex_derivative, def:linear}
  %\lean{}
  % \leanok
  If two functions $f, g \colon A \to \bC$ have complex derivatives
  $f'(z_0), g'(z_0)$ at a point $z_0 \in A$, then the sum function $f + g$
  has a complex derivative at $z_0$ given by
  \begin{align*}
    (f + g)'(z_0) = f'(z_0) + g'(z_0) .
  \end{align*}
  If a function $f \colon A \to \bC$ is has a complex derivative
  $f'(z_0)$ at a point $z_0 \in A$ and $c \in \bC$ is a complex number,
  then the function $c f$ has complex derivative
  \begin{align*}
    (c f)'(z_0) = c \, f'(z_0)
  \end{align*}
  at $z_0$.
\end{lemma}
\begin{proof}
  \uses{lem:complex_limit_properties}
  % \leanok
  \ldots
\end{proof}

\begin{lemma}[Leibniz rule \Palka{Sec.~III.1.2~(3.4)}]
  \label{lem:leibniz_rule}
  \uses{def:complex_derivative}
  %\lean{}
  % \leanok
  If two functions $f, g \colon A \to \bC$ have complex derivatives
  $f'(z_0), g'(z_0)$ at a point $z_0 \in A$, then the product function $f g$
  has complex derivative
  \begin{align*}
    (f g)'(z_0) = f'(z_0) \, g(z_0) + f(z_0) \, g'(z_0)
  \end{align*}
  at $z_0$.
\end{lemma}
\begin{proof}
  %\uses{}
  % \leanok
  \ldots
\end{proof}

\begin{lemma}[Derivative of a quotient \Palka{Sec.~III.1.2~(3.4)}]
  \label{lem:quotient_derivative}
  \uses{def:complex_derivative}
  %\lean{}
  % \leanok
  If two functions $f, g \colon A \to \bC$ have complex derivatives
  $f'(z_0), g'(z_0)$ at a point $z_0 \in A$ and $g(z_0) \ne 0$, then
  the quotient function $f / g$ has complex derivative
  \begin{align*}
    \left( \frac{f}{g} \right)'(z_0)
      = \frac{f'(z_0) \, g(z_0) - f(z_0) \, g'(z_0)}{g(z_0)^2} .
  \end{align*}
  at $z_0$.
\end{lemma}
\begin{proof}
  %\uses{}
  % \leanok
  \ldots
\end{proof}

\begin{lemma}[Chain rule \Palka{Thm~III.1.1}]
  \label{lem:chain_rule}
  \uses{def:complex_derivative}
  %\lean{}
  % \leanok
  If $f \colon A \to B \subset \bC$ is differentiable at $z_0 \in A$
  and $g \colon B \to \bC$ is differentiable at $f(z_0) \in B$,
  then the composition $g \circ f \colon A \to \bC$ is differentiable at $z_0$,
  with derivative
  \begin{align*}
    (g \circ f)'(z_0) = f'(z_0) \; g'\big( f(z_0) \big) .
  \end{align*}
\end{lemma}
\begin{proof}
  %\uses{}
  % \leanok
  \ldots
\end{proof}

\begin{lemma}[Derivative of inverse \Palka{Thm~III.4.1}]
  \label{lem:inverse_derivative}
  \uses{def:complex_derivative}
  %\lean{}
  % \leanok
  \ldots (with $w_0 = f(z_0)$)
  \begin{align*}
    (f^{-1})'(w_0) = \frac{1}{f'(z_0)} .
  \end{align*}
\end{lemma}
\begin{proof}
  \uses{lem:chain_rule}
  % \leanok
  \ldots
\end{proof}


\subsection{Analytic functions}

\begin{definition}[Analytic function \Palka{Sec.~III.1.3}]
  \label{def:analytic_function}
  \uses{def:complex_derivative, def:open_set}
  %\lean{}
  % \leanok
  A function $f \colon U \to \bC$ defined on an open set $U \subset \bC$ is
  said to be \term{analytic} (or \term{holomorphic}) if it is complex
  differentiable at every point $z_0 \in U$.
\end{definition}

\begin{theorem}[Cauchy-Riemann equations \Palka{Thm~III.2.2}]
  \label{thm:cauchy_riemann}
  \uses{def:analytic_function}
  %\lean{}
  % \leanok
  Let $f \colon U \to \bC$ be a function defined on an open set $U \subset \bC$,
  and let $u \colon U \to \bR$ and $v \colon U \to \bR$ be its
  real and imaginary parts, viewed as real-valued functions of two real variables,
  \begin{align*}
    u(x,y) = \re \Big( f (x + \ii y) \Big)
    \quad \text{and} \quad
    v(x,y) = \im \Big( f (x + \ii y) \Big)
  \end{align*}
  so that $f = u + \ii \, v$.

  Then the following are equivalent:
  \begin{itemize}
    \item The functions $u$ and $v$ are differentiable at every point in $U$
          and their partial derivatives satisfy the \term{Cauchy-Riemann equations}
          \begin{align*}
            %\pder{x} u(x,y) = \pder{y} v(x,y)
            %\pderof{u}{x} (x,y) = \pderof{v}{y} (x,y)
            \pderof{u}{x} = \pderof{v}{y}
            \quad \text{and} \quad
            %\pder{y} u(x,y) = -\pder{x} v(x,y) .
            %\pderof{u}{y} (x,y) = - \pderof{v}{x} (x,y) .
            \pderof{u}{y} = - \pderof{v}{x}
          \end{align*}
          in $U$.
    \item The function $f$ is analytic.
  \end{itemize}
  %Then we have:
%
  %\textbf{(i):} If $f$ is analytic, then $u$ and $v$ are real-differentiable
  %at every point $(x,y) \in U$ and their partial derivatives satisfy
  %the \term{Cauchy-Riemann equations}
  %\begin{align*}
  %  %\pder{x} u(x,y) = \pder{y} v(x,y)
  %  %\pderof{u}{x} (x,y) = \pderof{v}{y} (x,y)
  %  \pderof{u}{x} = \pderof{v}{y}
  %  \quad \text{and} \quad
  %  %\pder{y} u(x,y) = -\pder{x} v(x,y) .
  %  %\pderof{u}{y} (x,y) = - \pderof{v}{x} (x,y) .
  %  \pderof{u}{y} = - \pderof{v}{x} .
  %\end{align*}
%
  %\textbf{(ii):} If $u$ and $v$ are continuously real-differentiable
  %in all of $U$ and their partial derivatives satisfy
  %the Cauchy-Riemann equations above, then $f$ is analytic.
\end{theorem}
\begin{proof}
  \uses{lem:complex_derivative_implies_differentiable}
  % \leanok
  \ldots
\end{proof}

\begin{lemma}[Analytic functions are continuous]
  \label{lem:analytic_implies_continuous}
  \uses{def:analytic_function, def:continuous}
  % \lean{}
  % \leanok
  Every function $f \colon U \to \bC$ which is analytic on
  an open set $U \subset \bC$ is also continuous on $U$.
\end{lemma}
\begin{proof}
  \uses{lem:complex_derivative_implies_continuous}
  % \leanok
  \ldots
\end{proof}

\begin{lemma}[Polynomials are analytic]
  \label{lem:polynomial_analytic}
  \uses{def:polynomial, def:analytic_function}
  %\lean{}
  % \leanok
  Every polynomial function $p \colon \bC \to \bC$ is analytic.
\end{lemma}
\begin{proof}
  \uses{lem:derivative_linearity, lem:leibniz_rule}
  % \leanok
  \ldots
\end{proof}

\begin{lemma}[Rational functions are analytic]
  \label{lem:rational_function_analytic}
  \uses{def:rational_function, def:analytic_function}
  %\lean{}
  % \leanok
  Every rational function $f \colon U \to \bC$ is analytic
  on its domain of definition $U \subset \bC$.
\end{lemma}
\begin{proof}
  \uses{lem:polynomial_analytic, lem:quotient_derivative}
  % \leanok
  \ldots
\end{proof}

\begin{lemma}[The complex exponential is analytic]
  \label{lem:exp_analytic}
  \uses{def:complex_exp, def:analytic_function}
  %\lean{}
  % \leanok
  The complex exponential function $\exp \colon \bC \to \bC$ is analytic.
  Its (complex) derivative at $z \in \bC$ is $\exp'(z) = \exp(z)$.
\end{lemma}
\begin{proof}
  \uses{thm:cauchy_riemann}
  % \leanok
  \ldots
\end{proof}

\begin{lemma}[Branches of $n$th root functions are analytic]
  \label{lem:root_function_analytic}
  \uses{def:analytic_function, def:principal_complex_nth_root}
  % \lean{}
  % \leanok
  The principal branch of the $n$th root function
  $z \mapsto \sqrt[n]{z}$ is analytic on its domain
  $\bC \setminus (-\infty,0]$.

  (Different branch choices can be made to obtain analyticity on
  other domains, but for $n \ge 2$,
  no branch of $\sqrt[n]{z}$ can be made analytic on all of $\bC$.)
\end{lemma}
\begin{proof}
  \uses{lem:inverse_derivative}
  % \leanok
  \ldots
\end{proof}



\subsection{Consequences of Cauchy-Riemann equations}

\begin{lemma}[Analytic functions of vanishing derivative]
  \label{lem:zero_derivative_implies_constant}
  \uses{def:analytic_function}
  %\lean{}
  % \leanok
  Suppose that $f \colon D \to \bC$ is a analytic function
  on a domain $D \subset \bC$ (i.e. a connected open set)
  such that $f'(z) = 0$ for all $z \in D$.
  Then $f$ is a constant function.
\end{lemma}
\begin{proof}
  \uses{thm:cauchy_riemann, lem:vanishing_partial_derivatives_implies_constant}
  % \leanok
  \ldots
\end{proof}

\begin{theorem}[Criteria for constantness of a analytic function]
  \label{thm:constantness_criteria}
  \uses{def:analytic_function}
  %\lean{}
  % \leanok
  Suppose that $f \colon D \to \bC$ is a analytic function
  on a domain $D \subset \bC$ (i.e. a connected open set).
  If any of the functions $u = \re(f) \colon D \to \bR$,
  $v = \im(f) \colon D \to \bR$, $|f| \colon D \to \bR$,
  is constant on $D$, then $f$ is itself a constant function.
\end{theorem}
\begin{proof}
  \uses{thm:cauchy_riemann, lem:zero_derivative_implies_constant}
  % \leanok
  \ldots
\end{proof}


\chapter{Contour integration}
\section{Paths in the complex plane}

\begin{definition}[Path \Palka{Sec.~IV.1.1}]
  \label{def:path}
  \uses{def:continuous}
  %\lean{}
  % \leanok
  A \term{path} in the complex plane is a continuous function $\gamma \colon [a,b] \to \bC$
  from a closed interval $[a,b] \subset \bR$ to $\bC$.

  When $A \subset \bC$ is a subset of the complex plane, we say that $\gamma$ is a path in $A$
  if $\gamma(t) \in A$ for all $t \in [a,b]$.
\end{definition}

\begin{definition}[Smooth path \Palka{Sec.~IV.1.2}]
  \label{def:smooth_path}
  \uses{def:path, def:continuous}
  %\lean{}
  % \leanok
  A path $\gamma \colon [a,b] \to \bC$ is \term{smooth}
  if it is continuously differentiable, i.e., the derivative
  \begin{align*}
    \dot{\gamma}(t) = \frac{\ud}{\ud t} \gamma(t)
  \end{align*}
  with respect to the parameter $t$
  exists for all $t \in [a,b]$ (one-sided derivative at the end points),
  and defines a continuous complex-valued function $t \mapsto \dot{\gamma}(t)$
  on $[a,b]$.
\end{definition}

\begin{definition}[Contour / piecewise smooth path \Palka{Sec.~IV.1.2}]
  \label{def:contour}
  \uses{def:smooth_path, def:continuous}
  %\lean{}
  % \leanok
  A \term{contour} (also called a \term{piecewise smooth path})
  is a continuous function $\gamma \colon [a,b] \to \bC$
  such that for some finite subdivision $a = t_0 < t_1 < \ldots < t_n = b$,
  the restrictions $\gamma|_{[t_{j-1},t_j]}$, $j = 1, \ldots, n$ are smooth paths.
\end{definition}

\begin{definition}[Reverse path \Palka{Sec.~IV.1.4}]
  \label{def:reverse_path}
  \uses{def:path}
  %\lean{}
  % \leanok
  Given a path $\gamma \colon [a,b] \to \bC$,
  the \term{reverse path} $\opp{\gamma} \colon [a,b] \to \bC$
  is the path defined by
  \begin{align*}
    \opp{\gamma}(t) = \gamma(a+b-t) \qquad \text{ for } t \in [a,b] .
  \end{align*}
\end{definition}

\begin{definition}[Concatenation of paths \Palka{Sec.~IV.1.4}]
  \label{def:path_concatenation}
  \uses{def:path}
  %\lean{}
  % \leanok
  Given path $\gamma \colon [a,b] \to \bC$ and
  $\eta \colon [c,d] \to \bC$ with $\gamma(b) = \eta(c)$
  (the starting point of $\eta$ coincides with the end point of $\gamma$),
  the \term{concatenation} of $\gamma$ and $\eta$ is the path
  $\gamma \concat \eta \colon [a,b+d-c] \to \bC$ defined by
  \begin{align*}
    (\gamma \concat \eta)(t) =
    \begin{cases}
      \gamma(t) & \text{ for } t \in [a,b] , \\
      \eta(c+t-b)) & \text{ for } t \in [b,b+d-c] .
    \end{cases}
  \end{align*}

  (The slightly cumbersome formula in the second case is due to the fact
  that we need to attach the two parameter intervals of lengths
  $b-a$ and $d-c$ to each other, and we have, somewhat arbitrarily,
  chosen to glue them to form the interval $[a,b+d-c]$.)
\end{definition}

\begin{definition}[Reparametrization of paths \Palka{Sec.~IV.1.5}]
  \label{def:path_reparametrization}
  \uses{def:path, lem:continuous_bijection_homeomorphism, def:contour}
  %\lean{}
  % \leanok
  Given a path $\gamma \colon [a,b] \to \bC$ and a continuous increasing
  bijection $\phi \colon [c,d] \to [a,b]$, we define the \term{reparametrization}
  of $\gamma$ by $\phi$ as the path
  \begin{align*}
    \gamma \circ \phi \colon \, [c,d] \, \to \; & \bC
    t \, \mapsto \; & \gamma(\phi(t)) .
  \end{align*}

  Note that
  \begin{itemize}
    \item $\phi^{-1} \colon [a,b] \to [c,d]$ is also a continuous increasing
      bijection (a continuous bijection from the
      compact $[a,b]$ is automatically a homeomorphism; see ) and
      reparametrization can be undone by rereparametrizing by $\phi^{-1}$;
    \item If both $\gamma$ and the reparametrization function $\phi$ are
      piecewise smooth, then the reparametrized path $\gamma \circ \phi$
      is also piecewise smooth (i.e., a contour).
  \end{itemize}
\end{definition}



\section{Complex-valued integrals}

\begin{definition}[Integral of a complex-valued function]
  \label{def:complex_valued_integral}
  \uses{def:riemann_integral, lem:continuous_complex, lem:continuous_implies_riemann_integrable}
  % \lean{}
  % \leanok
  Let $f \colon [a,b] \to \bC$ be a complex-valued continuous function
  defined on a closed interval $[a,b] \subset \bR$. We define the
  \term{integral} of $f$ as
  \begin{align*}
    \int_a^b f(t) \, \ud t
      \; = \; \int_a^b \re \big( f(t) \big) \, \ud t
              \; + \; \ii \int_a^b \im \big( f(t) \big) \, \ud t .
  \end{align*}

  (Note that on the right hand side we have just Riemann integrals
  of the continuous real-valued functions $t \mapsto \re \big( f(t) \big)$
  and $t \mapsto \im \big( f(t) \big)$.)
\end{definition}


\section{Integrals along paths}

\begin{definition}[Contour integral along a smooth path \Palka{Sec.~IV.2.1}]
  \label{def:smooth_contour_integral}
  \uses{def:complex_valued_integral, def:smooth_path}
  %\lean{}
  % \leanok
  Let $f : A \to \bC$ be a continuous function defined on a subset $A \subset \bC$.
  %and write $u = \re f, v = \im f \colon [a,b] \to \bR$
  %for its real and imaginary parts so that $f(z) = u(z) + \ii v(z)$ for $z \in A$.
  Let $\gamma \colon [a,b] \to A$ be a smooth path in~$A$.
  We define the \term{integral of $f$ along $\gamma$} as
  \begin{align*}
    \int_\gamma f(z) \, dz
    \; = \; \int_a^b f \big( \gamma(t) \big) \, \dot{\gamma}(t) \; \ud t .
    %= \; \int_a^b u \big( \gamma(t) \big) \gamma'(t) \, \ud t
    %     + \ii \int_a^b v \big( \gamma(t) \big) \gamma'(t) \, \ud t .
  \end{align*}

  (Here $\dot{\gamma}(t) = \frac{\ud}{\ud t} \gamma(t)$ denotes the derivative of
  the smooth path $\gamma$ with respect to its parameter $t$.)
\end{definition}

Sometimes it is appropriate to integrate functions with respect to the
arc-length in the following sense.
\begin{definition}[Arc-length integral along a smooth path \Palka{Sec.~IV.2.1}]
  \label{def:smooth_arc_length_integral}
  \uses{def:complex_valued_integral, lem:continuous_implies_riemann_integrable, def:smooth_path}
  %\lean{}
  % \leanok
  Let $f : A \to \bC$ be a continuous function defined on a subset $A \subset \bC$.
  %and write $u = \re f, v = \im f \colon [a,b] \to \bR$
  %for its real and imaginary parts so that $f(z) = u(z) + \ii v(z)$ for $z \in A$.
  Let $\gamma \colon [a,b] \to A$ be a smooth path in~$A$.
  We define the \term{integral of $f$ with respect to the arc length of $\gamma$} as
  \begin{align*}
    \int_\gamma f(z) \, |dz|
    \; = \; \int_a^b f \big( \gamma(t) \big) \, |\gamma'(t)| \; \ud t .
    %= \; \int_a^b u \big( \gamma(t) \big) \gamma'(t) \, \ud t
    %     + \ii \int_a^b v \big( \gamma(t) \big) \gamma'(t) \, \ud t .
  \end{align*}
\end{definition}

In order to extend the definition of contour integrals to piecewise smooth paths,
we note that the definition behaves additively under path concatenation.

\begin{lemma}[Contour integrals and smooth path concatenation \Palka{Lem~IV.2.1(iv)}]
  \label{lem:integral_concatenation}
  \uses{def:smooth_contour_integral, def:smooth_arc_length_integral, def:path_concatenation}
  %\lean{}
  % \leanok
  If a smooth path $\gamma$ in $A$ is a concatenation of smooth
  paths $\eta_1, \ldots, \eta_n$, and
  $f : A \to \bC$ is a continuous function defined on $A \subset \bC$,
  then we have
  \begin{align*}
    \int_\gamma f(z) \, \ud z \; = \; \sum_{j=1}^n \int_{\eta_j} f(z) \, \ud z
  \end{align*}
  and
  \begin{align*}
    \int_\gamma f(z) \, |\ud z| \; = \; \sum_{j=1}^n \int_{\eta_j} f(z) \, |\ud z| .
  \end{align*}
\end{lemma}
\begin{proof}
  %\uses{}
  % \leanok
  %Straightforward from a corresponding property of Riemann integrals.
  \ldots
\end{proof}

By virtue of the above, the following gives a well-defined meaning
to integrals along piecewise smooth paths.
\begin{definition}[Contour integral along a path \Palka{Sec.~IV.2.1}]
  \label{def:contour_integral}
  \uses{def:smooth_contour_integral, def:contour, lem:integral_concatenation}
  %\lean{}
  % \leanok
  Let $f : A \to \bC$ be a continuous function defined on a subset $A \subset \bC$.
  Let $\gamma \colon [a,b] \to A$ be a piecewise smooth path in~$A$,
  which is a concatenation of smooth paths $\eta_1, \ldots, \eta_n$.
  We define the \term{integral of $f$ along $\gamma$} as
  \begin{align*}
    \int_\gamma f(z) \, dz \; = \; \sum_{j=1}^n \int_{\eta_j} f(z) \, dz .
  \end{align*}
\end{definition}
\begin{definition}[Arc-length integral along a path]
  \label{def:arc_length_integral}
  \uses{def:smooth_arc_length_integral, def:contour, lem:integral_concatenation}
  %\lean{}
  % \leanok
  Let $f : A \to \bC$ be a continuous function defined on a subset $A \subset \bC$.
  Let $\gamma \colon [a,b] \to A$ be a piecewise smooth path in~$A$,
  which is a concatenation of smooth paths $\eta_1, \ldots, \eta_n$.
  We define the \term{integral of $f$ along $\gamma$} as
  \begin{align*}
    \int_\gamma f(z) \, dz \; = \; \sum_{j=1}^n \int_{\eta_j} f(z) \, dz .
  \end{align*}
\end{definition}

\begin{lemma}[Reparametrization invariance of integrals \Palka{Lem~IV.2.1(v)}]
  \label{lem:reparametrization_invariance_of_integral}
  \uses{def:contour_integral, def:arc_length_integral, def:path_reparametrization}
  %\lean{}
  % \leanok
  Let $\gamma$ be a piecewise smooth path in~$A$,
  and let $\widetilde{\gamma}$ be obtained from $\gamma$ by an
  orientation-preserving reparametrization. Then for any continuous
  function $f : A \to \bC$ we have
  \begin{align*}
    \int_{\widetilde{\gamma}} f(z) \, dz \; = \; \int_{\gamma} f(z) \, dz
  \end{align*}
  and
  \begin{align*}
    \int_{\widetilde{\gamma}} f(z) \, |dz| \; = \; \int_{\gamma} f(z) \, |dz| .
  \end{align*}
\end{lemma}

\begin{lemma}[Contour integrals and path reversal \Palka{Lem~IV.2.1(iii)}]
  \label{lem:integral_reversal}
  \uses{def:contour_integral, def:arc_length_integral, def:reverse_path}
  %\lean{}
  % \leanok
  If $f : A \to \bC$ is a continuous function defined on $A \subset \bC$,
  and $\gamma$ is a piecewise path in $A$, then for the contour integral
  and the arc length integral behave as follows under path reversal:
  then we have
  \begin{align*}
    \int_{\opp{\gamma}} f(z) \, \ud z \; = \; - \int_{\gamma} f(z) \, \ud z
  \end{align*}
  and
  \begin{align*}
    \int_{\opp{\gamma}} f(z) \, |\ud z| \; = \; \int_{\gamma} f(z) \, |\ud z|
  \end{align*}
\end{lemma}
\begin{proof}
  %\uses{}
  % \leanok
  %Straightforward from a corresponding property of Riemann integrals.
  \ldots
\end{proof}

\begin{lemma}[Linearity of integrals \Palka{Lem~IV.2.1(i-ii)}]
  \label{lem:integral_linearity}
  \uses{def:contour_integral, def:arc_length_integral, def:linear}
  %\lean{}
  % \leanok
  \begin{align*}
    \int_\gamma \Big( f_1(z) + f_2(z) \Big) \, \ud z
    \; = \; \int_\gamma f_1(z) \, \ud z + \int_\gamma f_2(z) \, \ud z
  \end{align*}
  \begin{align*}
    \int_\gamma c \, f(z) \, \ud z \; = \; c \int_\gamma f(z) \, \ud z .
  \end{align*}
\end{lemma}
\begin{proof}
  %\uses{}
  % \leanok
  \ldots
\end{proof}

\begin{lemma}[Triangle inequality for contour integrals \Palka{Lem~IV.2.1(vi)}]
  \label{lem:integral_triangle_inequality}
  \uses{def:contour_integral, def:arc_length_integral}
  %\lean{}
  % \leanok
  \begin{align*}
    \left| \int_\gamma f(z) \, \ud z \right| \; \le \; \int_\gamma |f(z)| \, |dz| .
  \end{align*}
\end{lemma}
\begin{proof}
  %\uses{}
  % \leanok
  \ldots
\end{proof}



\section{Primitives}

\begin{definition}[Primitive of a function \Palka{Sec.~IV.2.3}]
  \label{def:primitive}
  \uses{def:analytic_function}
  %\lean{}
  % \leanok
  Let $f : U \to \bC$ be a function defined on an open subset $U \subset \bC$.
  A \term{primitive} of $f$ is a function $F : U \to \bC$ such that
  $F$ is holomorphic (i.e., complex differentiable) on $U$,
  \begin{align*}
    F'(z) = f(z) \qquad \text{ for all } z \in U .
  \end{align*}
\end{definition}

\begin{theorem}[Fundamental theorem of calculus for contour integrals \Palka{Thm~IV.2.2}]
  \label{thm:ftc_for_contour_integrals}
  \uses{def:primitive, def:contour_integral}
  %\lean{}
  % \leanok
  Suppose that $f \colon U \to \bC$ is a continuous function
  on an open set $U \subset \bC$, and that $f$ has a
  primitive $F \colon U \to \bC$.
  Then for any piecewise smooth path $\gamma \colon [a,b] \to U$ we have
  \begin{align*}
    \int_\gamma f(z) \, dz \; = \; F \big( \gamma(b) \big) - F \big( \gamma(a) \big) .
  \end{align*}
  In particular for any closed path $\gamma$ in $U$, we have
  \begin{align*}
    \oint_\gamma f(z) \, dz \; = \; 0 .
  \end{align*}
\end{theorem}


\chapter{Cauchy's theorem and consequences}
\section{Cauchy's integral theorem}

\begin{lemma}[Goursat's lemma \Palka{Lem~V.1.1}]
  \label{lem:goursat}
  \uses{def:analytic_function, def:contour_integral}
  \lean{}
  % \leanok
  Suppose that a function $f \colon U \to \bC$ is analytic on an open
  set $U \subset \bC$.
  Then for any closed triangle $\Tri \subset U$, we have
  \begin{align*}
    \oint_{\bdry \Tri} f(z) \, \ud z = 0 .
  \end{align*}
\end{lemma}

\begin{theorem}[Cauchy's integral theorem \Palka{Thm~V.1.5}]
  \label{thm:cauchy_theorem}
  \uses{def:analytic_function, def:contour_integral}
  %\lean{}
  % \leanok
  Suppose that a function $f \colon U \to \bC$ is analytic on an open
  set $U \subset \bC$ containing the disk $\DiskRC{r}{z} \subset U$.
  Then for any closed contour $\gamma$ in $\DiskRC{r}{z}$
  and $n \in \bN$ we have
  \begin{align*}
    \oint_\gamma f(z) \, \ud z = 0 .
  \end{align*}
\end{theorem}
\begin{proof}
  \uses{lem:goursat}
  % \leanok
  \ldots
\end{proof}



\section{Cauchy's integral formula}

%def:winding_number

\begin{theorem}[Cauchy's integral formula \Palka{Thm~V.2.3}]
  \label{thm:cauchy_formula}
  \uses{def:analytic_function, def:contour_integral}
  %\lean{}
  % \leanok
  Suppose that a function $f \colon U \to \bC$ is analytic on an open
  set $U \subset \bC$ containing the disk $\DiskRC{r}{z} \subset U$.
  Then for any closed contour $\gamma$ in $\DiskRC{r}{z}$
  and $n \in \bN$ we have
  \begin{align*}
    \windN{\gamma}{z} \, f(z)
      = \frac{1}{2\pi \ii} \oint_\gamma \frac{f(\zeta)}{\zeta - z} \, \ud \zeta .
  \end{align*}
\end{theorem}
\begin{proof}
  \uses{thm:cauchy_theorem}
  % \leanok
  \ldots
\end{proof}

\section{Analyticity of derivatives}

\begin{lemma}[Analyticity of derivatives \Palka{Thm~V.3.1}]
  \label{lem:derivative_analytic}
  \uses{def:analytic_function}
  %\lean{}
  % \leanok
  If a function $f \colon U \to \bC$ is analytic on an open set $U \subset \bC$,
  then its derivative $f'$ is also analytic on $U$.
  In particular, then $f$ is continuously differentiable,
  $f \in \ContDiff{1}(U)$.
\end{lemma}
\begin{proof}
  \uses{thm:CAUCHY_FORMULA, lem:derivative_auxiliary_integral}%\uses{lem:lemma_1_6}
  % \leanok
  \ldots
\end{proof}

\begin{corollary}[Analyticity of higher derivatives \Palka{Cor~V.3.2}]
  \label{lem:higher_derivatives_analytic}
  \uses{def:analytic_function}
  %\lean{}
  % \leanok
  If a function $f \colon U \to \bC$ is analytic on an open set $U \subset \bC$,
  then all its derivatives $f', f'', \ldots, f^{(k)}, \ldots$ are also analytic on $U$.
  In particular, then $f$ is infinitely differentiable,
  $f \in \Smooth(U)$.
\end{corollary}
\begin{proof}
  \uses{lem:derivative_analytic, lem:derivative_auxiliary_integral}
  % \leanok
  Straightforward induction using Lemma~\ref{lem:derivative_analytic}.
\end{proof}

\begin{theorem}[Morera's theorem \Palka{Thm~V.3.3}]
  \label{thm:morera}
  \uses{def:analytic_function}
  %\lean{}
  % \leanok
  Let $f \colon U \to \bC$ be a continuous function on an open set $U \subset \bC$.
  If $f$ has the property that
  \begin{align*}
    \oint_{\bdry \Tri} f(z) \, \ud z = 0
  \end{align*}
  for any closed triangle $\Tri \subset U$, then $f$ is analytic on $U$.
\end{theorem}
\begin{proof}
  \uses{def:primitive, lem:derivative_analytic}
  % \leanok
  \ldots
\end{proof}

\begin{theorem}[Cauchy's integral formula for derivatives]
  \label{thm:cauchy_formula_derivative}
  \uses{def:analytic_function}
  %\lean{}
  % \leanok
  Suppose that a function $f \colon U \to \bC$ is analytic on an open
  set $U \subset \bC$ containing the disk $\DiskRC{r}{z} \subset U$.
  Then for any closed contour $\gamma$ in $\DiskRC{r}{z}$
  and $n \in \bN$ we have
  \begin{align*}
    \windN{\gamma}{z} \, f^{(n)}(z)
      = \frac{n!}{2\pi \ii} \oint_\gamma \frac{f(\zeta)}{(\zeta - z)^{n+1}} \, \ud \zeta .
  \end{align*}
\end{theorem}
\begin{proof}
  \uses{thm:CAUCHY_FORMULA, lem:derivative_auxiliary_integral}
  % \leanok
  \ldots
\end{proof}

\begin{lemma}[Cauchy's estimate for derivatives \Palka{Thm~V.3.6}]
  \label{lem:cauchy_derivative_estimate}
  \uses{def:analytic_function}
  %\lean{}
  % \leanok
  Suppose that a function $f \colon U \to \bC$ is analytic on an open
  set $U \subset \bC$ containing the disk $\DiskRC{r}{z_0} \subset U$,
  and suppose that there exists a constant $M>0$ such that
  $|f(z)| \le M$ for all $z \in \DiskRC{r}{z_0}$.
  Then for any $n \in \bN$ and any $z \in \DiskRC{r}{z_0}$ we have
  the following bound for the $n$th derivative $f^{(n)}$ of $f$:
  \begin{align*}
    \big| f^{(n)}(z) \big| \le \frac{n! \, M \, r}{( r - |z-z_0| )^{n+1}} .
  \end{align*}
  In particular, for the center point $z_0$ of the disk, we have
  \begin{align*}
    \big| f^{(n)}(z_0) \big| \le n! \, M \, r^{-n} .
  \end{align*}
\end{lemma}
\begin{proof}
  \uses{thm:cauchy_formula_derivative, lem:integral_triangle_inequality}
  % \leanok
  \ldots
\end{proof}



\section{Liouville's theorem}

\begin{theorem}[Liouville's theorem \Palka{Thm~V.3.7}]
  \label{thm:liouville}
  \uses{def:analytic_function}
  %\lean{}
  % \leanok
  If a function $f \colon \bC \to \bC$ on the entire complex plane
  is analytic and bounded, then $f$ is a constant function.
\end{theorem}
\begin{proof}
  \uses{lem:cauchy_derivative_estimate}
  % \leanok
  \ldots
\end{proof}



\section{The fundamental theorem of algebra}

\begin{theorem}[Fundamental theorem of algebra \Palka{Thm~V.3.8}]
  \label{thm:FUNDAMENTAL_THEOREM_OF_ALGEBRA}
  \uses{def:polynomial}
  %\lean{}
  % \leanok
  Every non-constant polynomial function $p \colon \bC \to \bC$
  has a root, i.e., there exists a $z_0 \in \bC$ such that $p(z_0) = 0$.
\end{theorem}
\begin{proof}
  \uses{thm:liouville, lem:polynomial_analytic}
  % \leanok
  \ldots
\end{proof}

\begin{corollary}[Fundamental theorem of algebra \Palka{Thm~V.3.9}]
  \label{cor:complex_polynomial_factorization}
  \uses{def:polynomial}
  %\lean{}
  % \leanok
  A complex-coefficient
  polynomial $p(z) = a_0 + a_1 z + a_2 z^2 + \cdots + a_n z^n$
  of degree $n \in \bN$ can be factored as
  \begin{align*}
    p(z) = c \, (z - z_1) \, (z - z_2) \, \cdots \, (z - z_n)
  \end{align*}
  where $c = a_n \ne 0$, and $z_1 , \ldots, z_n \in \bC$
  are the roots of $p$ (with repetition according to the
  multiplicities of the roots).
\end{corollary}
\begin{proof}
  \uses{thm:FUNDAMENTAL_THEOREM_OF_ALGEBRA}
  % \leanok
  This follows from Theorem~\ref{thm:fundamental_theorem_of_algebra}
  by induction on the degree of the polynomial,
  using the polynomial division (Euclidean algorithm in the
  ring of univariate polynomials, see \textbf{MS-C1081 Abstract Algebra}).
\end{proof}



\section{Maximum principle}

\begin{theorem}[Maximum principle for analytic functions \Palka{Thm~V.3.10}]
  \label{thm:maximum_principle}
  \uses{def:analytic_function, def:connected}
  %\lean{}
  % \leanok
  Let $f \colon \Domain \to \bC$ be an analytic function on a
  connected open set $\Domain \subset \bC$.
  Suppose that there exists a point $z_0 \in \Domain$
  such that
  \begin{align*}
    |f(z)| \le |f(z_0)| \qquad \text{ for all } z \in \Domain .
  \end{align*}
  Then $f$ is a constant function.
\end{theorem}
\begin{proof}
  \uses{thm:CAUCHY_FORMULA}
  % \leanok
  \ldots
\end{proof}

\begin{corollary}[Maximum principle for analytic functions continuous up to the boundary \Palka{Cor~V.3.11}]
  \label{cor:maximum_modulus_on_boundary}
  \uses{def:analytic_function, def:connected}
  %\lean{}
  % \leanok
  Let $\Domain \subset \bC$ be a bounded connected open set.
  Let $f \colon \closOf{\Domain} \to \bC$ be a continuous function
  on its closure which is analytic in $\Domain$.
  Then $z \mapsto |f(z)|$ attains its maximum in $\closOf{\Domain}$
  at some point of the boundary $\bdry \Domain$.
\end{corollary}
\begin{proof}
  \uses{thm:maximum_principle}
  % \leanok
  \ldots
\end{proof}

\begin{lemma}[Schwarz's lemma \Palka{Thm~V.3.14}]
  \label{lem:schwarz_lemma}
  \uses{def:analytic_function}
  %\lean{}
  % \leanok
  Let $f \colon \DiskRC{1}{0} \to \bC$ be an analytic function
  on the open unit disk such that $|f(z)| \le 1$
  for all $z \in \DiskRC{1}{0}$ and $f(0) = 0$.
  Then we have
  \begin{align*}
    |f'(0)| \le 1
    \qquad \text{ and } \qquad
    |f(z)| \le |z| \quad \text{ for all } z \in \DiskRC{1}{0} .
  \end{align*}
  Furthermore, unless $f$ is of the form $f(z) = \lambda z$ for
  some $\lambda \in \bC$ with $|\lambda| = 1$, then we have
  \begin{align*}
    |f'(0)| < 1
    \qquad \text{ and } \qquad
    |f(z)| < |z| \quad \text{ for all } z \in \DiskRC{1}{0} \setminus \set{0}.
  \end{align*}
\end{lemma}
\begin{proof}
  \uses{cor:maximum_modulus_on_boundary}
  % \leanok
  \ldots
\end{proof}



\section{The mean value property}

\begin{theorem}[Mean value property for analytic functions]
  \label{thm:mean_value_property}
  \uses{def:analytic_function}
  %\lean{}
  % \leanok
  Suppose that a function $f \colon U \to \bC$ is analytic on an open
  set $U \subset \bC$ containing the closed disk $\CDiskRC{r}{z} \subset U$.
  Then we have
  \begin{align*}
    f(z) = \frac{1}{2 \pi r} \oint_{\bdry \DiskRC{r}{z}}
            \frac{f(\zeta)}{\zeta - z} \, |\ud \zeta| .
  \end{align*}
\end{theorem}
\begin{proof}
  \uses{thm:CAUCHY_FORMULA}
  % \leanok
  \ldots
\end{proof}


\chapter{Power series}
\section{Uniform convergence}

\begin{definition}[Uniform convergence \Palka{Sec.~VII.1.1}]
  \label{def:uniform_limit}
  \uses{def:limit}
  %\lean{}
  % \leanok
  Let $(f_n)_{n \in \bN}$ be a sequence of functions
  $f_n \colon X \to \bC$, and let $f \colon X \to \bC$
  also be a such function. We say that the sequence
  $(f_n)_{n \in \bN}$ \term{converges uniformly} to $f$
  (on $X$) if for every $\eps > 0$ there exists an $N \in \bN$
  such that for all $n \ge N$ we have
  \begin{align*}
    \big| f_n(x) - f(x) \big| < \eps
    \qquad \text{ for all } x \in X .
  \end{align*}
\end{definition}

\begin{lemma}[Continuity is preserved in uniform limits \Palka{Thm~VII.1.1}]
  \label{lem:uniform_limit_continuous}
  \uses{def:uniform_limit, def:continuous}
  %\lean{}
  % \leanok
  Let $X$ be a metric space (e.g., $\bR$, $\bC$, or a subset of these).
  If a sequence $(f_n)_{n \in \bN}$ of continuous functions
  $f_n \colon X \to \bC$ converges uniformly to a function
  $f \colon X \to \bC$, then $f$ is continuous.
\end{lemma}
\begin{proof}
  %\uses{def:uniform_limit}
  % \leanok
  %\ldots
  See \textbf{MS-C1541 Metric Spaces}.
\end{proof}

\begin{lemma}[Integration commutes with uniform limits \Palka{Thm~VII.1.1}]
  \label{lem:uniform_limit_integral}
  \uses{def:uniform_limit, def:continuous, def:riemann_integral, lem:uniform_limit_continuous}
  %\lean{}
  % \leanok
  If a sequence $(f_n)_{n \in \bN}$ of continuous functions
  $f_n \colon [a,b] \to \bC$ on a closed interval $[a,b] \subset \bR$
  converges uniformly to a function
  $f \colon [a,b] \to \bC$, then we have
  \begin{align*}
    \lim_{n \to \infty} \int_a^b f_n(x) \, \ud x
      = \int_a^b f(x) \, \ud x .
  \end{align*}
\end{lemma}
\begin{proof}
  %\uses{}
  % \leanok
  \ldots
\end{proof}

\begin{corollary}[Contour integration commutes with uniform limits \Palka{Thm~VII.1.1}]
  \label{cor:uniform_limit_contour_integral}
  \uses{def:uniform_limit, def:continuous, def:contour_integral}
  %\lean{}
  % \leanok
  %Let $A \subset \bC$ be a subset of the complex plane.
  If a sequence $(f_n)_{n \in \bN}$ of continuous functions
  $f_n \colon A \to \bC$ on a subset $A \subset \bC$  of the complex plane
  converges uniformly to a function
  $f \colon A \to \bC$, then for any piecewise smooth path $\gamma$
  in $A$ we have
  \begin{align*}
    \lim_{n \to \infty} \int_\gamma f_n(z) \, \ud z
      = \int_\gamma f(z) \, \ud z .
  \end{align*}
\end{corollary}
\begin{proof}
  \uses{lem:uniform_limit_integral}
  % \leanok
  This follows straightforwardly from the definition of contour integration
  and Lemma~\ref{lem:uniform_limit_integral} above.
\end{proof}

\begin{lemma}[Sequence of analytic functions \Palka{Thm~VII.3.1}]
  \label{lem:uoc_convergent_analytic}
  \uses{def:uniform_limit, def:analytic_function}
  %\lean{}
  Suppose that functions $f_1, f_2, \ldots \colon U \to \bC$
  are analytic functions on an open set $U \subset \bC$
  such that on each compact subset $K \subset U$, the sequence
  $(f_n)_{n \in \bN}$ converges uniformly to a
  function $f$. Then $f$ is analytic on $U$.
  Moreover, for any $k \in \bN$, the sequence $(f^{(k)}_n)_{n \in \bN}$
  of $k$th derivatives converges uniformly on compacts to $f^{(k)}$.
  % \leanok
\end{lemma}
\begin{proof}
  \uses{cor:uniform_limit_contour_integral, thm:cauchy_formula_derivative}
  % \leanok
  \ldots
\end{proof}

\begin{lemma}[Cauchy criterion for uniform convergence \Palka{Thm~VII.1.2}]
  \label{lem:cauchy_uniform_limit_criterion}
  \uses{def:uniform_limit}
  %\lean{}
  % \leanok
  \ldots
\end{lemma}
\begin{proof}
  \uses{lem:complex_complete}
  % \leanok
  \ldots
\end{proof}



\section{Complex series}

\begin{definition}[Complex series \Palka{Sec.~VII.2.1}]
  \label{def:complex_series}
  \uses{def:limit, thm:complex_number_field}
  %\lean{}
  % \leanok
  \ldots
\end{definition}



\section{Series of functions}

\begin{definition}[Series of functions \Palka{Sec.~VII.2.2}]
  \label{def:function_series}
  \uses{def:limit, def:uniform_limit, def:complex_series}
  %\lean{}
  % \leanok
  \ldots
\end{definition}

\begin{lemma}[Weierstrass M-test \Palka{Thm~VII.2.2}]
  \label{lem:weierstrass_test}
  \uses{def:function_series}
  %\lean{}
  % \leanok
  Suppose that $M_1,M_2,\ldots \ge 0$ are nonnegative
  numbers such that the series $\sum_{n=1}^\infty M_n$ converges.
  Suppose also that for each $n \in \bN$,
  $f_n \colon X \to \bC$ is a function on $X$
  such that $|f_n(x)| \le M_n$ for all $x \in X$.
  Then the series $\sum_{n=1}^\infty f_n$
  converges absolutely and uniformly on~$X$.
\end{lemma}
\begin{proof}
  \uses{lem:cauchy_uniform_limit_criterion}
  % \leanok
  \ldots
\end{proof}

\begin{lemma}[Series of analytic functions \Palka{Thm~VII.3.2}]
  \label{lem:uoc_convergent_analytic_series}
  \uses{def:uniform_limit, def:function_series, def:analytic_function}
  %\lean{}
  Suppose that functions $f_1, f_2, \ldots \colon U \to \bC$
  are analytic functions on an open set $U \subset \bC$
  such that on each compact subset $K \subset U$, the series
  $\sum_{n=1}^\infty f_n$ converges uniformly to a
  function $f$. Then $f$ is analytic on $U$.
  Moreover, for any $k \in \bN$, the series $\sum_{n=1}^\infty f^{(k)}_n$
  of $k$th derivatives converges uniformly on compacts to $f^{(k)}$.
  % \leanok
\end{lemma}
\begin{proof}
  \uses{lem:uoc_convergent_analytic}
  % \leanok
  \ldots
\end{proof}



\section{Power series}

\begin{definition}[Power series \Palka{Sec.~VII.3.3}]
  \label{def:power_series}
  \uses{def:function_series}
  %\lean{}
  % \leanok
  Let $z_0 \in \bC$ be a point in the complex plane
  and let $a_0,a_1,a_2\ldots \in \bC$ be coefficients.
  A function series of the form
  \begin{align*}
  \sum_{n=0}^\infty a_n \, (z-z_0)^n
    = a_0 + a_1 \, (z - z_0) + a_2 \, (z - z_0)^2 + \cdots
  \end{align*}
  is called a \term{power series} centered at $z_0$.
\end{definition}

\begin{lemma}[Hadamard's formula for the radius of convergence \Palka{Thm~VII.3.3}]
  \label{lem:radius_of_convergence}
  \uses{def:power_series}
  %\lean{}
  % \leanok
  Let $z_0 \in \bC$ be a point in the complex plane
  and let $a_0,a_1,a_2\ldots \in \bC$ be coefficients.
  Define
  \begin{align*}
    R = \frac{1}{\limsup_{n \to \infty} \sqrt[n]{|a_n|}} .
  \end{align*}
  Then the power series
  \begin{align*}
    \sum_{n=0}^\infty a_n \, (z-z_0)^n
      = a_0 + a_1 \, (z - z_0) + a_2 \, (z - z_0)^2 + \cdots
  \end{align*}
  diverges for any $z \in \bC$ such that $|z-z_0| > R$
  and converges for any $z \in \bC$ such that $|z-z_0| < R$.
  Moreover, the convergence is uniform on compact subsets
  of $\BallRC{R}{z_0}$ (but not necessarily uniform on $\BallRC{R}{z_0}$).

  (The number $R$ above is called the \term{radius of convergence}
  of the power series.)
\end{lemma}
\begin{proof}
  \uses{lem:weierstrass_test}
  % \leanok
  \ldots
\end{proof}

\begin{lemma}[Analyticity of the sum of a power series \Palka{Thm~VII.3.3}]
  \label{lem:power_series_analytic}
  \uses{def:power_series, def:analytic_function, lem:radius_of_convergence}
  %\lean{}
  % \leanok
  Let $z_0 \in \bC$ be a point in the complex plane
  and let $a_0,a_1,a_2\ldots \in \bC$ be coefficients.
  Suppose that the power series
  \begin{align*}
    f(z) = \sum_{n=0}^\infty a_n \, (z-z_0)^n
  \end{align*}
  has radius of convergence $R > 0$.
  Then it defines an analytic function~$f$ on the disk $\DiskRC{R}{z_0}$.
  The derivative of $f$ is given by the power series
  \begin{align*}
    f'(z) = \sum_{n=1}^\infty n \, a_n \, (z-z_0)^{n-1} .
  \end{align*}
  Moreover, the coefficients $a_k$ are related to the $k$th derivatives of $f$
  at $z_0$ through the formula
  \begin{align*}
    a_k = \frac{f^{(k)}(z_0)}{k!} .
  \end{align*}
\end{lemma}
\begin{proof}
  \uses{lem:uoc_convergent_analytic_series}
  % \leanok
  \ldots
\end{proof}



\section{Taylor series and local representation of analytic functions}

\begin{theorem}[Taylor series of analytic functions \Palka{Thm~VII.3.4}]
  \label{thm:analytic_function_taylor_series}
  \uses{def:power_series, def:analytic_function}
  %\lean{}
  % \leanok
  Suppose that $f \colon U \to \bC$ is an analytic function
  on an open set $U \subset \bC$ which contains
  a disk $\DiskRC{r}{z_0} \subset U$.
  Then the function $f$ can be represented in $\DiskRC{r}{z_0}$
  as a power series
  \begin{align*}
    f(z) = \sum_{n=0}^\infty \frac{f^{(n)}(z_0)}{n!} \, (z-z_0)^n .
  \end{align*}
  Moreover, this is the unique power series centered at $z_0$
  that representats $f$ in a neighborhood of $z_0$.
\end{theorem}
\begin{proof}
  \uses{thm:cauchy_formula_derivative, lem:uniform_limit_integral, lem:power_series_analytic}
  % \leanok
  \ldots
\end{proof}

\begin{theorem}[Equivalent characterizations of analyticity]
  \label{thm:TFAE_ANALYTIC}
  \uses{def:analytic_function, def:power_series, def:primitive}
  % \lean{}
  % \leanok
  Let $f \colon U \to \bC$ be a continuous function on an open
  set $U \subset \bC$.
  Then the following are equivalent:
  \begin{itemize}
    \item $f$ is analytic on $U$;
    \item for any $z \in U$ there exists a neighborhood of $z$
      in which $f$ has a primitive;
    \item for any $z \in U$ there exists a neighborhood of $z$
      in which $f$ can be represented as a convergent power series.
  \end{itemize}
\end{theorem}
\begin{proof}
  \uses{thm:analytic_function_taylor_series, lem:power_series_analytic, thm:morera}
  % \leanok
  \ldots
\end{proof}

\begin{lemma}[No vanishing of all derivatives at a point \Palka{Thm~VIII.1.1}]
  \label{lem:no_vanishing_of_all_order_derivatives}
  \uses{def:analytic_function, lem:higher_derivatives_analytic}
  %\lean{}
  % \leanok
  Suppose that $f \colon \Domain \to \bC$ is an analytic function
  on a connected open set $\Domain \subset \bC$.
  If there exists a point $z_0 \in \Domain$
  such that $f^{(n)}(z_0) = 0$ for all $n \in \bN$, then $f$ is a constant
  function.
\end{lemma}
\begin{proof}
  \uses{thm:analytic_function_taylor_series}
  % \leanok
  \ldots
\end{proof}

\begin{theorem}[Factor theorem for analytic functions \Palka{Thm~VIII.1.2}]
  \label{thm:factor_theorem_for_analytic_functions}
  \uses{def:analytic_function}
  %\lean{}
  % \leanok
  Suppose that $f \colon \Domain \to \bC$ is a non-constant analytic function
  on a connected open set $\Domain \subset \bC$, and $z_0 \in \Domain$
  is a point where $f(z_0) = 0$.
  Then $f$ can be uniquely represented as
  \begin{align*}
    f(z) = (z - z_0)^m \, g(z) \qquad \text{ for } z \in \Domain ,
  \end{align*}
  where $m \in \bN$ and $g \colon \Domain \to \bC$ is an analytic
  function such that $g(z_0) \ne 0$.
\end{theorem}
\begin{proof}
  \uses{thm:analytic_function_taylor_series, lem:no_vanishing_of_all_order_derivatives}
  % \leanok
  \ldots
\end{proof}

\begin{corollary}[Local representation of analytic functions \Palka{Cor~VIII.1.3}]
  \label{cor:analytic_local_representation}
  \uses{def:analytic_function}
  %\lean{}
  % \leanok
  Suppose that $f \colon \Domain \to \bC$ is a non-constant analytic function
  on a connected open set $\Domain \subset \bC$.
  Then for any $z_0 \in \Domain$, we can write $f$ uniquely in the form
  \begin{align*}
    f(z) = f(z_0) + (z - z_0)^m \, g(z) \qquad \text{ for } z \in \Domain ,
  \end{align*}
  where $m \in \bN$ and $g \colon \Domain \to \bC$ is an analytic
  function such that $g(z_0) \ne 0$.
\end{corollary}
\begin{proof}
  \uses{thm:factor_theorem_for_analytic_functions}
  % \leanok
  Apply Theorem~\ref{thm:factor_theorem_for_analytic_functions} to the
  function $z \mapsto f(z) - f(z_0)$.
\end{proof}

\begin{theorem}[L'Hospital's rule for analytic functions \Palka{Thm~VIII.1.4}]
  \label{thm:l_hospital_rule}
  \uses{def:analytic_function}
  %\lean{}
  % \leanok
  Let $f$ and $g$ be functions that are analytic in a neighborhood of $z_0$
  such that $f(z_0) = 0$ and $g(z_0) = 0$. Then we have
  \begin{align*}
    \lim_{z \to z_0} \frac{f(z)}{g(z)} = \lim_{z \to z_0} \frac{f'(z)}{g'(z)} ,
  \end{align*}
  understood in the sense that either both limits exist and are equal
  to each other, or else neither limit exists.
\end{theorem}
\begin{proof}
  \uses{cor:analytic_local_representation}
  % \leanok
  \ldots
\end{proof}

\begin{theorem}[Discrete mapping theorem \Palka{Thm~VIII.1.5}]
  \label{thm:discrete_mapping_theorem}
  \uses{def:analytic_function}
  %\lean{}
  % \leanok
  Suppose that $f \colon \Domain \to \bC$ is a non-constant
  analytic function on a connected open set $\Domain \subset \bC$.
  Then the set of zeros of $f$ is discrete, i.e., for
  every $z_0 \in \Domain$ such that $f(z_0)=0$,
  there exists a $r>0$ such that $f(z) \ne 0$
  for all $z \in \BallRC{r}{z_0} \setminus \set{z_0}$.
\end{theorem}
\begin{proof}
  \uses{thm:factor_theorem_for_analytic_functions}
  % \leanok
  \ldots
\end{proof}

\begin{corollary}[Principle of analytic continuation \Palka{Cor~VIII.1.6}]
  \label{cor:ANALYTIC_CONTINUATION}
  \uses{def:analytic_function}
  %\lean{}
  % \leanok
  Let $f, g \colon \Domain \to \bC$ be two analytic functions on a connected
  open set $\Domain \subset \bC$.
  If $f(z) = g(z)$ for all $z$ in
  some subset of $\Domain$ which has an accumulation point in $\Domain$,
  then we have $f(z) = g(z)$ for all $z \in \Domain$.
\end{corollary}
\begin{proof}
  \uses{thm:discrete_mapping_theorem}
  % \leanok
  \ldots
\end{proof}



\section{Laurent series}

\begin{definition}[Doubly infinite series \Palka{Sec.~VII.2.1}]
  \label{def:doubly_infinite_series}
  \uses{def:complex_series}
  %\lean{}
  % \leanok
  A doubly infinite series of complex numbers is a series of the form
  \begin{align*}
    \sum_{n=-\infty}^\infty z_n
      = \cdots + z_{-2} + z_{-1} + z_0 + z_1 + z_2 + \cdots ,
  \end{align*}
  where $\ldots, z_{-2}, z_{-1}, z_0, z_1, z_2, \ldots \in \bC$.
  We say that such a series \term{converges} to $s \in \bC$ if
  for all $\eps > 0$ there exists an $N \in \bN$
  such that for all $m_+ \ge N$ and $m_- \le -N$ we have
  \begin{align*}
    \Big| \sum_{n=m_-}^{m_+} z_n - s \Big| < \eps .
  \end{align*}
\end{definition}

\begin{lemma}[Convergence of doubly infinite series \Palka{Lem~VII.2.1}]
  \label{lem:convergence_doubly_infinite}
  \uses{def:doubly_infinite_series}
  %\lean{}
  % \leanok
  A doubly infinite series
  \begin{align*}
    \sum_{n=-\infty}^\infty z_n
      = \cdots + z_{-2} + z_{-1} + z_0 + z_1 + z_2 + \cdots ,
  \end{align*}
  of complex numbers converges if and only if both the series
  $\sum_{n=0}^\infty z_n$ and $\sum_{n=1}^\infty z_{-n}$ converge.
\end{lemma}
\begin{proof}
  \uses{def:limit}
  % \leanok
  \ldots
\end{proof}

\begin{definition}[Laurent series \Palka{Sec.~VII.3.4}]
  \label{def:laurent_series}
  \uses{def:doubly_infinite_series}
  %\lean{}
  % \leanok
  A \term{Laurent series} centered at $z_0 \in \bC$ is a doubly infinite series
  of functions of the form
  \begin{align*}
    z \mapsto & \; \sum_{n=-\infty}^\infty a_n (z - z_0)^n \\
         = & \; \cdots + \frac{a_{-2}}{(z-z_0)^2} + \frac{a_{-1}}{z-z_0}
                + a_0 + a_1 (z-z_0) + a_2 (z-z_0)^2 + \cdots .
  \end{align*}
\end{definition}

\begin{lemma}[Annulus of convergence of Laurent power series \Palka{Thm~VII.3.5}]
  \label{lem:laurent_series_analytic}
  \uses{def:laurent_series}
  %\lean{}
  % \leanok
  Consider a Laurent series
  \begin{align*}
    f(z) = & \; \sum_{n=-\infty}^\infty a_n (z - z_0)^n .
  \end{align*}
  Denote
  \begin{align*}
    \rho_- = \limsup_{n \to \infty} \sqrt[n]{|a_{-n}|}, \qquad
    \rho_+ = \Big( \limsup_{n \to \infty} \sqrt[n]{|a_{n}|} \Big)^{-1} .
  \end{align*}
  Then the series $\sum_{n=-\infty}^\infty a_n (z - z_0)^n$
  converges for all $z$ is the annulus
  \begin{align*}
    \AnnRRC{\rho_-}{\rho_+}{z_0}
      := \Big\{ z \in \bC \; \Big| \; \rho_- < |z-z_0| < \rho_+ \Big\} .
  \end{align*}
  Moreover, the convergence is uniform on compact subsets
  of $\AnnRRC{\rho_-}{\rho_+}{z_0}$, and the series defines an analytic
  function $f(z)$ on the annulus $\AnnRRC{\rho_-}{\rho_+}{z_0}$.
\end{lemma}
\begin{proof}
  \uses{lem:power_series_analytic}
  % \leanok
  \ldots
\end{proof}

\begin{theorem}[Laurent series for analytic functions \Palka{Thm~VII.3.6}]
  \label{thm:ANALYTIC_FUNCTION_LAURENT_SERIES}
  \uses{def:laurent_series, def:analytic_function}
  %\lean{}
  % \leanok
  Suppose that $f \colon U \to \bC$ is an analytic function
  on an open set $U \subset \bC$ which contains
  an annulus
  \begin{align*}
    \AnnRRC{r_1}{r_2}{z_0} = \Big\{z \in \bC \; \Big| \; r_1 < |z - z_0| < r_2 \Big\}
  \end{align*}
  for some $z_0 \in \bC$ and $0 \le r_1 < r_2$.
  %$\DiskRC{r}{z_0} \subset U$.
  Then the function $f$ can be represented in $\AnnRRC{r_1}{r_2}{z_0}$
  as a series
  \begin{align*}
    f(z) = \sum_{n=-\infty}^\infty a_n \, (z-z_0)^n ,
  \end{align*}
  where the coefficients $a_n$, for $n \in \bZ$, are given by
  \begin{align*}
    a_n = \frac{1}{2\pi \ii}
      \oint_{\bdry \BallRC{r}{z_0}} \frac{f(z)}{(z - z_0)^{n+1}} \, \ud z
    \qquad \text{ for any } r \in (r_1, r_2) .
  \end{align*}
\end{theorem}
\begin{proof}
  \uses{thm:cauchy_formula_derivative, lem:uniform_limit_integral,
        lem:laurent_series_analytic, thm:analytic_function_taylor_series}
  % \leanok
  \ldots
\end{proof}


\chapter{Isolated singularities and residues}
\section{Isolated singularities of analytic functions}

\begin{definition}[Isolated singularity \Palka{Sec.~VIII.2.1}]
  \label{def:isolated_singularity}
  \uses{def:analytic_function}
  %\lean{}
  % \leanok
  Let $f : U \to \bC$ be an analytic function on an open set $U \subset \bC$.
  We say that $f$ has an \term{isolated singularity} at $z_0 \in \bC$ if
  $\BallRC{r}{z_0} \setminus \set{z_0} \subset U$ for some $r>0$ but $z_0 \notin U$.
\end{definition}

\begin{definition}[Classification of isolated singularities \Palka{Sec.~VIII.2.1}]
  \label{def:singularity_classification}
  \uses{def:isolated_singularity, thm:analytic_function_laurent_series}
  %\lean{}
  % \leanok
  Let $z_0 \in \bC$ be an isolated singularity of
  an analytic function $f : U \to \bC$.
  Let $r > 0$ be such that $\BallRC{r}{z_0} \setminus \set{z_0} \subset U$,
  so that by Theorem~\ref{thm:analytic_function_laurent_series}
  $f$ can be represented in $\BallRC{r}{z_0} \setminus \set{z_0}$
  uniquely as a Laurent series
  \begin{align*}
    f(z) = \sum_{n=-\infty}^\infty a_n (z - z_0)^n .
  \end{align*}
  Depending on the coefficients $a_n$ of negative indices $n < 0$,
  we distinguish three types of singularities:
  \begin{itemize}
    \item $f$ has a \term{removable singularity} at $z_0$ if $a_n = 0$ for all $n < 0$;
    \item $f$ has a \term{pole of order $m \in \bN$} at $z_0$
      if $a_{-m} \ne 0$ and $a_n = 0$ for all $n < -m$;
    \item $f$ has an \term{essential singularity} at $z_0$
      $a_n \ne 0$ for infinitely many $n < 0$.
  \end{itemize}
\end{definition}

\begin{definition}[Residue at an isolated singularity \Palka{Sec.~VIII.2.1}]
  \label{def:residue}
  \uses{def:isolated_singularity, thm:analytic_function_laurent_series}
  %\lean{}
  % \leanok
  Let $z_0 \in \bC$ be an isolated singularity of
  an analytic function $f : U \to \bC$.
  Let $r > 0$ be such that $\BallRC{r}{z_0} \setminus \set{z_0} \subset U$,
  so that by Theorem~\ref{thm:analytic_function_laurent_series}
  $f$ can be represented in $\BallRC{r}{z_0} \setminus \set{z_0}$
  uniquely as a Laurent series
  \begin{align*}
    f(z) = \sum_{n=-\infty}^\infty a_n (z - z_0)^n .
  \end{align*}
  The coefficient $a_{-1}$ is called the \term{residue} of $f$ at $z_0$,
  and is denoted $\ResAtOf{z_0}{f} = a_{-1} \in \bC$.
\end{definition}

\begin{theorem}[Riemann extension theorem \Palka{Thm~VIII.2.1}]
  \label{thm:removable_singularity}
  \uses{def:singularity_classification}
  %\lean{}
  % \leanok
  Let $z_0 \in \bC$ be an isolated singularity of
  an analytic function $f : U \to \bC$.
  Then the singularity is removable if and only if
  $f$ is bounded in some punctured disk $\BallRC{r}{z_0} \setminus \{z_0\}$
  with $r>0$.
\end{theorem}
\begin{proof}
  %\uses{}
  % \leanok
  \ldots
\end{proof}

\begin{theorem}[Characterization of poles \Palka{Thm~VIII.2.4}]
  \label{thm:pole}
  \uses{def:singularity_classification}
  %\lean{}
  % \leanok
  Let $z_0 \in \bC$ be an isolated singularity of
  an analytic function $f : U \to \bC$.
  Then the singularity is a pole if and only if
  \begin{align*}
    \lim_{z \to z_0} | f(z) | = \infty .
  \end{align*}
\end{theorem}
\begin{proof}
  %\uses{}
  % \leanok
  \ldots
\end{proof}

\begin{theorem}[Characterization of poles \Palka{Thm~VIII.2.4}]
  \label{thm:essential_singularity}
  \uses{def:singularity_classification}
  %\lean{}
  % \leanok
  Let $z_0 \in \bC$ be an isolated singularity of
  an analytic function $f : U \to \bC$.
  Then the singularity is an essential singularity if and only if
  for any $r>0$ such that $\PDisk = \BallRC{r}{z_0} \setminus \{z_0\} \subset U$
  we have that the image $f[\PDisk] \subset \bC$ is dense.
\end{theorem}
\begin{proof}
  %\uses{}
  % \leanok
  \ldots
\end{proof}



\section{The residue theorem}

\begin{theorem}[Residue theorem \Palka{Thm~VIII.3.1}]
  \label{thm:residue_theorem}
  \uses{def:residue}
  %\lean{}
  % \leanok
  Let $U \subset \bC$ be an open set and $\gamma$ a contractible closed contour in $U$.
  Let $f : U \setminus S \to \bC$ be an analytic function with isolated
  singularities at a countable set $S \subset U$ of points.
  Then
  \begin{align*}
    \oint_\gamma f(z) \, dz
      = 2 \pi \ii \sum_{w \in S} \windN{w}{\gamma} \, \ResAtOf{w}{f} .
  \end{align*}
\end{theorem}
\begin{proof}
  \uses{thm:cauchy_integral_formula}
  % \leanok
  \ldots
\end{proof}

\begin{corollary}[Residue theorem for Jordan contours \Palka{Cor~VIII.3.2}]
  \label{thm:jordan_residue_theorem}
  \uses{def:residue}
  %\lean{}
  % \leanok
  Let $U \subset \bC$ be an open set and $S \subset U$ a discrete subset of it.
  Let $\Domain$ be a Jordan domain such that $\closOf{\Domain} \subset U$
  and $\bdry \Domain \cap S = \emptyset$.
  Let $\gamma$ be a closed contour traversing the boundary $\bdry \Domain$
  of the Jordan domain in the positive orientation.
  Let $f : U \setminus S \to \bC$ be an analytic function with isolated
  singularities at the points of $S$.
  Then
  \begin{align*}
    \oint_\gamma f(z) \, dz = 2 \pi \ii \sum_{w \in S \cap \Domain} \ResAtOf{w}{f} .
  \end{align*}
\end{corollary}
\begin{proof}
  \uses{thm:residue_theorem}
  % \leanok
  \ldots
\end{proof}


\appendix

\chapter{Topological preliminaries}
\section{Metrics and related concepts}

\begin{definition}[Metric]
  \label{def:metric_space}
  %\uses{}
  %\lean{}
  % \leanok
  A \term{metric} on a set $X$ is a function
  $\genmet \colon X \times X \to [0,\infty)$ such that
  for all $p_1, p_2, p_3 \in X$ we have
  \begin{align*}
    \genmet(p_1, p_3) \le \; & \genmet(p_1,p_2) + \genmet(p_2,p_3)
            & \text{(triangle inequality)} \\
    \genmet(p_1, p_2) = \; & \genmet(p_2,p_1)
            & \text{(symmetricity)} \\
    \genmet(p_1, p_2) = \; & 0 \text{ if and only if } p_1 = p_2 .
            & \text{(separation of points)}
  \end{align*}
  The set $X$ equipped with the metric $\genmet$ on
  it is called a \term{metric space}.
\end{definition}

\begin{lemma}[Metric in the complex plane]
  \label{lem:metric_on_C}
  \uses{def:metric_space, def:absolute_value}
  % \lean{}
  % \leanok
  The formula
  \begin{align*}
    \genmet(z,w) = |z - w|
    \qquad \text{ for } z,w \in \bC
  \end{align*}
  defines a metric on the complex plane $\bC$.

  (Thus $\bC$ becomes a metric space. Also any subset of $\bC$,
  in particular $\bR \subset \bC$, becomes a metric space when equipped
  with the metric given by the above formula restricted to the subset.)
\end{lemma}
\begin{proof}
  \uses{lem:absolute_value_properties}
  % \leanok
  \ldots
\end{proof}

\begin{definition}[Ball (disk)]
  \label{def:ball}
  \uses{def:metric_space}
  %\lean{}
  % \leanok
  Let $X$ be a metric space with metric $\genmet \colon X \times X \to [0,\infty)$.
  Let $p_0 \in X$ be a point and let $r>0$.

  The set
  \begin{align*}
    \BallRC{r}{p_0} = \Big\{ p \in X \; \Big| \; \genmet(p,p_0) < r \Big\}
  \end{align*}
  is called an \term{open ball} in $X$, centered at $p_0$, and with radius $r$.

  The set
  \begin{align*}
    \CBallRC{r}{p_0} = \Big\{ p \in X \; \Big| \; \genmet(p,p_0) \le r \Big\}
  \end{align*}
  is called a \term{closed ball} in $X$, centered at $p_0$, and with radius $r$.

  %The set
  %\begin{align*}
  %  \SphereRC{r}{x_0} = \set{x \in X \; \Big| \; \genmet(x,x_0) = r}
  %\end{align*}
  %is called a \term{sphere} in $X$, centered at $x_0$, and with radius $r$.

  (In the case of the complex plane $\bC$, the term \term{disk}
  is often used instead of the general metric space theory term \term{ball}.)
  %and the term \term{circle}
  %may be used instead of the general term sphere.)
\end{definition}

\begin{definition}[Interior point]
  \label{def:interior}
  \uses{def:ball}
  % \lean{}
  % \leanok
  Let $X$ be a metric space, and $A \subset X$ a subset.
  A point $p \in A$ is said to be an \term{interior point} of $A$
  if for some $r > 0$ we have $\BallRC{r}{p} \subset A$.
  %The set of all interior points of $A$ is denoted \ldots and called the \term{interior} of $A$,
\end{definition}

\begin{definition}[Exterior point]
  \label{def:exterior}
  \uses{def:ball, def:interior}
  % \lean{}
  % \leanok
  Let $X$ be a metric space, and $A \subset X$ a subset.
  A point $p \in X \setminus A$ is said to be an \term{exterior point} of $A$
  if for some $r > 0$ we have $\BallRC{r}{p} \subset X \setminus A$.

  (It is easy to see that the exterior points of $A$ are exactly the interior
  points)
\end{definition}

\begin{definition}[Boundary]
  \label{def:boundary}
  \uses{def:ball, def:interior, def:exterior}
  % \lean{}
  % \leanok
  Let $X$ be a metric space, and $A \subset X$ a subset.
  A point $p \in X$ is said to be a \term{boundary point} of $A$
  if for all $r > 0$ we have that $\BallRC{r}{p}$ contains points
  of $A$ and $X \setminus A$ (i.e. $\BallRC{r}{p} \cap A \ne \emptyset$
  and $\BallRC{r}{p} \setminus A \ne \emptyset$).

  The set of all boundary points of $A$ is denoted $\bdry A$ and called the
  \term{boundary} of $A$.

  (It is easy to see that the boundary $\bdry A \subset X$ is exactly
  the set of points of $X$ which are neither interior nor exterior points
  of $A$.)
\end{definition}

\begin{definition}[Open set]
  \label{def:open_set}
  \uses{def:metric_space, def:ball, def:interior}
  % \lean{}
  % \leanok
  Let $X$ be a metric space.
  % with metric $\genmet \colon X \times X \to [0,\infty)$.
  A subset $U \subset X$ is said to be an \term{open set} if
  %for every $x \in U$ there exists an $r > 0$ such that
  %$\BallRC{r}{x} \subset U$.
  each point $p \in U$ is an interior point of $U$.
\end{definition}

\begin{definition}[Closed set]
  \label{def:closed_set}
  \uses{def:open_set, def:exterior}
  % \lean{}
  % \leanok
  Let $X$ be a metric space.
  % with metric $\genmet \colon X \times X \to [0,\infty)$.
  A subset $F \subset X$ is said to be a \term{closed set} if
  the complement $X \setminus F \subset X$ is an open set.

  (Equivalently, each point $p \in X \setminus F$ in the complement
  of $F$ is an exterior point of $F$.)
\end{definition}

\begin{definition}[Boundedness]
  \label{def:bounded}
  \uses{def:metric_space}
  % \lean{}
  % \leanok
  Let $X$ be a metric space.
  with metric $\genmet \colon X \times X \to [0,\infty)$.

  A subset $A \subset X$ is \term{bounded} if there exists
  a number $M>0$ such that $\genmet(p,q) \le M$ for all $p,q \in A$.
  (If $X$ is nonempty, an equivalent definition would be that
  $A$ is bounded if it is a subset of some ball in $X$.)

  A function $f \colon Z \to X$ with values in a metric space~$X$
  is \term{bounded} if the set $f[Z] \subset X$ of its values is a
  bounded subset of~$X$.

  (In the case $X = \bC$ we have the following further characterizations:
  A subset $A \subset \bC$ is bounded if and only if there exists an $R>0$
  such that $|z| \le R$ for all $z \in A$. A function $f \colon Z \to \bC$
  is bounded if and only if there exists an $R>0$ such that $|f(z)| \le R$
  for all $z \in Z$.)
\end{definition}



\section{Limits}

\begin{definition}[Limit]
  \label{def:limit}
  \uses{def:metric_space}
  %\lean{}
  % \leanok
  Let $X$ be a metric space and
  %with metric $\genmet \colon X \times X \to [0,\infty)$.
  let $(x_n)_{n \in \bN}$ be a sequence of points in $X$.
  %We say that $x \in X$ is the \term{limit} of the sequence $(x_n)_{n \in \bN}$
  %%(or that the sequence \term{converges} to the limit $x$)
  We say that the sequence $(x_n)_{n \in \bN}$ \term{converges}
  to a \term{limit} $x \in X$
  if for any $\eps > 0$ there exists an $N \in \bN$ such that
  for all $n \ge N$ we have $x_n \in \BallRC{\eps}{x}$
  (i.e., $\genmet(x_n,x) < \eps$).
  We then denote
  \begin{align*}
    \lim_{n \to \infty} x_n = x .
  \end{align*}
  (It is straightforward to check that the limit is
  unique if it exists.)

  Let then $X$ and $Y$ be metric spaces, with respective metrics
  $\genmet_X$ and $\genmet_Y$, and
  let $f \colon X \to Y$ be a function.
  We say that the function $f$ has a \term{limit} $y \in Y$
  \term{at a point} $p_0 \in X$
  if for any $\eps > 0$ there exists a $\delta > 0$ such that
  for all $p \in \BallRC{\delta}{p_0} \setminus \set{p_0}$
  %(i.e., when $0 < \genmet_X (p,p_0) < \delta$)
  we have $f(p) \in \BallRC{\eps}{y}$.
  %(i.e., $\genmet_Y \big( f(p),y \big) < \eps$).
  We then denote
  \begin{align*}
    \lim_{p \to p_0} f(p) = y .
  \end{align*}
  (It is straightforward to check that the limit is
  unique if it exists.)

  (Equivalently, written in terms of distances,
  $\lim_{p \to p_0} f(p) = y$ means that
  for any $\eps > 0$ there exists a $\delta > 0$ such that
  we have $\genmet_Y \big( f(p), y \big) < \eps$ whenever
  $0 < \genmet_X (p,p_0) < \delta$.)
\end{definition}

\begin{lemma}[Limits in the complex plane]
  \label{lem:complex_limit}
  \uses{def:limit, lem:metric_on_C}
  % \lean{}
  % \leanok
  For a sequence $(z_n)_{n \in \bN}$ of complex numbers
  %$z_n = x_n + \ii y_n$ (with $x_n,y_n \in \bR$) converges
  %to a limit $z = x + \ii y$ if and only if the real and imaginary
  %part sequences converge to the real and imaginary parts of the limit,
  %i.e., $\lim_{n \to \infty} x_n = x$ and $\lim_{n \to \infty} y_n = y$.
  we have
  \begin{align*}
    \lim_{n \to \infty} z_n = z
  \end{align*}
  if and only if
  \begin{align*}
    \lim_{n \to \infty} \re(z_n) = \re(z)
    \quad \text{and} \quad
    \lim_{n \to \infty} \im(z_n) = \im(z) .
  \end{align*}

  Let $X$ be a metric space,
  let $f \colon X \to \bC$ a complex-valued
  function on $X$, and
  %Write $u, v \colon X \to \bR$
  %for the real and imaginary parts of the function, so that
  %$f(p) = u(p) + \ii \, v(p)$ for $p \in X$.
  let $p_0 \in X$ be a point. Then we have
  \begin{align*}
    \lim_{p \to p_0} f(p) = z
  \end{align*}
  if and only if
  \begin{align*}
    \lim_{p \to p_0} \re \big( f(p) \big) = \re(z)
    \quad \text{and} \quad
    \lim_{p \to p_0} \im \big( f(p) \big) = \im(z) .
  \end{align*}
\end{lemma}
\begin{proof}
  %\uses{lem:absolute_value_properties}
  % \leanok
  \ldots
\end{proof}

\begin{lemma}[Operations with complex limits]
  \label{lem:complex_limit_properties}
  \uses{lem:complex_limit, thm:complex_number_field}
  % \lean{}
  % \leanok
  Let $(z_n)_{n \in \bN}$ and $(w_n)_{n \in \bN}$
  be complex number sequences converging to limits
  \begin{align*}
    \lim_{n \to \infty} z_n = z
    \quad \text{and} \quad
    \lim_{n \to \infty} w_n = w .
  \end{align*}
  Then we have
  \begin{align*}
    \lim_{n \to \infty} (z_n + w_n) \, = \, z + w ,
    \quad
    \lim_{n \to \infty} (z_n w_n) \, = \, z w ,
    \quad
    \lim_{n \to \infty} \frac{z_n}{w_n} \, = \, \frac{z}{w}
    \; \text{ if $w \ne 0$}.
  \end{align*}
    %If moreover $w \ne 0$ then we also have
    %\begin{align*}
    %  \lim_{n \to \infty} \frac{z_n}{w_n} \; = \; \frac{z}{w} .
    %\end{align*}
  %In particular, taking one of the sequences to be a constant sequence $c \in \bC$,
  %we get $\lim_{n \to \infty} (z_n + c) = z + c$ and
  %$\lim_{n \to \infty} (c \, z_n) = c z$.
  %%In particular, taking $(w_n)_{n \in \bN}$ to be a constant sequence
  %%$w_n = c \in \bC$, we have
  %%\begin{align*}
  %%  \lim_{n \to \infty} (z_n + c) = z + c
  %%  \quad \text{and} \quad
  %%  \lim_{n \to \infty} (c \, z_n) = c z .
  %%\end{align*}

  Let $X$ be a metric space, let $p_0 \in X$
  be a point, and let $f,g \colon X \to \bC$
  be two complex-valued functions on $X$ such that
  \begin{align*}
    \lim_{p \to p_0} f(p) = z
    \quad \text{and} \quad
    \lim_{p \to p_0} g(p) = w .
  \end{align*}
  Then we have
  \begin{align*}
    \lim_{p \to p_0} \big( f(p) + g(p) \big) \, = \, z + w ,
    \quad
    \lim_{p \to p_0} \big( f(p) \, g(p) \big) \, = \, z w ,
    \quad
    \lim_{p \to p_0} \frac{f(p)}{g(p)} \, = \, \frac{z}{w}
    \; \text{ if $w \ne 0$}.
  \end{align*}
    %If moreover $w \ne 0$ then we also have
    %\begin{align*}
    %  \lim_{p \to p_0} \frac{f(p)}{g(p)} \; = \; \frac{z}{w}
    %\end{align*}
  %In particular taking one of the functions to be a constant function
  %$c \in \bC$, we get $\lim_{p \to p_0} \big( f(p) + c \big) = z + c$
  %and $\lim_{p \to p_0} \big( c \, f(p) \big) = c z$.
  %%In particular, taking $g$ to be a constant function
  %%$g(p) = c \in \bC$, we have
  %%\begin{align*}
  %%  \lim_{p \to p_0} \big( f(p) + c \big) = z + c
  %%  \quad \text{and} \quad
  %%  \lim_{p \to p_0} \big( c \, f(p) \big) = c z .
  %%\end{align*}
\end{lemma}
\begin{proof}
  The arguments are similar to the proofs given in
  \MetSpCourse{} for the real-valued cases.
\end{proof}

\begin{definition}[Cauchy sequence]
  \label{def:cauchy_sequence}
  \uses{def:metric_space}
  %\lean{}
  % \leanok
  %Let $X$ be a metric space with metric $\genmet \colon X \times X \to [0,\infty)$.
  \ldots
\end{definition}

\begin{lemma}[Every real Cauchy sequence converges]
  \label{lem:real_complete}
  \uses{def:cauchy_sequence, def:limit}
  %\lean{}
  % \leanok
  If a real number sequence $(x_n)_{n \in \bN}$ is Cauchy,
  then it converges to a limit $\lim_{n \to \infty} x_n \in \bR$.

  (This property is known as \term{completeness} of the metric space $\bR$.)
\end{lemma}
\begin{proof}
  %\uses{def:cauchy_sequence}
  % \leanok
  See \MetSpCourse{}.
\end{proof}

\begin{lemma}[Every complex Cauchy sequence converges]
  \label{lem:complex_complete}
  \uses{def:cauchy_sequence, def:limit, lem:metric_on_C}
  %\lean{}
  % \leanok
  If a complex number sequence $(z_n)_{n \in \bN}$ is Cauchy,
  then it converges to a limit $\lim_{n \to \infty} z_n \in \bC$.

  (This property is known as \term{completeness} of the metric space $\bC$.)
\end{lemma}
\begin{proof}
  \uses{lem:real_complete, lem:complex_limit}
  % \leanok
  See \MetSpCourse{}.

  (Idea: This follows from Lemma~\ref{lem:real_complete} by
  considering real and imaginary parts separately and picking
  a subsequence of a subsequence.)
\end{proof}



\section{Continuity}

%\begin{definition}[Limit of a function at a point]
%  \label{def:function_limit}
%  \uses{def:metric_space}
%  %\lean{}
%  % \leanok
%  Let $X$ and $Y$ be metric spaces.
%  Let $f \colon X \to Y$ be
%  a function, and let $x_0 \in X$ be a point.
%  We say that the function $f$ has a \term{limit} $y \in Y$
%  \term{at the point} $x_0 \in X$
%  if for all $\eps > 0$ there exists a $\delta > 0$ such that
%  for all $x \in \BallRC{\delta}{x_0} \setminus \set{x_0}$
%  %(i.e., when $0 < \genmet_X (x,x_0) < \delta$)
%  we have $f(x) \in \BallRC{\eps}{y}$.
%  %(i.e., $\genmet_Y \big( f(x),y \big) < \eps$).
%  We then denote
%  \begin{align*}
%    \lim_{x \to x_0} f(x) = y .
%  \end{align*}
%  (It is straightforward to check that the limit is
%  unique if it exists.)
%
%  (Equivalently, written in terms of distances,
%  $\lim_{x \to x_0} f(x) = y$ means that
%  for every $\eps > 0$ there exists a $\delta > 0$ such that
%  we have $\genmet_Y \big( f(x),y \big) < \eps$ whenever
%  $0 < \genmet_X (x,x_0) < \delta$.)
%\end{definition}
%
%\begin{lemma}[Limits of complex valued functions]
%  \label{lem:complex_function_limit}
%  \uses{def:function_limit, lem:metric_on_C}
%  % \lean{}
%  % \leanok
%  Let $X$ be a metric space.
%  Let $f \colon X \to \bC$ a complex-valued
%  function on $X$ and
%  %Write $u, v \colon X \to \bR$
%  %for the real and imaginary parts of the function, so that
%  %$f(p) = u(p) + \ii \, v(p)$ for $p \in X$.
%  let $p_0 \in X$ be a point. Then we have
%  \begin{align*}
%    \lim_{p \to p_0} f(p) = z
%  \end{align*}
%  if and only if
%  \begin{align*}
%    \lim_{p \to p_0} \re \big( f(p) \big) = \re(z)
%    \quad \text{and} \quad
%    \lim_{p \to p_0} \im \big( f(p) \big) = \im(z) .
%  \end{align*}
%%\end{lemma}
%
%\begin{lemma}[Operations with limits of complex-valued functions]
%  \label{lem:complex_function_limit_properties}
%  \uses{lem:complex_function_limit}
%  % \lean{}
%  % \leanok
%  Let $X$ be a metric space, and let $f,g \colon X \to \bC$
%  be two complex-valued functions on $X$. Let $p_0 \in X$
%  be a point, and assume that
%  \begin{align*}
%    \lim_{p \to p_0} f(p) = z
%    \quad \text{and} \quad
%    \lim_{p \to p_0} g(p) = w .
%  \end{align*}
%  Then we have
%  \begin{align*}
%    \lim_{p \to p_0} \big( f(p) + g(p) \big) = z + w
%    \quad \text{and} \quad
%    \lim_{p \to p_0} \big( f(p) \, g(p) \big) = z w .
%  \end{align*}
%
%  In particular, taking $g$ to be a constant function
%  $g(p) = c \in \bC$, we have
%  \begin{align*}
%    \lim_{p \to p_0} \big( f(p) + c \big) = z + c
%    \quad \text{and} \quad
%    \lim_{p \to p_0} \big( c \, f(p) \big) = c z .
%  \end{align*}
%\end{lemma}

\begin{definition}[Continuity]
  \label{def:continuous}
  \uses{def:metric_space, def:limit}
  %\lean{}
  % \leanok
  Let $X$ and $Y$ be metric spaces.
  A function $f \colon X \to Y$ is said to be \term{continuous at a point}
  $p_0 \in X$ if $\lim_{p \to p_0} f(p) = f(p_0)$.

  (Equivalently, for every $\eps > 0$ there
  exists a $\delta > 0$ such that for any $p \in \BallRC{\delta}{p_0}$
  we have $f(p) \in \BallRC{\eps}{f(p_0)}$.)

  A function $f \colon X \to Y$ is said to be \term{continuous} if
  it is continuous at every point $p_0 \in X$.
\end{definition}

\begin{lemma}[Continuity of complex-valued functions]
  \label{lem:continuous_complex}
  \uses{def:continuous}
  % \lean{}
  % \leanok
  Let $X$ be a metric space, and let $f \colon X \to \bC$ be
  a complex-valued function on $X$. Then $f$ is continuous
  at $p_0 \in X$ if and only if its real and imaginary parts
  $p \mapsto \re \big( f(p) \big)$
  and $p \mapsto \im \big( f(p) \big)$ are continuous at $p_0$.
\end{lemma}
\begin{proof}
  \uses{lem:complex_limit}
  % \leanok
  \ldots
\end{proof}

\begin{corollary}[Continuity of coordinate projections]
  \label{cor:coordinate_continuous}
  \uses{def:continuous}
  % \lean{}
  % \leanok
  The coordinate projections
  \begin{align*}
    \re \colon \; & \bC \to \bR
    \qquad\qquad \text{ and } & \quad
    \im \colon \; & \bC \to \bR \\
    & z \mapsto \re(z) &
    & z \mapsto \im(z)
  \end{align*}
  are continuous functions.
\end{corollary}
\begin{proof}
  \uses{lem:continuous_complex}
  % \leanok
  \ldots
\end{proof}

\begin{lemma}[Operations with continuous complex-valued functions]
  \label{lem:complex_continuity_properties}
  \uses{def:continuous, thm:complex_number_field}
  % \lean{}
  % \leanok
  Let $X$ be a metric space, let $p_0 \in X$
  be a point, and let $f,g \colon X \to \bC$
  be two complex-valued functions on $X$
  which are continuous at $p_0$.
  Then also the functions
  \begin{align*}
    p \mapsto f(p) + g(p)
    \quad \text{and} \quad
    p \mapsto f(p) \, g(p)
  \end{align*}
  are continuous at $p_0$.

  If moreover $g(p_0) \ne 0$, then
  also the function
  $p \mapsto \frac{f(p)}{g(p)}$
  is continuous at $p_0$.
\end{lemma}
\begin{proof}
  \uses{lem:continuous_complex, lem:complex_limit_properties}
  \ldots
\end{proof}

\begin{lemma}[Continuity characterization]
  \label{lem:continuity_characterization}
  \uses{def:continuous, def:open_set, def:closed_set}
  % \lean{}
  % \leanok
  Let $X$ and $Y$ be metric spaces, and let $f \colon X \to Y$ be a function.
  Then the following are equivalent:
  \begin{itemize}
    \item $f$ is a continuous function;
    \item for every open set $V \subset Y$, the preimage
      $f^{-1}[V] = \big\{ x \in X \, \big| \, f(x) \in V \big\}$
      is an open set in $X$;
    \item for every closed set $A \subset Y$, the preimage
      $f^{-1}[A] = \big\{ x \in X \, \big| \, f(x) \in A \big\}$
      is a closed set in $X$.
  \end{itemize}
\end{lemma}
\begin{proof}
  % \uses{}
  % \leanok
  See \MetSpCourse{}.
\end{proof}

\begin{lemma}[Composition of continuous functions]
  \label{lem:composition_continuous}
  \uses{def:continuous}
  % \lean{}
  % \leanok
  Let $X$, $Y$, and $Z$ be metric spaces, and
  let $f \colon X \to Y$ and $g \colon Y \to Z$ be functions.
  If $f$ is continuous at $x_0 \in X$ and $g$ is continuous at $f(x_0) \in Y$,
  then the composition $g \circ f \colon X \to Z$ is continuous at $x_0$.

  (The composition $g \circ f$ is defined by
  the formula $(g \circ f)(x) = g \big( f(x) \big)$.)
\end{lemma}
\begin{proof}
  \uses{lem:continuity_characterization}
  % \leanok
  \ldots
\end{proof}

\begin{corollary}[Real multivariate polynomials are continuous]
  \label{cor:multivariate_polynomial_continuous}
  \uses{def:continuous}
  % \lean{}
  % \leanok
  Let $N \in \bN$ be a natural number, and let
  $c_{n,m} \in \bR$ be real numbers for $n,m \in \{0,1,\ldots,N\}$.
  Then the function $p \colon \bC \to \bR$ defined by
  \begin{align*}
    p(x + \ii y) = \sum_{m=0}^N \sum_{n=0}^N c_{m,n} \, x^m \, y^n
  \end{align*}
  is continuous.
\end{corollary}
\begin{proof}
  \uses{lem:composition_continuous, lem:complex_continuity_properties, cor:coordinate_continuous}
  % \leanok
  See \MetSpCourse{}.
\end{proof}

\begin{definition}[Uniform continuity]
  \label{def:uniformly_continuous}
  \uses{def:metric_space}
  %\lean{}
  % \leanok
  Let $X$ and $Y$ be metric spaces.
  A function $f \colon X \to Y$ is \term{uniformly continuous} if
  for every $\eps > 0$ there exists a $\delta > 0$ such that for
  any $p_0 \in X$ and $p \in \BallRC{\delta}{p_0}$
  we have $f(p) \in \BallRC{\eps}{f(p_0)}$.
\end{definition}

\begin{lemma}[Uniform continuity implies continuity]
  \label{lem:uniformly_continuous_implies_continuous}
  \uses{def:continuous, def:uniformly_continuous}
  %\lean{}
  % \leanok
  If a function $f \colon X \to Y$ is uniformly continuous, then it is continuous.
\end{lemma}
\begin{proof}
  See \MetSpCourse{}.

  (The easy proof is also a good exercise.)
\end{proof}


\section{Connectedness and path-connectedness}

\begin{definition}[Connectedness]
  \label{def:connected}
  \uses{def:metric_space}
  %\lean{}
  % \leanok
  %\ldots
  A set $A \subset X$ in a metric space $X$
  is \term{disconnected} if there exists
  a continuous surjective function
  $f \colon A \to \set{0,1}$ onto the two-element discrete
  set $\set{0,1}$.
  Otherwise $A$ is \term{connected};
  then every continuous function $A \to \set{0,1}$ must be
  either constant $0$ or constant $1$.

  (The usual definition in topology textbooks reads
  slightly differently, but it is
  equivalent to the one we chose here by
  Lemma~\ref{lem:continuity_characterization}.)
\end{definition}

\begin{definition}[Path-connectedness]
  \label{def:path_connected}
  \uses{def:path}
  %\lean{}
  % \leanok
  A set $A \subset X$ in a metric space $X$
  is \term{path connected} if for any two points $p, q \in X$
  there exists a continuous function $\gamma \colon [0,1] \to X$
  such that $\gamma(0) = p$ and $\gamma(1) = q$
  (a parametrized path in $X$ starting from $p$ and ending at $q$).
\end{definition}

\begin{lemma}[Path-connectedness implies connectedness]
  \label{lem:path_connected_implies_connected}
  \uses{def:connected, def:path_connected}
  %\lean{}
  % \leanok
  If a metric space $X$ is path-connected, then it is connected.
\end{lemma}
\begin{proof}
  See \MetSpCourse{}.
\end{proof}

\begin{lemma}[Open connected sets are path-connected]
  \label{lem:open_connected_iff_path_connected}
  \uses{def:connected, def:path_connected, def:open_set, lem:metric_on_C}
  %\lean{}
  % \leanok
  Suppose that $U \subset \bC$ is an open subset of the complex plane.
  Then $U$ is connected if and only if it is path-connected.
\end{lemma}
\begin{proof}
  \uses{lem:path_connected_implies_connected}
  See \MetSpCourse{}.
\end{proof}

\section{Compactness}

\begin{definition}[Compactness]
  \label{def:compact}
  \uses{def:limit}
  %\lean{}
  % \leanok
  \ldots
\end{definition}

\begin{theorem}[Bolzano-Weierstrass theorem]
  \label{thm:bolzano_weierstrass}
  \uses{def:compact, def:closed_set, lem:metric_on_C, def:bounded}
  %\lean{}
  % \leanok
  %Let $X$ be a metric space.
  A subset $B \subset \bR$ of the real line is compact
  if an only if it is closed and bounded.

  A subset $A \subset \bC$ of the complex plane is compact
  if an only if it is closed and bounded.
\end{theorem}
\begin{proof}
  % \leanok
  See \MetSpCourse{}.
\end{proof}

\begin{theorem}[Boundedness of continuous functions on compacts]
  \label{thm:continuous_on_compact_implies_bounded}
  \uses{def:compact, def:bounded, def:continuous}
  % \lean{}
  % \leanok
  Suppose that $X$ is compact.
  Then every continuous function $f \colon X \to \bR$
  is bounded.
\end{theorem}
\begin{proof}
  % \uses{}
  % \leanok
  \ldots
\end{proof}

\begin{lemma}[On a compact domain continuity implies uniform continuity]
  \label{lem:on_compact_continuous_implies_uniformly_continuous}
  \uses{def:continuous, def:uniformly_continuous, def:compact}
  %\lean{}
  % \leanok
  If $X$ is compact and a function $f \colon X \to Y$ is continuous, then it is uniformly continuous.
\end{lemma}
\begin{proof}
  See \MetSpCourse{}.
\end{proof}

\begin{lemma}[Continuous bijection from a compact domain is a homeomorphism]
  \label{lem:continuous_bijection_homeomorphism}
  \uses{def:compact, def:continuous}
  % \lean{}
  % \leanok
  Let $X$ and $Y$ be metric spaces and assume that $X$ is compact.
  Then for any continuous bijection $f \colon X \to Y$,
  also the inverse $f^{-1} \colon Y \to X$ is continuous.
\end{lemma}
\begin{proof}
  \uses{lem:continuity_characterization}
  % \leanok
  See \MetSpCourse{}.
\end{proof}

\begin{theorem}[Cantor's intersection theorem]
  \label{thm:intersect_nested_compacts}
  \uses{def:compact}
  %\lean{}
  % \leanok
  \ldots
\end{theorem}
\begin{proof}
  % \uses{}
  % \leanok
  See \MetSpCourse{}.
\end{proof}



\section{Simple connectedness}

%\begin{definition}[Path homotopy]
%  \label{def:path_homotopy}
%  \uses{def:path, def:continuous}
%  % \lean{}
%  % \leanok
%  Let $X$   be a metric space
%\end{definition}


\chapter{Preliminaries from calculus}
\section{Riemann integral}

For the purposes of this course, it suffices to know the Riemann
integral. (Those who already know Lebesgue integration theory can
substitute that more general notion of integral everywhere.)

\begin{definition}[Riemann integral]
  \label{def:riemann_integral}
  %\uses{}
  %\lean{}
  % \leanok
  \ldots
\end{definition}

\begin{lemma}[Riemann integrability of continuous functions]
  \label{lem:continuous_implies_riemann_integrable}
  \uses{def:riemann_integral}
  %\lean{}
  % \leanok
  Any continuous function $f \colon [a,b] \to \bR$ is Riemann
  integrable on $[a,b]$.
\end{lemma}
\begin{proof}
  \uses{lem:on_compact_continuous_implies_uniformly_continuous}
  % \leanok
  See \textbf{MS-C1541 Metric Spaces}.
\end{proof}

