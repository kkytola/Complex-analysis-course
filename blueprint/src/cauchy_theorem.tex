\section{Convex, star-shaped, and simply connected domains}

\begin{definition}[Line segment]
  \label{def:line_segment}
  \uses{lem:complex_number_field, def:path}
  % \lean{}
  % \leanok
  Given two points $z_1, z_2 \in \bC$ in the complex plane,
  the \term{line segment} between $z_1$ and $z_2$ is the path
  \begin{align*}
    \gamma \colon [0,1] & \; \to \bC \\
    \gamma(t) & \; = z_1 + t \, (z_2 - z_1) .
  \end{align*}

  We also often view the line segment as a subset of~$\bC$
  rather than a parametrized path, and we then denote it by
  \begin{align*}
    [z_1, z_2] = \set{z_1 + t \, (z_2 - z_1) \; \big| \; t \in [0,1] } .
  \end{align*}
\end{definition}

\begin{definition}[Convex set]
  \label{def:convex}
  \uses{def:line_segment}
  % \lean{}
  % \leanok
  A subset $A \subset \bC$ of the complex plane is called
  \term{convex} if for any two points $z_1, z_2 \in A$,
  the line segment between them is contained in the subset,
  \begin{align*}
    [z_1, z_2] \, \subset \, A .
  \end{align*}
\end{definition}

\begin{definition}[Star-shaped set]
  \label{def:star_shaped}
  \uses{def:line_segment}
  % \lean{}
  % \leanok
  A subset $A \subset \bC$ of the complex plane is called
  \term{star-shaped} if there exists a point $z_* \in A$
  such that for any $z \in A$,
  the line segment between $z_*$ and $z$ is contained in the subset,
  \begin{align*}
    [z_*, z] \, \subset \, A .
  \end{align*}
\end{definition}

\begin{lemma}[Convex sets are star-shaped]
  \label{lem:convex_implies_star_shaped}
  \uses{def:convex, def:star_shaped}
  % \lean{}
  % \leanok
  Any nonempty convex set is star-shaped.
\end{lemma}
\begin{proof}
  % \uses{}
  % \leanok
  \ldots
\end{proof}



\section{Cauchy's integral theorem}

\begin{lemma}[Goursat's lemma \Palka{Lem~V.1.1}]
  \label{lem:goursat}
  \uses{def:analytic_function, def:contour_integral}
  %\lean{}
  % \leanok
  Suppose that a function $f \colon U \to \bC$ is analytic on an open
  set $U \subset \bC$.
  Then for any closed triangle $\Tri \subset U$, we have
  \begin{align*}
    \oint_{\bdry \Tri} f(z) \, \ud z = 0 .
  \end{align*}
\end{lemma}

\begin{theorem}[Cauchy's integral theorem \Palka{Thm~V.1.5}]
  \label{thm:cauchy_theorem}
  \uses{def:analytic_function, def:contour_integral}
  %\lean{}
  % \leanok
  Suppose that a function $f \colon U \to \bC$ is analytic on an open
  set $U \subset \bC$ containing the disk $\DiskRC{r}{z} \subset U$.
  Then for any closed contour $\gamma$ in $\DiskRC{r}{z}$ we have
  \begin{align*}
    \oint_\gamma f(z) \, \ud z = 0 .
  \end{align*}
\end{theorem}
\begin{proof}
  \uses{lem:goursat}
  % \leanok
  \ldots
\end{proof}



\section{Cauchy's integral formula}

\begin{definition}[Winding number]
  \label{def:winding_number}
  \uses{def:contour_integral}
  % \lean{}
  % \leanok
  Let $z \in \bC$, and let $\gamma$ be a closed contour
  in $\bC \setminus \set{z}$.
  The \term{winding number} of $\gamma$ around $z$ is defined as
  \begin{align*}
    \windN{\gamma}{z} = \frac{1}{2\pi \ii} \oint_\gamma \frac{\ud \zeta}{\zeta - z} .
  \end{align*}
\end{definition}

\begin{theorem}[Cauchy's integral formula \Palka{Thm~V.2.3}]
  \label{thm:CAUCHY_FORMULA}
  \uses{def:analytic_function, def:contour_integral, def:winding_number}
  %\lean{}
  % \leanok
  Suppose that a function $f \colon U \to \bC$ is analytic on an open
  set $U \subset \bC$ containing the disk $\DiskRC{r}{z} \subset U$.
  Then for any closed contour $\gamma$ in $\DiskRC{r}{z}$ we have
  \begin{align*}
    \windN{\gamma}{z} \, f(z)
      = \frac{1}{2\pi \ii} \oint_\gamma \frac{f(\zeta)}{\zeta - z} \, \ud \zeta .
  \end{align*}
\end{theorem}
\begin{proof}
  \uses{thm:cauchy_theorem}
  % \leanok
  \ldots
\end{proof}
