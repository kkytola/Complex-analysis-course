\section{Convex, star-shaped, and simply connected domains}

\begin{definition}[Line segment]
  \label{def:line_segment}
  \uses{thm:complex_number_field, def:path}
  % \lean{}
  % \leanok
  Given two points $z_1, z_2 \in \bC$ in the complex plane,
  the \term{line segment} between $z_1$ and $z_2$ is the path
  \begin{align*}
    \gamma \colon [0,1] & \; \to \bC \\
    \gamma(t) & \; = z_1 + t \, (z_2 - z_1) .
  \end{align*}

  We also often view the line segment as a subset of~$\bC$
  rather than a parametrized path, and we then denote it by
  \begin{align*}
    [z_1, z_2]
    \, = \, \Big\{ z_1 + t \, (z_2 - z_1) \; \Big| \; t \in [0,1] \Big\}
    \, \subset \, \bC.
  \end{align*}
\end{definition}

\begin{definition}[Convex set]
  \label{def:convex}
  \uses{def:line_segment}
  % \lean{}
  % \leanok
  A subset $A \subset \bC$ of the complex plane is called
  \term{convex} if for any two points $z_1, z_2 \in A$,
  the line segment between them is contained in the subset,
  \begin{align*}
    [z_1, z_2] \, \subset \, A .
  \end{align*}
\end{definition}

\begin{definition}[Star-shaped set]
  \label{def:star_shaped}
  \uses{def:line_segment}
  % \lean{}
  % \leanok
  A subset $A \subset \bC$ of the complex plane is called
  \term{star-shaped} if there exists a point $z_* \in A$
  such that for any $z \in A$,
  the line segment between $z_*$ and $z$ is contained in the subset,
  \begin{align*}
    [z_*, z] \, \subset \, A .
  \end{align*}
\end{definition}

\begin{lemma}[Convex sets are star-shaped]
  \label{lem:convex_implies_star_shaped}
  \uses{def:convex, def:star_shaped}
  % \lean{}
  % \leanok
  Any nonempty convex set is star-shaped.
\end{lemma}
\begin{proof}
  % \uses{}
  % \leanok
  \ldots
\end{proof}

\begin{lemma}[Star-shaped sets are path connected and simply connected]
  \label{lem:star_shaped_implies_simply_connected}
  \uses{def:simply_connected, def:star_shaped, def:path_connected}
  % \lean{}
  % \leanok
  Any star-shaped set $U \subset \bC$ is path connected and simply connected.
\end{lemma}
\begin{proof}
  % \uses{}
  % \leanok
  \ldots
\end{proof}



\section{Cauchy's integral theorem}

\begin{lemma}[Goursat's lemma \Palka{Lem~V.1.1}]
  \label{lem:goursat}
  \uses{def:analytic_function, def:contour_integral}
  %\lean{}
  % \leanok
  Suppose that a function $f \colon U \to \bC$ is analytic on an open
  set $U \subset \bC$.
  Then for any closed triangle $\Tri \subset U$, we have
  \begin{align*}
    \oint_{\bdry \Tri} f(z) \, \ud z = 0 .
  \end{align*}
\end{lemma}
\begin{proof}
  \uses{thm:intersect_nested_compacts, lem:local_linear_approximation}
  % \leanok
  \ldots
\end{proof}

\begin{lemma}[Existence of primitives in star-shaped domains]
  \label{lem:primitive_star_shaped}
  \uses{def:primitive, def:star_shaped}
  % \lean{}
  % \leanok
  Every analytic function $f \colon U \to \bC$ on a star-shaped
  domain~$U \subset \bC$ has a primitive in~$U$.
\end{lemma}
\begin{proof}
  \uses{lem:goursat}
  % \leanok
  \ldots
\end{proof}

\begin{theorem}[Cauchy's integral theorem for star-shaped domains \Palka{Thm~V.1.5}]
  \label{thm:cauchy_theorem_star_shaped}
  \uses{def:analytic_function, def:contour_integral, def:star_shaped}
  %\lean{}
  % \leanok
  Suppose that a function $f \colon U \to \bC$ is analytic on a
  star-shaped open subset $U \subset \bC$.
  Then for any closed contour $\gamma$ in~$U$ we have
  \begin{align*}
    \oint_\gamma f(z) \, \ud z = 0 .
  \end{align*}
\end{theorem}
\begin{proof}
  \uses{thm:primitive_characterization, lem:primitive_star_shaped}
  % \leanok
  \ldots
\end{proof}

\begin{corollary}[Local Cauchy's integral theorem \Palka{Thm~V.5.1}]
  \label{cor:local_cauchy_theorem}
  \uses{def:analytic_function, def:contour_integral, def:contractible}
  %\lean{}
  % \leanok
  Suppose that a function $f \colon U \to \bC$ is analytic on a open
  set $U \subset \bC$.
  Then for any disk $B \subset U$ contained in the domain $U$
  and any closed contour $\gamma$ in $B$ we have
  \begin{align*}
    \oint_\gamma f(z) \, \ud z = 0 .
  \end{align*}
\end{corollary}
\begin{proof}
  \uses{lem:goursat, thm:cauchy_theorem_star_shaped}
  % \leanok
  \ldots
\end{proof}



\section{Cauchy's integral formula}

\begin{theorem}[Cauchy's integral formula for star-shaped subdomains \Palka{Thm~V.2.3}]
  \label{thm:cauchy_formula_star_shaped}
  \uses{def:analytic_function, def:contour_integral}
  %\lean{}
  % \leanok
  Suppose that a function $f \colon U \to \bC$ is analytic on an open
  set $U \subset \bC$, and suppose that $\gamma$ is a closed contour
  in $U$ parametrizing the boundary of a star-shaped Jordan domain
  $V \subset U$ in a counterclockwise orientation.
  Then for any point $z \in V$ we have
  \begin{align*}
    f(z) = \frac{1}{2\pi \ii} \oint_\gamma \frac{f(\zeta)}{\zeta - z} \, \ud \zeta .
  \end{align*}
\end{theorem}
\begin{proof}
  \uses{thm:cauchy_theorem_star_shaped}
  % \leanok
  \ldots
\end{proof}

By far the most commonly used special case of
Theorem~\ref{thm:cauchy_formula_star_shaped}
is when the contour $\gamma$ is a circle, encircling a disk
whose closure is contained in the domain of the analytic function
(recall that disks are convex and therefore star-shaped).

\begin{corollary}[Cauchy's integral formula for circles]
  \label{cor:cauchy_formula_circle}
  \uses{def:analytic_function, def:contour_integral}
  % \lean{}
  % \leanok
  Suppose that a function $f \colon U \to \bC$ is analytic on an open
  set $U \subset \bC$.
  Let $\CDiskRC{r}{z_0} \subset U$ be a closed disk contained in~$U$.
  Then for any point $z \in \DiskRC{r}{z_0}$ we have
  \begin{align*}
    f(z) = \frac{1}{2\pi \ii}
        \oint_{\bdry \DiskRC{z_0}{r}} \frac{f(\zeta)}{\zeta - z} \, \ud \zeta
  \end{align*}
  where the circle $\bdry \DiskRC{z_0}{r}$ is parametrized in the
  counterclockwise orientation.
\end{corollary}
\begin{proof}
  \uses{thm:cauchy_formula_star_shaped}
  % \leanok
  \ldots
\end{proof}



\section{Ideas underlying the generalizations}

The generalizations of Cauchy's integral theorem and Cauchy's integral
formula are based on the following homotopy invariance property of
contour integrals.

\begin{lemma}[Homotopy invariance of contour integrals]
  \label{lem:homotopy_invariance}
  \uses{def:path_homotopy, def:contour_integral}
  % \lean{}
  % \leanok
  Let $f \colon U \to \bC$ be an analytic function on an open
  set $U \subset \bC$, and let $\gamma_0$ and $\gamma_1$ be two
  closed contours in $U$ which are homotopic to each other in $U$.
  Then we have
  \begin{align*}
    \oint_{\gamma_0} f(z) \, \ud z = \oint_{\gamma_1} f(z) \, \ud z .
  \end{align*}
\end{lemma}

This readily implies the following generalization of Cauchy's integral
theorem.

\begin{theorem}[Cauchy's integral theorem \Palka{Thm~V.5.1}]
  \label{thm:cauchy_theorem}
  \uses{def:analytic_function, def:contour_integral, def:contractible}
  %\lean{}
  % \leanok
  Suppose that a function $f \colon U \to \bC$ is analytic on a open
  set $U \subset \bC$.
  Then for any contractible closed contour $\gamma$ we have
  \begin{align*}
    \oint_\gamma f(z) \, \ud z = 0 .
  \end{align*}

  In particular, if $U$ is simply connected, then
  for any closed contour $\gamma$ in~$U$ we have
  \begin{align*}
    \oint_\gamma f(z) \, \ud z = 0 ,
  \end{align*}
  and the analytic function $f$ has a primitive in~$U$.
\end{theorem}
\begin{proof}
  \uses{lem:goursat, cor:local_cauchy_theorem, lem:homotopy_invariance}
  % \leanok
  \ldots
\end{proof}

The other ingredient of generalization of Cauchy's integral formula
to arbitrary contours and points not lying on those contours is
the winding number of a contour around a point.

\begin{definition}[Winding number]
  \label{def:winding_number}
  \uses{def:contour_integral}
  % \lean{}
  % \leanok
  Let $z \in \bC$, and let $\gamma$ be a closed contour
  in $\bC \setminus \set{z}$.
  The \term{winding number} of $\gamma$ around $z$ is defined as
  \begin{align*}
    \windN{\gamma}{z} = \frac{1}{2\pi \ii} \oint_\gamma \frac{\ud \zeta}{\zeta - z} .
  \end{align*}
\end{definition}

\begin{lemma}[Winding number concatenation and reversal]
  \label{lem:winding_additivity}
  \uses{def:winding_number}
  % \lean{}
  % \leanok
  Let $\gamma$ and $\eta$ be closed contours in $\bC$
  both starting and ending at the same point $z_0 \in \bC$.
  Then for any point $z \in \bC \setminus (\gamma \cup \eta)$
  we have
  \begin{align*}
    \windN{\gamma \concat \eta}{z}
    \; = \; \windN{\gamma}{z} + \windN{\eta}{z}
  \end{align*}
  and for any point $z \in \bC \setminus \gamma$ we have
  \begin{align*}
    \windN{\opp{\gamma}}{z}
    \; = \; -\windN{\gamma}{z} .
  \end{align*}
\end{lemma}
\begin{proof}
  % \uses{}
  % \leanok
  \ldots
\end{proof}

\begin{lemma}[Winding number properties]
  \label{lem:winding_number_properties}
  \uses{def:winding_number}
  % \lean{}
  % \leanok
  Let $\gamma$ be a closed contour in $\bC$. Then the winding
  numbers $\windN{\gamma}{z}$ of points $z \in \bC \setminus \gamma$
  satisfy:
  \begin{itemize}
    \item[(a)] $z \mapsto \windN{\gamma}{z}$ is constant on
      each connected component of $\bC \setminus \gamma$;
    \item[(b)] $\windN{\gamma}{z} = 0$ for all $z$ in the
      unbounded connected component of $\bC \setminus \gamma$;
    \item[(c)] If $\gamma$ is a Jordan contour and
      $V \subset \bC \setminus \gamma$ is the bounded connected
      component of $\bC \setminus \gamma$, then
      either $\windN{\gamma}{z} = 1$ for all $z \in V$
      or $\windN{\gamma}{z} = -1$ for all $z \in V$.
  \end{itemize}
\end{lemma}
\begin{proof}
  \uses{lem:derivative_auxiliary_integral}
  % \leanok
  \ldots
\end{proof}

\begin{lemma}[Homotopy invariance of winding numbers]
  \label{lem:winding_homotopy_invariance}
  \uses{def:winding_number, def:path_homotopy}
  % \lean{}
  % \leanok
  Let $z \in \bC$ be a point and let $\gamma$ and $\eta$ be two
  closed contours in $\bC \setminus \set{z}$ which are homotopic
  to each other in $\bC \setminus \set{z}$. Then we have
  \begin{align*}
    \windN{\gamma}{z} = \windN{\eta}{z} .
  \end{align*}
\end{lemma}
\begin{proof}
  \uses{lem:homotopy_invariance}
  % \leanok
  \ldots
\end{proof}

The following is then a version of Cauchy's integral formula
which has no restrictions on the closed contour and
no restrictions on the position of the point with respect to
the contour, except that the point must not lie on the contour
(for otherwise there is a singularity in the integrand).

\begin{theorem}[Cauchy's integral formula \Palka{Thm~V.2.3}]
  \label{thm:CAUCHY_FORMULA}
  \uses{def:analytic_function, def:contour_integral, def:winding_number}
  %\lean{}
  % \leanok
  Suppose that a function $f \colon U \to \bC$ is analytic on an open
  set $U \subset \bC$ containing the disk $\DiskRC{r}{z} \subset U$.
  Then for any closed contour $\gamma$ in $\DiskRC{r}{z}$ we have
  \begin{align*}
    \oint_\gamma \frac{f(\zeta)}{\zeta - z} \, \ud \zeta
    \; = \; 2\pi \ii \, \windN{\gamma}{z} \, f(z) .
  \end{align*}
\end{theorem}
\begin{proof}
  \uses{thm:cauchy_theorem, thm:cauchy_formula_star_shaped}
  % \leanok
  \ldots
\end{proof}
