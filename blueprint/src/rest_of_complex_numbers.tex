\section{The algebra and geometry of complex numbers}


\subsection{The field of complex numbers}


\subsection{Conjugate, modulus, and argument}

\begin{definition}[Complex conjugate \Palka{Sec.~I.1.2}]
  \label{def:complex_conjugate}
  \uses{def:complex_numbers}
  %\lean{}
  % \leanok
  The \term{complex conjugate} of a complex number $z = x + \ii y$
  (where $x,y \in \bR$) is the complex number $\cconj{z} = x - \ii y$.
\end{definition}


\begin{lemma}[Properties of complex conjugate \Palka{Sec.~I.1.2~(1.1)}]
  \label{lem:complex_conjugate_properties}
  \uses{def:complex_conjugate}
  %\lean{}
  % \leanok
  For any $z , w \in \bC$, we have
  \begin{align*}
    \cconj{\cconj{z}} = z , \qquad
    \cconj{z + w} = \cconj{z} + \cconj{w} , \qquad
    \cconj{z w} = \cconj{z} \, \cconj{w} ,
  \end{align*}
  \begin{align*}
    \re (z) = \frac{z + \cconj{z}}{2} , \qquad
    \im (z) = \frac{z - \cconj{z}}{2 \ii} .
  \end{align*}
\end{lemma}
\begin{proof}
  %\uses{}
  % \leanok
  Direct calculation.
\end{proof}

\begin{definition}[Absolute value (modulus) \Palka{Sec.~I.1.2}]
  \label{def:absolute_value}
  \uses{def:complex_numbers}
  %\lean{}
  % \leanok
  The \term{absolute value} (or \term{modulus}) of a
  complex number $z = x + \ii y$ (where $x,y \in \bR$)
  is the nonnegative real number $|z| = \sqrt{x^2 + y^2}$.
\end{definition}

\begin{lemma}[Properties of absolute value \Palka{Sec.~I.1.2~(1.2)}]
  \label{lem:absolute_value_properties}
  \uses{def:absolute_value, def:complex_conjugate}
  %\lean{}
  % \leanok
  For any $z, w \in \bC$, we have
  \begin{align*}
    |z|^2 = z \, \cconj{z} , \qquad
    |z w| = |z| \, |w| ,
  \end{align*}
  \begin{align*}
    \re(z) \le |z| , \qquad
    \im(z) \le |z| ,
  \end{align*}
  \begin{align*}
    |z + w| \le |z| + |w| , \qquad
    |z + w| \ge \big| |z| - |w| \big| .
  \end{align*}
  Also, if $z \ne 0$, then
  \begin{align*}
    z^{-1} = \frac{\cconj{z}}{|z|^2} , \qquad
    \qquad
    \Big| \frac{w}{z} \Big| = \frac{|w|}{|z|} .
  \end{align*}
\end{lemma}
\begin{proof}
  %\uses{}
  % \leanok
  Straightforward.
\end{proof}

\begin{definition}[Argument \Palka{Sec.~I.1.2}]
  \label{def:argument}
  \uses{def:complex_numbers, def:absolute_value}
  %\lean{}
  % \leanok
  A real number $\theta \in \bR$ is an \term{argument} of a complex
  number $z \in \bC$ if
  \begin{align*}
    z = |z| \, \Big( \cos (\theta) + \ii \, \sin (\theta) \Big) .
  \end{align*}

  (Note: the argument of a nonzero complex number is defined only
  modulo addition of integer multiples of $2 \pi$.)

  The \term{principal argument} of a nonzero complex number $z \in \bC$
  is its unique argument on the interval $(-\pi, \pi]$, and it is denoted
  by $\Arg(z)$.
\end{definition}

\begin{lemma}[Multiplication in polar form \Palka{Sec.~I.1.2~(1.6)}]
  \label{lem:multiplication_polar}
  \uses{thm:complex_number_field, def:argument, def:absolute_value}
  %\lean{}
  % \leanok
  For any $z,w \in \bC$, we have
  \begin{align*}
    |zw| = |z| \, |w|
    \qquad \text{ and } \qquad
    \Arg(zw) = \Arg(z) + \Arg(w) .
  \end{align*}
\end{lemma}

\begin{lemma}[De Moivre's formula \Palka{Sec.~I.1.2~(1.7)}]
  \label{lem:de_moivre}
  \uses{thm:complex_number_field}
  %\lean{}
  % \leanok
  For any $\theta \in \bR$ and $n \in \bZ$, we have
  \begin{align*}
    \big( \cos \theta + i \sin \theta \big)^n = \cos(n \theta) + i \sin(n \theta) .
  \end{align*}
\end{lemma}
\begin{proof}
  \uses{lem:multiplication_polar}
  % \leanok
  Straightforward induction using Lemma~\ref{lem:multiplication_polar}.
\end{proof}

%\begin{itemize}
%  \item $n$th roots of unity / of a complex number
%\end{itemize}



\section{Functions of a complex variable}


\subsection{Exponentials and logarithms}

\begin{definition}[Complex exponential]
  \label{def:complex_exp}
  % \uses{}
  % \lean{}
  % \leanok

  foo.
\end{definition}


\subsection{Polynomials and rational functions}

\begin{definition}[Polynomial]
  \label{def:polynomial}
  \uses{thm:complex_number_field}
  %\lean{}
  % \leanok
  Polynomial functions are functions $p : \bC \to \bC$ of the form
  \begin{align*}
    p(z) = a_n z^n + a_{n-1} z^{n-1} + \cdots + a_1 z + a_0
  \end{align*}
  where $a_0,a_1,\ldots,a_{n-1},a_n \in \bC$ are coefficients.
\end{definition}

\begin{definition}[Rational function]
  \label{def:rational_function}
  \uses{def:polynomial}
  %\lean{}
  % \leanok
  Rational functions are functions $f : D \to \bC$ which can be written as
  ratios $f(z) = \frac{p(z)}{q(z)}$ of two polynomials
  $p, q \colon \bC \to \bC$ on a domain $D \subset \bC$
  where the denominator polynomial $q$ has no zeroes.
\end{definition}
