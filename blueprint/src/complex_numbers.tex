\section{The field of complex numbers}

\begin{definition}[Complex numbers and their arithmetic operations \Palka{Sec. I.1.1}]
  \label{def:complex_numbers}
  %\uses{}
  %\lean{}
  % \leanok

  The set of \term{complex numbers} is $\bC = \bR \times \bR$, i.e.,
  the set of pairs $(x,y)$ of real numbers $x,y \in \bR$.

  The operations of addition and multiplication on $\bC$ are defined by the formulas
  \begin{align*}
    (x_1,y_2) + (x_2,y_2) = \; & (x_1+x_2, y_1+y_2) \\
    (x_1,y_1) \cdot (x_2,y_2) = \; & (x_1 x_2 - y_1 y_2 , \, x_1 y_2 + y_1 x_2) .
  \end{align*}
  %(x_1,y_2) + (x_2,y_2) = (x_1+x_2, y_1+y_2)
  %$(x_1,y_1) \cdot (x_2,y_2) = (x_1 x_2 - y_1 y_2 , \, x_1 y_2 + y_1 x_2)$.

  Denote
  $0 = (0,0) \in \bC$ and $1 = (1,0) \in \bC$.

  For $z = (x,y) \in \bC$, denote
  $-z %= -x - \ii \, y
    = (-x,-y) \in \bC$ and if $z \ne 0$ then denote
  $z^{-1} %= \frac{x}{x^2 + y^2} + \ii \frac{-y}{x^2 + y^2}
            = \Big(\frac{x}{x^2 + y^2} , \frac{-y}{x^2 + y^2} \Big) \in \bC$.

  We write a complex number $(x,y)$ as $x + \ii \, y$.
  The compex number $\ii = (0,1) \in \bC$ is called the \term{imaginary unit}.
  %Denote
  %$0 = 0 + 0 \, \ii = (0,0) \in \bC$ and $1 = 1 + 0 \, \ii = (1,0) \in \bC$.
  \end{definition}

Typically used variable names for complex number are $z,w,\zeta \in \bC$ etc.

\begin{theorem}[The field of complex numbers \Palka{Sec. I.1.1}]
  \label{thm:complex_number_field}
  \uses{def:complex_numbers}
  %\lean{}
  %\leanok

  The set $\bC$ of compex numbers with its operations of addition and multiplication,
  is a \term{field}, i.e., the following properties hold for
  all $z, w, z_1, z_2, z_3 \in \bC$:
  \begin{itemize}
    \item $z + w = w + z$ \term{(commutativity of addition)}
    \item $z w = w z$ \term{(commutativity of multiplication)}
    \item $z_1 + (z_2 + z_3) = (z_1 + z_2) + z_3$ \term{(associativity of addition)}
    \item $z_1 (z_2 z_3) = (z_1 z_2) z_3$ \term{(associativity of multiplication)}
    \item $0 = 0 + 0 \, \ii = (0,0) \in \bC$ satisfies $z + 0 = z$
          \term{(neutral element for addition)}
    \item $1 = 1 + 0 \, \ii = (1,0) \in \bC$ satisfies $z \cdot 1= z$
          \term{(neutral element for multiplication)}
    \item %for any $z = x + \ii y$, the element $-z %= -x - \ii \, y
          %= (-x,-y) \in \bC$ satisfies
          $z + (-z) = 0$ for any $z \in \bC$ \term{(opposite element / additive inverse)}
    \item %for any $z = x + \ii y \ne 0$, the element
          %  $z^{-1} %= \frac{x}{x^2 + y^2} + \ii \frac{-y}{x^2 + y^2}
          %  = \Big(\frac{x}{x^2 + y^2} , \frac{-y}{x^2 + y^2} \Big) \in \bC$ satisfies
          $z \, z^{-1} = 1$ for any $z \in \bC \setminus \set{0}$ \term{(multiplicative inverse)}
    \item $(z_1 + z_2) w = z_1 w + z_2 w$ \term{(distributivity)}.
  \end{itemize}
\end{theorem}
\begin{proof}
  Straightforward calculations using the definitions of the operations
  (\defref{def:complex_numbers}).
  %\uses{}
  % \leanok
\end{proof}
