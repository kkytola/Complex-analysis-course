\section{Uniform convergence}

\begin{definition}[Uniform convergence \Palka{Sec.~VII.1.1}]
  \label{def:uniform_limit}
  \uses{def:limit}
  %\lean{}
  % \leanok
  Let $(f_n)_{n \in \bN}$ be a sequence of functions
  $f_n \colon X \to \bC$, and let $f \colon X \to \bC$
  also be a such function. We say that the sequence
  $(f_n)_{n \in \bN}$ \term{converges uniformly} to $f$
  (on $X$) if for every $\eps > 0$ there exists an $N \in \bN$
  such that for all $n \ge N$ we have
  \begin{align*}
    \big| f_n(x) - f(x) \big| < \eps
    \qquad \text{ for all } x \in X .
  \end{align*}
\end{definition}

\begin{lemma}[Cauchy criterion for uniform convergence \Palka{Thm~VII.1.2}]
  \label{lem:cauchy_uniform_limit_criterion}
  \uses{def:uniform_limit}
  %\lean{}
  % \leanok
  Let $f_n \colon A \to \bC$, $n \in \bN$,
  be complex-valued functions defined on the same set $A$.
  Then the sequence $(f_n)_{n \in \bN}$ converges uniformly on $A$
  if and only if for every $\eps > 0$ there exists an $N \in \bN$
  such that for all $m,n \ge N$ and all $z \in A$
  we have $|f_n(z) - f_m(z)| < \eps$.

  (When $(f_n)_{n \in \bN}$ satisfies the condition above, it could
  be called a \term{uniform Cauchy sequence} on~$A$.)
\end{lemma}
\begin{proof}
  \uses{lem:complex_complete}
  % \leanok
  \ldots
\end{proof}

\begin{lemma}[Continuity is preserved in uniform limits \Palka{Thm~VII.1.1}]
  \label{lem:uniform_limit_continuous}
  \uses{def:uniform_limit, def:continuous}
  %\lean{}
  % \leanok
  Let $X$ be a metric space (e.g., $\bR$, $\bC$, or a subset of these).
  If a sequence $(f_n)_{n \in \bN}$ of continuous functions
  $f_n \colon X \to \bC$ converges uniformly to a function
  $f \colon X \to \bC$, then $f$ is continuous.
\end{lemma}
\begin{proof}
  %\uses{def:uniform_limit}
  % \leanok
  %\ldots
  See \textbf{MS-C1541 Metric Spaces}.
\end{proof}

\begin{lemma}[Integration commutes with uniform limits \Palka{Thm~VII.1.1}]
  \label{lem:uniform_limit_integral}
  \uses{def:uniform_limit, def:continuous, def:riemann_integral, lem:uniform_limit_continuous}
  %\lean{}
  % \leanok
  If a sequence $(f_n)_{n \in \bN}$ of continuous functions
  $f_n \colon [a,b] \to \bC$ on a closed interval $[a,b] \subset \bR$
  converges uniformly to a function
  $f \colon [a,b] \to \bC$, then we have
  \begin{align*}
    \lim_{n \to \infty} \int_a^b f_n(x) \, \ud x
      = \int_a^b f(x) \, \ud x .
  \end{align*}
\end{lemma}
\begin{proof}
  %\uses{}
  % \leanok
  \ldots
\end{proof}

\begin{corollary}[Contour integration commutes with uniform limits \Palka{Thm~VII.1.1}]
  \label{cor:uniform_limit_contour_integral}
  \uses{def:uniform_limit, def:continuous, def:contour_integral}
  %\lean{}
  % \leanok
  %Let $A \subset \bC$ be a subset of the complex plane.
  If a sequence $(f_n)_{n \in \bN}$ of continuous functions
  $f_n \colon A \to \bC$ on a subset $A \subset \bC$  of the complex plane
  converges uniformly to a function
  $f \colon A \to \bC$, then for any piecewise smooth path $\gamma$
  in $A$ we have
  \begin{align*}
    \lim_{n \to \infty} \int_\gamma f_n(z) \, \ud z
      = \int_\gamma f(z) \, \ud z .
  \end{align*}
\end{corollary}
\begin{proof}
  \uses{lem:uniform_limit_integral}
  % \leanok
  This follows straightforwardly from the definition of contour integration
  and Lemma~\ref{lem:uniform_limit_integral} above.
\end{proof}

\begin{definition}[Convergence uniformly on compacts \Palka{Sec.~VII.1.2}]
  \label{def:uoc_limit}
  \uses{def:uniform_limit}
  %\lean{}
  % \leanok
  Let $(f_n)_{n \in \bN}$ be a sequence of functions
  $f_n \colon A \to \bC$ on $A \subset \bC$, and
  let $f \colon A \to \bC$ also be a such function. We say
  that the sequence   $(f_n)_{n \in \bN}$ \term{converges uniformly
  on compacts} (UOC) to $f$ if for every compact subset $K \subset A$
  the restrictions $f_n|_K \colon K \to \bC$ converge
  uniformly on $K$ to $f|_K \colon K \to \bC$. We then write
  \begin{align*}
    f_n \xrightarrow{\UOC} f \qquad \text{ as } n \to \infty .
  \end{align*}

  (This notion is also called by the alternative names
  \term{locally uniform convergence} and \term{normal convergence}.)
\end{definition}

\begin{lemma}[UOC limit of analytic functions \Palka{Thm~VII.3.1}]
  \label{lem:uoc_limit_analytic}
  \uses{def:uoc_limit, def:analytic_function}
  %\lean{}
  Suppose that functions $f_1, f_2, \ldots \colon U \to \bC$
  are analytic functions on an open set $U \subset \bC$
  and the sequence $(f_n)_{n \in \bN}$ converges uniformly
  on compacts to a function $f$. Then $f$ is analytic on $U$.
  Moreover, for any $k \in \bN$, the sequence $(f^{(k)}_n)_{n \in \bN}$
  of $k$th derivatives converges uniformly on compacts to $f^{(k)}$.
  % \leanok
\end{lemma}
\begin{proof}
  \uses{cor:uniform_limit_contour_integral, thm:cauchy_formula_derivative}
  % \leanok
  \ldots
\end{proof}



\section{Complex series}

\begin{definition}[Complex series \Palka{Sec.~VII.2.1}]
  \label{def:complex_series}
  \uses{def:limit, thm:complex_number_field}
  %\lean{}
  % \leanok
  Let $z_1, z_2, z_3, \ldots \in \bC$ be complex numbers.
  For $N \in \bN$, define the $N$th \term{partial sum} of these as
  \begin{align*}
    S_N \; = \; \sum_{n=1}^N z_n \; = \; z_1 + z_2 + \cdots + z_N \, .
  \end{align*}
  We say that the series $\sum_{n=1}^\infty z_n$ \term{converges}
  if the sequence $(S_N)_{N \in \bN}$ of partial sums has
  a limit, and we then denote
  \begin{align*}
    \sum_{n=1}^\infty z_n = \lim_{N \to \infty} \; \sum_{n=1}^N z_n \, .
  \end{align*}

  (Obvious modifications to the above definition are made
  if the terms' indexing starts from $n=0$ or some other index, and
  the notation is correspondingly changed to, e.g., $\sum_{n=0}^\infty$.)
\end{definition}

\begin{lemma}[Terms of a convergent series tend to zero]
  \label{lem:terms_tend_to_zero}
  \uses{def:complex_series}
  % \lean{}
  % \leanok
  If a complex series $\sum_{n=1}^\infty z_n$ converges,
  then we have
  \begin{align*}
    \lim_{n \to \infty} z_n = 0 .
  \end{align*}
\end{lemma}
\begin{proof}
  % \uses{}
  % \leanok
  \ldots
\end{proof}

\begin{lemma}[Geometric series]
  \label{lem:geometric_series}
  \uses{def:complex_series}
  % \lean{}
  % \leanok
  The geometric series
  \begin{align*}
    \sum_{n=0}^\infty z^n \; = \; 1 + z + z^2 + z^3 + \cdots ,
  \end{align*}
  with ratio $z \in \bC$ converges if and only if
  $|z| < 1$. In that case its sum is
  \begin{align*}
    \sum_{n=0}^\infty z^n \; = \; \frac{1}{1-z} .
  \end{align*}
\end{lemma}
\begin{proof}
  % \uses{}
  % \leanok
  \ldots
\end{proof}

\begin{definition}[Absolute convergence of a complex series]
  \label{def:absolute_convergence}
  \uses{def:complex_series, def:absolute_value}
  % \lean{}
  % \leanok
  A complex series $\sum_{n=1}^\infty z_n$ is said to
  \term{converge absolutely} if the series of absolute values
  $\sum_{n=1}^\infty |z_n|$ converges.
\end{definition}

\begin{lemma}[Absolute convergence implies convergence]
  \label{lem:absolute_convergence_implies_convergence}
  \uses{def:absolute_convergence, def:complex_series, }
  % \lean{}
  % \leanok
  If a complex series %$\sum_{n=1}^\infty z_n$
  converges absolutely, then it converges.
\end{lemma}
\begin{proof}
  \uses{lem:complex_complete}
  % \leanok
  \ldots
\end{proof}

\begin{lemma}[D'Alembert's ratio test]
  \label{lem:ratio_test}
  \uses{def:complex_series, def:absolute_convergence}
  % \lean{}
  % \leanok
  Suppose that $\sum_{n=1}^\infty z_n$ is a complex
  series such that the limit
  \begin{align*}
    r = \lim_{n \to \infty} \frac{|z_{n+1}|}{|z_n|}
  \end{align*}
  exists. Then:
  \begin{itemize}
    \item[(i)] If $r < 1$, then the series $\sum_{n=1}^\infty z_n$ converges absolutely.
    \item[(ii)] If $r > 1$, then the series $\sum_{n=1}^\infty z_n$ does not converge.
  \end{itemize}
\end{lemma}
\begin{proof}
  \uses{lem:terms_tend_to_zero, lem:geometric_series, lem:absolute_convergence_implies_convergence}
  % \leanok
  \ldots
\end{proof}



\section{Series of functions}

\begin{definition}[Series of functions \Palka{Sec.~VII.2.2}]
  \label{def:function_series}
  \uses{def:limit, def:uniform_limit, def:complex_series}
  %\lean{}
  % \leanok
  Let $f_1, f_2, f_3, \ldots$ be complex-valued functions on a set~$A$.
  For $N \in \bN$,
  define their $N$th \term{partial sum} function $F_N \colon A \to \bC$ by
  \begin{align*}
    F_N(z) \; = \; \sum_{n=1}^N f_n(z) \; = \; f_1(z) + \cdots + f_N(z) \, .
    %\qquad \text{ for } N \in \bN \text{ and } z \in A \, .
  \end{align*}
  We say that the function series $\sum_{n=1}^\infty f_n$
  \term{converges pointwise}
  if the sequence $\big(F_N(z)\big)_{N \in \bN}$ of partial sums has
  a limit at every $z \in A$.
  We say that the function series $\sum_{n=1}^\infty f_n$
  \term{converges uniformly} on $A$
  if the sequence $(F_N)_{N \in \bN}$ of partial sum functions
  converges uniformly on~$A$.
  We say that the function series $\sum_{n=1}^\infty f_n$
  \term{converges uniformly on compacts}
  if the sequence $(F_N)_{N \in \bN}$ of partial sum functions
  converges uniformly on compacts.

  The limit function is then denoted by $\sum_{n=1}^\infty f_n$.
  %\begin{align*}
  %  \sum_{n=1}^\infty f_n = \lim_{N \to \infty} \; \sum_{n=1}^N F_n .
  %\end{align*}

  (Obvious modifications to the above are made
  if the terms' indexing starts from $n=0$ or some other index, and
  the notation is correspondingly changed to, e.g., $\sum_{n=0}^\infty$.)
\end{definition}

\begin{lemma}[Weierstrass M-test \Palka{Thm~VII.2.2}]
  \label{lem:weierstrass_test}
  \uses{def:function_series}
  %\lean{}
  % \leanok
  Suppose that $M_1,M_2,\ldots \ge 0$ are nonnegative
  numbers such that the series $\sum_{n=1}^\infty M_n$ converges.
  Suppose also that for each $n \in \bN$,
  $f_n \colon X \to \bC$ is a function on $X$
  such that $|f_n(x)| \le M_n$ for all $x \in X$.
  Then the series $\sum_{n=1}^\infty f_n$
  converges absolutely and uniformly on~$X$.
\end{lemma}
\begin{proof}
  \uses{lem:cauchy_uniform_limit_criterion}
  % \leanok
  \ldots
\end{proof}

\begin{lemma}[Series of analytic functions \Palka{Thm~VII.3.2}]
  \label{lem:uoc_analytic_series}
  \uses{def:uniform_limit, def:function_series, def:analytic_function}
  %\lean{}
  Suppose that functions $f_1, f_2, \ldots \colon U \to \bC$
  are analytic functions on an open set $U \subset \bC$
  such that the series
  $\sum_{n=1}^\infty f_n$ converges uniformly on compacts to a
  function $f \colon U \to \bC$. Then $f$ is analytic on $U$.
  Moreover, for any $k \in \bN$, the series $\sum_{n=1}^\infty f^{(k)}_n$
  of $k$th derivatives converges uniformly on compacts to $f^{(k)}$.
  % \leanok
\end{lemma}
\begin{proof}
  \uses{lem:uoc_limit_analytic}
  % \leanok
  \ldots
\end{proof}



\section{Power series}

\begin{definition}[Power series \Palka{Sec.~VII.3.3}]
  \label{def:power_series}
  \uses{def:function_series}
  %\lean{}
  % \leanok
  Let $z_0 \in \bC$ be a point in the complex plane
  and let $a_0,a_1,a_2\ldots \in \bC$ be coefficients.
  A function series of the form
  \begin{align*}
  \sum_{n=0}^\infty a_n \, (z-z_0)^n
    = a_0 + a_1 \, (z - z_0) + a_2 \, (z - z_0)^2 + \cdots
  \end{align*}
  is called a \term{power series} centered at $z_0$.
\end{definition}

\begin{lemma}[Abel's theorem]
  \label{lem:abel}
  \uses{def:power_series, def:absolute_convergence}
  % \lean{}
  % \leanok
  If a power series
  \begin{align*}
    \sum_{n=0}^\infty a_n \, (z-z_0)^n
  \end{align*}
  converges at $z = w \in \bC$, then it converges
  absolutely for all $z \in \bC$ such that $|z-z_0| < |w-z_0|$.
\end{lemma}
\begin{proof}
  \uses{lem:absolute_convergence_implies_convergence}
  % \leanok
  \ldots
\end{proof}

\begin{corollary}[Abel's theorem in the contrapositive]
  \label{cor:abel_contrapositive}
  \uses{def:power_series, def:absolute_convergence}
  % \lean{}
  % \leanok
  If a power series
  \begin{align*}
    \sum_{n=0}^\infty a_n \, (z-z_0)^n
  \end{align*}
  does not converge at $z = w \in \bC$, then it does not converge
  at any $z \in \bC$ such that $|z-z_0| > |w-z_0|$.
\end{corollary}
\begin{proof}
  \uses{lem:abel}
  % \leanok
  \ldots
\end{proof}

\begin{definition}[Radius of convergence]
  \label{def:radius_of_convergence}
  \uses{def:power_series, lem:abel, cor:abel_contrapositive}
  % \lean{}
  % \leanok
  The \term{radius of convergence} of a power series
  \begin{align*}
    \sum_{n=0}^\infty a_n \, (z-z_0)^n
  \end{align*}
  is defined as
  \begin{align*}
    R \; := \;
      \sup \bigg\{ |z-z_0| \; \bigg| \;
        \sum_{n=0}^\infty a_n \, (z-z_0)^n \, \text{ converges} \bigg\} .
  \end{align*}

  From Lemma~\ref{lem:abel} and Corollary~\ref{cor:abel_contrapositive}
  it follows that the power series $\sum_{n=0}^\infty a_n \, (z-z_0)^n$
  converges for all $z \in \bC$ such that $|z-z_0| < R$
  and diverges for all $z \in \bC$ such that $|z-z_0| > R$.
  The disk $\DiskRC{R}{z_0}$ is called the \term{disk of convergence}
  of the power series $\sum_{n=0}^\infty a_n \, (z-z_0)^n$.

  (If $R=+\infty$, we interpret $\DiskRC{R}{z_0} = \bC$.)
\end{definition}

\begin{lemma}[D'Alembert's ratio test for the radius of convergence]
  \label{lem:ratio_test_radius_of_convergence}
  \uses{def:radius_of_convergence}
  % \lean{}
  % \leanok
  Suppose that for the coefficients of a power series
  \begin{align*}
    \sum_{n=0}^\infty a_n \, (z-z_0)^n
  \end{align*}
  the limit
  \begin{align*}
    \rho = \lim_{n \to \infty} \frac{|a_n|}{|a_{n+1}|}
  \end{align*}
  exists. Then the radius of convergence $R$ of the power series
  is $R = \rho$.
\end{lemma}
\begin{proof}
  \uses{lem:ratio_test}
  % \leanok
  \ldots
\end{proof}

\begin{theorem}[Hadamard's formula for the radius of convergence \Palka{Thm~VII.3.3}]
  \label{thm:hadamard_formula}
  \uses{def:radius_of_convergence, def:limsup}
  %\lean{}
  % \leanok
  Let $z_0 \in \bC$ be a point in the complex plane
  and let $a_0,a_1,a_2\ldots \in \bC$ be coefficients.
  The radius of convergence of a power series
  \begin{align*}
    \sum_{n=0}^\infty a_n \, (z-z_0)^n
  \end{align*}
  is given by the formula
  \begin{align*}
    R = \frac{1}{\limsup_{n \to \infty} \sqrt[n]{|a_n|}} ,
  \end{align*}
  with the conventions $\frac{1}{+\infty}=0$ and $\frac{1}{0} = +\infty$.
\end{theorem}
\begin{proof}
  \uses{lem:weierstrass_test, lem:terms_tend_to_zero}
  % \leanok
  \ldots
\end{proof}

\begin{lemma}[Analyticity of power series \Palka{Thm~VII.3.3}]
  \label{lem:power_series_analytic}
  \uses{def:power_series, def:analytic_function, thm:hadamard_formula}
  %\lean{}
  % \leanok
  Let $z_0 \in \bC$ be a point in the complex plane
  and let $a_0,a_1,a_2\ldots \in \bC$ be coefficients.
  Suppose that the power series
  \begin{align*}
    f(z) = \sum_{n=0}^\infty a_n \, (z-z_0)^n
  \end{align*}
  has radius of convergence $R > 0$.
  Then it defines an analytic function~$f$ on the disk $\DiskRC{R}{z_0}$.
  The derivative of $f$ is given by the power series
  \begin{align*}
    f'(z) = \sum_{n=1}^\infty n \, a_n \, (z-z_0)^{n-1} .
  \end{align*}
  Moreover, the coefficients $a_k$ are related to the $k$th derivatives of $f$
  at $z_0$ through the formula
  \begin{align*}
    a_k = \frac{f^{(k)}(z_0)}{k!} .
  \end{align*}
\end{lemma}
\begin{proof}
  \uses{lem:uoc_analytic_series, lem:polynomial_analytic}
  % \leanok
  \ldots
\end{proof}

\begin{lemma}[Uniqueness of power series representation]
  \label{lem:power_series_uniqueness}
  \uses{def:power_series}
  % \lean{}
  % \leanok
  Suppose that two power series $\sum_{n=0}^\infty a_n \, (z-z_0)^n$
  and $\sum_{n=0}^\infty b_n \, (z-z_0)^n$
  converge in a disk $\DiskRC{r}{z_0}$ of radius $r > 0$
  and represent the same function
  \begin{align*}
    \sum_{n=0}^\infty a_n \, (z-z_0)^n
    \; = \; \sum_{n=0}^\infty b_n \, (z-z_0)^n
    \qquad \text{ for } z \in \DiskRC{r}{z_0} .
  \end{align*}
  Then their coefficients must be equal:
  $a_n = b_n$ for all~$n$.
\end{lemma}
\begin{proof}
  \uses{lem:power_series_analytic}
  % \leanok
  \ldots
\end{proof}



\section{Taylor series and local representation of analytic functions}

\begin{theorem}[Taylor series of analytic functions \Palka{Thm~VII.3.4}]
  \label{thm:analytic_function_taylor_series}
  \uses{def:power_series, def:analytic_function}
  %\lean{}
  % \leanok
  Suppose that $f \colon U \to \bC$ is an analytic function
  on an open set $U \subset \bC$ which contains
  a disk $\DiskRC{r}{z_0} \subset U$.
  Then the function $f$ can be represented in $\DiskRC{r}{z_0}$
  as a power series
  \begin{align*}
    f(z) = \sum_{n=0}^\infty \frac{f^{(n)}(z_0)}{n!} \, (z-z_0)^n .
  \end{align*}
  Moreover, this is the unique power series centered at $z_0$
  that representats $f$ in a neighborhood of $z_0$.
\end{theorem}
\begin{proof}
  \uses{thm:cauchy_formula_derivative, lem:uniform_limit_integral, lem:power_series_analytic}
  % \leanok
  \ldots
\end{proof}

\begin{theorem}[Equivalent characterizations of analyticity]
  \label{thm:TFAE_ANALYTIC}
  \uses{def:analytic_function, def:power_series, def:primitive}
  % \lean{}
  % \leanok
  Let $f \colon U \to \bC$ be a continuous function on an open
  set $U \subset \bC$.
  Then the following are equivalent:
  \begin{itemize}
    \item $f$ is analytic on $U$;
    \item for any $z \in U$ there exists a neighborhood of $z$
      in which $f$ has a primitive;
    \item for any $z \in U$ there exists a neighborhood of $z$
      in which $f$ can be represented as a convergent power series.
  \end{itemize}
\end{theorem}
\begin{proof}
  \uses{thm:analytic_function_taylor_series, lem:power_series_analytic, thm:morera}
  % \leanok
  \ldots
\end{proof}

\begin{lemma}[No vanishing of all derivatives at a point \Palka{Thm~VIII.1.1}]
  \label{lem:no_vanishing_of_all_order_derivatives}
  \uses{def:analytic_function, lem:higher_derivatives_analytic}
  %\lean{}
  % \leanok
  Suppose that $f \colon \Domain \to \bC$ is an analytic function
  on a connected open set $\Domain \subset \bC$.
  If there exists a point $z_0 \in \Domain$
  such that $f^{(n)}(z_0) = 0$ for all $n \in \bN$, then $f$ is a constant
  function.
\end{lemma}
\begin{proof}
  \uses{thm:analytic_function_taylor_series}
  % \leanok
  \ldots
\end{proof}

\begin{theorem}[Factor theorem for analytic functions \Palka{Thm~VIII.1.2}]
  \label{thm:factor_theorem_for_analytic_functions}
  \uses{def:analytic_function}
  %\lean{}
  % \leanok
  Suppose that $f \colon \Domain \to \bC$ is a non-constant analytic function
  on a connected open set $\Domain \subset \bC$, and $z_0 \in \Domain$
  is a point where $f(z_0) = 0$.
  Then $f$ can be uniquely represented as
  \begin{align*}
    f(z) = (z - z_0)^m \, g(z) \qquad \text{ for } z \in \Domain ,
  \end{align*}
  where $m \in \bN$ and $g \colon \Domain \to \bC$ is an analytic
  function such that $g(z_0) \ne 0$.
\end{theorem}
\begin{proof}
  \uses{thm:analytic_function_taylor_series, lem:no_vanishing_of_all_order_derivatives}
  % \leanok
  \ldots
\end{proof}

\begin{corollary}[Local representation of analytic functions \Palka{Cor~VIII.1.3}]
  \label{cor:analytic_local_representation}
  \uses{def:analytic_function}
  %\lean{}
  % \leanok
  Suppose that $f \colon \Domain \to \bC$ is a non-constant analytic function
  on a connected open set $\Domain \subset \bC$.
  Then for any $z_0 \in \Domain$, we can write $f$ uniquely in the form
  \begin{align*}
    f(z) = f(z_0) + (z - z_0)^m \, g(z) \qquad \text{ for } z \in \Domain ,
  \end{align*}
  where $m \in \bN$ and $g \colon \Domain \to \bC$ is an analytic
  function such that $g(z_0) \ne 0$.
\end{corollary}
\begin{proof}
  \uses{thm:factor_theorem_for_analytic_functions}
  % \leanok
  Apply Theorem~\ref{thm:factor_theorem_for_analytic_functions} to the
  function $z \mapsto f(z) - f(z_0)$.
\end{proof}

\begin{theorem}[L'Hospital's rule for analytic functions \Palka{Thm~VIII.1.4}]
  \label{thm:l_hospital_rule}
  \uses{def:analytic_function}
  %\lean{}
  % \leanok
  Let $f$ and $g$ be functions that are analytic in a neighborhood of $z_0$
  such that $f(z_0) = 0$ and $g(z_0) = 0$. Then we have
  \begin{align*}
    \lim_{z \to z_0} \frac{f(z)}{g(z)} = \lim_{z \to z_0} \frac{f'(z)}{g'(z)} ,
  \end{align*}
  understood in the sense that either both limits exist and are equal
  to each other, or else neither limit exists.
\end{theorem}
\begin{proof}
  \uses{cor:analytic_local_representation}
  % \leanok
  \ldots
\end{proof}

\begin{theorem}[Discrete mapping theorem \Palka{Thm~VIII.1.5}]
  \label{thm:discrete_mapping_theorem}
  \uses{def:analytic_function}
  %\lean{}
  % \leanok
  Suppose that $f \colon \Domain \to \bC$ is a non-constant
  analytic function on a connected open set $\Domain \subset \bC$.
  Then the set of zeros of $f$ is discrete, i.e., for
  every $z_0 \in \Domain$ such that $f(z_0)=0$,
  there exists a $r>0$ such that $f(z) \ne 0$
  for all $z \in \BallRC{r}{z_0} \setminus \set{z_0}$.
\end{theorem}
\begin{proof}
  \uses{thm:factor_theorem_for_analytic_functions}
  % \leanok
  \ldots
\end{proof}

\begin{corollary}[Principle of analytic continuation \Palka{Cor~VIII.1.6}]
  \label{cor:ANALYTIC_CONTINUATION}
  \uses{def:analytic_function}
  %\lean{}
  % \leanok
  Let $f, g \colon \Domain \to \bC$ be two analytic functions on a connected
  open set $\Domain \subset \bC$.
  If $f(z) = g(z)$ for all $z$ in
  some subset of $\Domain$ which has an accumulation point in $\Domain$,
  then we have $f(z) = g(z)$ for all $z \in \Domain$.
\end{corollary}
\begin{proof}
  \uses{thm:discrete_mapping_theorem}
  % \leanok
  \ldots
\end{proof}



\section{Laurent series}

\begin{definition}[Doubly infinite series \Palka{Sec.~VII.2.1}]
  \label{def:doubly_infinite_series}
  \uses{def:complex_series}
  %\lean{}
  % \leanok
  A doubly infinite series of complex numbers is a series of the form
  \begin{align*}
    \sum_{n=-\infty}^\infty z_n
      = \cdots + z_{-2} + z_{-1} + z_0 + z_1 + z_2 + \cdots ,
  \end{align*}
  where $\ldots, z_{-2}, z_{-1}, z_0, z_1, z_2, \ldots \in \bC$.
  We say that such a series \term{converges} to $s \in \bC$ if
  for all $\eps > 0$ there exists an $N \in \bN$
  such that for all $m_+ \ge N$ and $m_- \le -N$ we have
  \begin{align*}
    \Big| \sum_{n=m_-}^{m_+} z_n - s \Big| < \eps .
  \end{align*}
\end{definition}

\begin{lemma}[Convergence of doubly infinite series \Palka{Lem~VII.2.1}]
  \label{lem:convergence_doubly_infinite}
  \uses{def:doubly_infinite_series}
  %\lean{}
  % \leanok
  A doubly infinite series
  \begin{align*}
    \sum_{n=-\infty}^\infty z_n
      = \cdots + z_{-2} + z_{-1} + z_0 + z_1 + z_2 + \cdots ,
  \end{align*}
  of complex numbers converges if and only if both the series
  $\sum_{n=0}^\infty z_n$ and $\sum_{n=1}^\infty z_{-n}$ converge.
\end{lemma}
\begin{proof}
  \uses{def:limit}
  % \leanok
  \ldots
\end{proof}

\begin{definition}[Laurent series \Palka{Sec.~VII.3.4}]
  \label{def:laurent_series}
  \uses{def:doubly_infinite_series}
  %\lean{}
  % \leanok
  A \term{Laurent series} centered at $z_0 \in \bC$ is a doubly infinite series
  of functions of the form
  \begin{align*}
    z \mapsto & \; \sum_{n=-\infty}^\infty a_n (z - z_0)^n \\
         = & \; \cdots + \frac{a_{-2}}{(z-z_0)^2} + \frac{a_{-1}}{z-z_0}
                + a_0 + a_1 (z-z_0) + a_2 (z-z_0)^2 + \cdots .
  \end{align*}
\end{definition}

\begin{lemma}[Annulus of convergence of Laurent power series \Palka{Thm~VII.3.5}]
  \label{lem:laurent_series_analytic}
  \uses{def:laurent_series}
  %\lean{}
  % \leanok
  Consider a Laurent series
  \begin{align*}
    f(z) = & \; \sum_{n=-\infty}^\infty a_n (z - z_0)^n .
  \end{align*}
  Denote
  \begin{align*}
    \rho_- = \limsup_{n \to \infty} \sqrt[n]{|a_{-n}|}, \qquad
    \rho_+ = \Big( \limsup_{n \to \infty} \sqrt[n]{|a_{n}|} \Big)^{-1} .
  \end{align*}
  Then the series $\sum_{n=-\infty}^\infty a_n (z - z_0)^n$
  converges for all $z$ is the annulus
  \begin{align*}
    \AnnRRC{\rho_-}{\rho_+}{z_0}
      := \Big\{ z \in \bC \; \Big| \; \rho_- < |z-z_0| < \rho_+ \Big\} .
  \end{align*}
  Moreover, the convergence is uniform on compact subsets
  of $\AnnRRC{\rho_-}{\rho_+}{z_0}$, and the series defines an analytic
  function $f(z)$ on the annulus $\AnnRRC{\rho_-}{\rho_+}{z_0}$.
\end{lemma}
\begin{proof}
  \uses{lem:power_series_analytic}
  % \leanok
  \ldots
\end{proof}

\begin{theorem}[Laurent series for analytic functions \Palka{Thm~VII.3.6}]
  \label{thm:ANALYTIC_FUNCTION_LAURENT_SERIES}
  \uses{def:laurent_series, def:analytic_function}
  %\lean{}
  % \leanok
  Suppose that $f \colon U \to \bC$ is an analytic function
  on an open set $U \subset \bC$ which contains
  an annulus
  \begin{align*}
    \AnnRRC{r_1}{r_2}{z_0} = \Big\{z \in \bC \; \Big| \; r_1 < |z - z_0| < r_2 \Big\}
  \end{align*}
  for some $z_0 \in \bC$ and $0 \le r_1 < r_2$.
  %$\DiskRC{r}{z_0} \subset U$.
  Then the function $f$ can be uniquely
  represented in $\AnnRRC{r_1}{r_2}{z_0}$ as a series
  \begin{align*}
    f(z) = \sum_{n=-\infty}^\infty a_n \, (z-z_0)^n ,
  \end{align*}
  where the coefficients $a_n$, for $n \in \bZ$, are given by
  \begin{align*}
    a_n = \frac{1}{2\pi \ii}
      \oint_{\bdry \BallRC{r}{z_0}} \frac{f(z)}{(z - z_0)^{n+1}} \, \ud z
    \qquad \text{ for any } r \in (r_1, r_2) .
  \end{align*}
\end{theorem}
\begin{proof}
  \uses{thm:cauchy_formula_derivative, lem:uniform_limit_integral,
        lem:laurent_series_analytic, thm:analytic_function_taylor_series}
  % \leanok
  \ldots
\end{proof}
