\section{Metrics and related concepts}

\begin{definition}[Metric]
  \label{def:metric_space}
  %\uses{}
  %\lean{}
  % \leanok
  A \term{metric} on a set $X$ is a function
  $\genmet \colon X \times X \to [0,\infty)$ such that
  for all $p_1, p_2, p_3 \in X$ we have
  \begin{align*}
    \genmet(p_1, p_3) \le \; & \genmet(p_1,p_2) + \genmet(p_2,p_3)
            & \text{(triangle inequality)} \\
    \genmet(p_1, p_2) = \; & \genmet(p_2,p_1)
            & \text{(symmetricity)} \\
    \genmet(p_1, p_2) = \; & 0 \text{ if and only if } p_1 = p_2 .
            & \text{(separation of points)}
  \end{align*}
  The set $X$ equipped with the metric $\genmet$ on
  it is called a \term{metric space}.
\end{definition}

\begin{lemma}[Metric in the complex plane]
  \label{lem:metric_on_C}
  \uses{def:metric_space, def:absolute_value}
  % \lean{}
  % \leanok
  The formula
  \begin{align*}
    \genmet(z,w) = |z - w|
    \qquad \text{ for } z,w \in \bC
  \end{align*}
  defines a metric on the complex plane $\bC$.

  (Thus $\bC$ becomes a metric space. Also any subset of $\bC$,
  in particular $\bR \subset \bC$, becomes a metric space when equipped
  with the metric given by the above formula restricted to the subset.)
\end{lemma}
\begin{proof}
  \uses{lem:absolute_value_properties}
  % \leanok
  \ldots
\end{proof}

\begin{definition}[Ball (disk)]
  \label{def:ball}
  \uses{def:metric_space}
  %\lean{}
  % \leanok
  Let $X$ be a metric space with metric $\genmet \colon X \times X \to [0,\infty)$.
  Let $p_0 \in X$ be a point and let $r>0$.

  The set
  \begin{align*}
    \BallRC{r}{p_0} = \Big\{ p \in X \; \Big| \; \genmet(p,p_0) < r \Big\}
  \end{align*}
  is called an \term{open ball} in $X$, centered at $p_0$, and with radius $r$.

  The set
  \begin{align*}
    \CBallRC{r}{p_0} = \Big\{ p \in X \; \Big| \; \genmet(p,p_0) \le r \Big\}
  \end{align*}
  is called a \term{closed ball} in $X$, centered at $p_0$, and with radius $r$.

  %The set
  %\begin{align*}
  %  \SphereRC{r}{x_0} = \set{x \in X \; \Big| \; \genmet(x,x_0) = r}
  %\end{align*}
  %is called a \term{sphere} in $X$, centered at $x_0$, and with radius $r$.

  (In the case of the complex plane $\bC$, the term \term{disk}
  is often used instead of the general metric space theory term \term{ball}.)
  %and the term \term{circle}
  %may be used instead of the general term sphere.)
\end{definition}

\begin{definition}[Interior point]
  \label{def:interior}
  \uses{def:ball}
  % \lean{}
  % \leanok
  Let $X$ be a metric space, and $A \subset X$ a subset.
  A point $p \in A$ is said to be an \term{interior point} of $A$
  if for some $r > 0$ we have $\BallRC{r}{p} \subset A$.
  %The set of all interior points of $A$ is denoted \ldots and called the \term{interior} of $A$,
\end{definition}

\begin{definition}[Exterior point]
  \label{def:exterior}
  \uses{def:ball, def:interior}
  % \lean{}
  % \leanok
  Let $X$ be a metric space, and $A \subset X$ a subset.
  A point $p \in X \setminus A$ is said to be an \term{exterior point} of $A$
  if for some $r > 0$ we have $\BallRC{r}{p} \subset X \setminus A$.

  (It is easy to see that the exterior points of $A$ are exactly the interior
  points)
\end{definition}

\begin{definition}[Boundary]
  \label{def:boundary}
  \uses{def:ball, def:interior, def:exterior}
  % \lean{}
  % \leanok
  Let $X$ be a metric space, and $A \subset X$ a subset.
  A point $p \in X$ is said to be a \term{boundary point} of $A$
  if for all $r > 0$ we have that $\BallRC{r}{p}$ contains points
  of $A$ and $X \setminus A$ (i.e. $\BallRC{r}{p} \cap A \ne \emptyset$
  and $\BallRC{r}{p} \setminus A \ne \emptyset$).

  The set of all boundary points of $A$ is denoted $\bdry A$ and called the
  \term{boundary} of $A$.

  (It is easy to see that the boundary $\bdry A \subset X$ is exactly
  the set of points of $X$ which are neither interior nor exterior points
  of $A$.)
\end{definition}

\begin{definition}[Open set]
  \label{def:open_set}
  \uses{def:metric_space, def:ball, def:interior}
  % \lean{}
  % \leanok
  Let $X$ be a metric space.
  % with metric $\genmet \colon X \times X \to [0,\infty)$.
  A subset $U \subset X$ is said to be an \term{open set} if
  %for every $x \in U$ there exists an $r > 0$ such that
  %$\BallRC{r}{x} \subset U$.
  each point $p \in U$ is an interior point of $U$.
\end{definition}

\begin{definition}[Closed set]
  \label{def:closed_set}
  \uses{def:open_set, def:exterior}
  % \lean{}
  % \leanok
  Let $X$ be a metric space.
  % with metric $\genmet \colon X \times X \to [0,\infty)$.
  A subset $F \subset X$ is said to be a \term{closed set} if
  the complement $X \setminus F \subset X$ is an open set.

  (Equivalently, each point $p \in X \setminus F$ in the complement
  of $F$ is an exterior point of $F$.)
\end{definition}

\begin{definition}[Boundedness]
  \label{def:bounded}
  \uses{def:metric_space}
  % \lean{}
  % \leanok
  Let $X$ be a metric space.
  with metric $\genmet \colon X \times X \to [0,\infty)$.

  A subset $A \subset X$ is \term{bounded} if there exists
  a number $M>0$ such that $\genmet(p,q) \le M$ for all $p,q \in A$.
  (If $X$ is nonempty, an equivalent definition would be that
  $A$ is bounded if it is a subset of some ball in $X$.)

  A function $f \colon Z \to X$ with values in a metric space~$X$
  is \term{bounded} if the set $f[Z] \subset X$ of its values is a
  bounded subset of~$X$.

  (In the case $X = \bC$ we have the following further characterizations:
  A subset $A \subset \bC$ is bounded if and only if there exists an $R>0$
  such that $|z| \le R$ for all $z \in A$. A function $f \colon Z \to \bC$
  is bounded if and only if there exists an $R>0$ such that $|f(z)| \le R$
  for all $z \in Z$.)
\end{definition}



\section{Limits}

\begin{definition}[Limit]
  \label{def:limit}
  \uses{def:metric_space}
  %\lean{}
  % \leanok
  Let $X$ be a metric space and
  %with metric $\genmet \colon X \times X \to [0,\infty)$.
  let $(x_n)_{n \in \bN}$ be a sequence of points in $X$.
  %We say that $x \in X$ is the \term{limit} of the sequence $(x_n)_{n \in \bN}$
  %%(or that the sequence \term{converges} to the limit $x$)
  We say that the sequence $(x_n)_{n \in \bN}$ \term{converges}
  to a \term{limit} $x \in X$
  if for all $\eps > 0$ there exists an $N \in \bN$ such that
  for all $n \ge N$ we have $x_n \in \BallRC{\eps}{x}$
  (i.e., $\genmet(x_n,x) < \eps$).
  We then denote
  \begin{align*}
    \lim_{n \to \infty} x_n = x .
  \end{align*}
  (It is straightforward to check that the limit is
  unique if it exists.)

  Let then $X$ and $Y$ be metric spaces, with respective metrics
  $\genmet_X$ and $\genmet_Y$, and
  let $f \colon X \to Y$ be a function.
  We say that the function $f$ has a \term{limit} $y \in Y$
  \term{at a point} $p_0 \in X$
  if for all $\eps > 0$ there exists a $\delta > 0$ such that
  for all $p \in \BallRC{\delta}{p_0} \setminus \set{p_0}$
  %(i.e., when $0 < \genmet_X (p,p_0) < \delta$)
  we have $f(p) \in \BallRC{\eps}{y}$.
  %(i.e., $\genmet_Y \big( f(p),y \big) < \eps$).
  We then denote
  \begin{align*}
    \lim_{p \to p_0} f(p) = y .
  \end{align*}
  (It is straightforward to check that the limit is
  unique if it exists.)

  (Equivalently, written in terms of distances,
  $\lim_{p \to p_0} f(p) = y$ means that
  for every $\eps > 0$ there exists a $\delta > 0$ such that
  we have $\genmet_Y \big( f(p), y \big) < \eps$ whenever
  $0 < \genmet_X (p,p_0) < \delta$.)
\end{definition}

\begin{lemma}[Limits in the complex plane]
  \label{lem:complex_limit}
  \uses{def:limit, lem:metric_on_C}
  % \lean{}
  % \leanok
  For a sequence $(z_n)_{n \in \bN}$ of complex numbers
  %$z_n = x_n + \ii y_n$ (with $x_n,y_n \in \bR$) converges
  %to a limit $z = x + \ii y$ if and only if the real and imaginary
  %part sequences converge to the real and imaginary parts of the limit,
  %i.e., $\lim_{n \to \infty} x_n = x$ and $\lim_{n \to \infty} y_n = y$.
  we have
  \begin{align*}
    \lim_{n \to \infty} z_n = z
  \end{align*}
  if and only if
  \begin{align*}
    \lim_{n \to \infty} \re(z_n) = \re(z)
    \quad \text{and} \quad
    \lim_{n \to \infty} \im(z_n) = \im(z) .
  \end{align*}

  Let $X$ be a metric space,
  let $f \colon X \to \bC$ a complex-valued
  function on $X$, and
  %Write $u, v \colon X \to \bR$
  %for the real and imaginary parts of the function, so that
  %$f(p) = u(p) + \ii \, v(p)$ for $p \in X$.
  let $p_0 \in X$ be a point. Then we have
  \begin{align*}
    \lim_{p \to p_0} f(p) = z
  \end{align*}
  if and only if
  \begin{align*}
    \lim_{p \to p_0} \re \big( f(p) \big) = \re(z)
    \quad \text{and} \quad
    \lim_{p \to p_0} \im \big( f(p) \big) = \im(z) .
  \end{align*}
\end{lemma}
\begin{proof}
  %\uses{lem:absolute_value_properties}
  % \leanok
  \ldots
\end{proof}

\begin{lemma}[Operations with complex limits]
  \label{lem:complex_limit_properties}
  \uses{lem:complex_limit, thm:complex_number_field}
  % \lean{}
  % \leanok
  Let $(z_n)_{n \in \bN}$ and $(w_n)_{n \in \bN}$
  be complex number sequences converging to limits
  \begin{align*}
    \lim_{n \to \infty} z_n = z
    \quad \text{and} \quad
    \lim_{n \to \infty} w_n = w .
  \end{align*}
  Then we have
  \begin{align*}
    \lim_{n \to \infty} (z_n + w_n) \, = \, z + w ,
    \quad
    \lim_{n \to \infty} (z_n w_n) \, = \, z w ,
    \quad
    \lim_{n \to \infty} \frac{z_n}{w_n} \, = \, \frac{z}{w}
    \; \text{ if $w \ne 0$}.
  \end{align*}
    %If moreover $w \ne 0$ then we also have
    %\begin{align*}
    %  \lim_{n \to \infty} \frac{z_n}{w_n} \; = \; \frac{z}{w} .
    %\end{align*}
  %In particular, taking one of the sequences to be a constant sequence $c \in \bC$,
  %we get $\lim_{n \to \infty} (z_n + c) = z + c$ and
  %$\lim_{n \to \infty} (c \, z_n) = c z$.
  %%In particular, taking $(w_n)_{n \in \bN}$ to be a constant sequence
  %%$w_n = c \in \bC$, we have
  %%\begin{align*}
  %%  \lim_{n \to \infty} (z_n + c) = z + c
  %%  \quad \text{and} \quad
  %%  \lim_{n \to \infty} (c \, z_n) = c z .
  %%\end{align*}

  Let $X$ be a metric space, let $p_0 \in X$
  be a point, and let $f,g \colon X \to \bC$
  be two complex-valued functions on $X$ such that
  \begin{align*}
    \lim_{p \to p_0} f(p) = z
    \quad \text{and} \quad
    \lim_{p \to p_0} g(p) = w .
  \end{align*}
  Then we have
  \begin{align*}
    \lim_{p \to p_0} \big( f(p) + g(p) \big) \, = \, z + w ,
    \quad
    \lim_{p \to p_0} \big( f(p) \, g(p) \big) \, = \, z w ,
    \quad
    \lim_{p \to p_0} \frac{f(p)}{g(p)} \, = \, \frac{z}{w}
    \; \text{ if $w \ne 0$}.
  \end{align*}
    %If moreover $w \ne 0$ then we also have
    %\begin{align*}
    %  \lim_{p \to p_0} \frac{f(p)}{g(p)} \; = \; \frac{z}{w}
    %\end{align*}
  %In particular taking one of the functions to be a constant function
  %$c \in \bC$, we get $\lim_{p \to p_0} \big( f(p) + c \big) = z + c$
  %and $\lim_{p \to p_0} \big( c \, f(p) \big) = c z$.
  %%In particular, taking $g$ to be a constant function
  %%$g(p) = c \in \bC$, we have
  %%\begin{align*}
  %%  \lim_{p \to p_0} \big( f(p) + c \big) = z + c
  %%  \quad \text{and} \quad
  %%  \lim_{p \to p_0} \big( c \, f(p) \big) = c z .
  %%\end{align*}
\end{lemma}
\begin{proof}
  The arguments are similar to the proofs given in
  \textbf{MS-C1541 Metric Spaces} for the real-valued cases.
\end{proof}

\begin{definition}[Cauchy sequence]
  \label{def:cauchy_sequence}
  \uses{def:metric_space}
  %\lean{}
  % \leanok
  %Let $X$ be a metric space with metric $\genmet \colon X \times X \to [0,\infty)$.
  \ldots
\end{definition}

\begin{lemma}[Every real Cauchy sequence converges]
  \label{lem:real_complete}
  \uses{def:cauchy_sequence, def:limit}
  %\lean{}
  % \leanok
  If a real number sequence $(x_n)_{n \in \bN}$ is Cauchy,
  then it converges to a limit $\lim_{n \to \infty} x_n \in \bR$.

  (This property is known as \term{completeness} of the metric space $\bR$.)
\end{lemma}
\begin{proof}
  %\uses{def:cauchy_sequence}
  % \leanok
  See \textbf{MS-C1541 Metric Spaces}.
\end{proof}

\begin{lemma}[Every complex Cauchy sequence converges]
  \label{lem:complex_complete}
  \uses{def:cauchy_sequence, def:limit, lem:metric_on_C}
  %\lean{}
  % \leanok
  If a complex number sequence $(z_n)_{n \in \bN}$ is Cauchy,
  then it converges to a limit $\lim_{n \to \infty} z_n \in \bC$.

  (This property is known as \term{completeness} of the metric space $\bC$.)
\end{lemma}
\begin{proof}
  \uses{lem:real_complete, lem:complex_limit}
  % \leanok
  See \textbf{MS-C1541 Metric Spaces}.

  (Idea: This follows from Lemma~\ref{lem:real_complete} by
  considering real and imaginary parts separately and picking
  a subsequence of a subsequence.)
\end{proof}



\section{Continuity}

%\begin{definition}[Limit of a function at a point]
%  \label{def:function_limit}
%  \uses{def:metric_space}
%  %\lean{}
%  % \leanok
%  Let $X$ and $Y$ be metric spaces.
%  Let $f \colon X \to Y$ be
%  a function, and let $x_0 \in X$ be a point.
%  We say that the function $f$ has a \term{limit} $y \in Y$
%  \term{at the point} $x_0 \in X$
%  if for all $\eps > 0$ there exists a $\delta > 0$ such that
%  for all $x \in \BallRC{\delta}{x_0} \setminus \set{x_0}$
%  %(i.e., when $0 < \genmet_X (x,x_0) < \delta$)
%  we have $f(x) \in \BallRC{\eps}{y}$.
%  %(i.e., $\genmet_Y \big( f(x),y \big) < \eps$).
%  We then denote
%  \begin{align*}
%    \lim_{x \to x_0} f(x) = y .
%  \end{align*}
%  (It is straightforward to check that the limit is
%  unique if it exists.)
%
%  (Equivalently, written in terms of distances,
%  $\lim_{x \to x_0} f(x) = y$ means that
%  for every $\eps > 0$ there exists a $\delta > 0$ such that
%  we have $\genmet_Y \big( f(x),y \big) < \eps$ whenever
%  $0 < \genmet_X (x,x_0) < \delta$.)
%\end{definition}
%
%\begin{lemma}[Limits of complex valued functions]
%  \label{lem:complex_function_limit}
%  \uses{def:function_limit, lem:metric_on_C}
%  % \lean{}
%  % \leanok
%  Let $X$ be a metric space.
%  Let $f \colon X \to \bC$ a complex-valued
%  function on $X$ and
%  %Write $u, v \colon X \to \bR$
%  %for the real and imaginary parts of the function, so that
%  %$f(p) = u(p) + \ii \, v(p)$ for $p \in X$.
%  let $p_0 \in X$ be a point. Then we have
%  \begin{align*}
%    \lim_{p \to p_0} f(p) = z
%  \end{align*}
%  if and only if
%  \begin{align*}
%    \lim_{p \to p_0} \re \big( f(p) \big) = \re(z)
%    \quad \text{and} \quad
%    \lim_{p \to p_0} \im \big( f(p) \big) = \im(z) .
%  \end{align*}
%%\end{lemma}
%
%\begin{lemma}[Operations with limits of complex-valued functions]
%  \label{lem:complex_function_limit_properties}
%  \uses{lem:complex_function_limit}
%  % \lean{}
%  % \leanok
%  Let $X$ be a metric space, and let $f,g \colon X \to \bC$
%  be two complex-valued functions on $X$. Let $p_0 \in X$
%  be a point, and assume that
%  \begin{align*}
%    \lim_{p \to p_0} f(p) = z
%    \quad \text{and} \quad
%    \lim_{p \to p_0} g(p) = w .
%  \end{align*}
%  Then we have
%  \begin{align*}
%    \lim_{p \to p_0} \big( f(p) + g(p) \big) = z + w
%    \quad \text{and} \quad
%    \lim_{p \to p_0} \big( f(p) \, g(p) \big) = z w .
%  \end{align*}
%
%  In particular, taking $g$ to be a constant function
%  $g(p) = c \in \bC$, we have
%  \begin{align*}
%    \lim_{p \to p_0} \big( f(p) + c \big) = z + c
%    \quad \text{and} \quad
%    \lim_{p \to p_0} \big( c \, f(p) \big) = c z .
%  \end{align*}
%\end{lemma}

\begin{definition}[Continuity]
  \label{def:continuous}
  \uses{def:metric_space, def:limit}
  %\lean{}
  % \leanok
  Let $X$ and $Y$ be metric spaces.
  A function $f \colon X \to Y$ is said to be \term{continuous at a point}
  $p_0 \in X$ if $\lim_{p \to p_0} f(p) = f(p_0)$.

  (Equivalently, for every $\eps > 0$ there
  exists a $\delta > 0$ such that for any $p \in \BallRC{\delta}{p_0}$
  we have $f(p) \in \BallRC{\eps}{f(p_0)}$.)

  A function $f \colon X \to Y$ is said to be \term{continuous} if
  it is continuous at every point $p_0 \in X$.
\end{definition}

\begin{lemma}[Continuity of complex-valued functions]
  \label{lem:continuous_complex}
  \uses{def:continuous}
  % \lean{}
  % \leanok
  Let $X$ be a metric space, and let $f \colon X \to \bC$ be
  a complex-valued function on $X$. Then $f$ is continuous
  at $p_0 \in X$ if and only if its real and imaginary parts
  $p \mapsto \re \big( f(p) \big)$
  and $p \mapsto \im \big( f(p) \big)$ are continuous at $p_0$.
\end{lemma}
\begin{proof}
  \uses{lem:complex_limit}
  % \leanok
  \ldots
\end{proof}

\begin{corollary}[Continuity of coordinate projections]
  \label{cor:coordinate_continuous}
  \uses{def:continuous}
  % \lean{}
  % \leanok
  The coordinate projections
  \begin{align*}
    \re \colon \; & \bC \to \bR
    \qquad\qquad \text{ and } & \quad
    \im \colon \; & \bC \to \bR \\
    & z \mapsto \re(z) &
    & z \mapsto \im(z)
  \end{align*}
  are continuous functions.
\end{corollary}
\begin{proof}
  \uses{lem:continuous_complex}
  % \leanok


\end{proof}

\begin{lemma}[Operations with continuous complex-valued functions]
  \label{lem:complex_continuity_properties}
  \uses{def:continuous, thm:complex_number_field}
  % \lean{}
  % \leanok
  Let $X$ be a metric space, let $p_0 \in X$
  be a point, and let $f,g \colon X \to \bC$
  be two complex-valued functions on $X$
  which are continuous at $p_0$.
  Then also the functions
  \begin{align*}
    p \mapsto f(p) + g(p)
    \quad \text{and} \quad
    p \mapsto f(p) \, g(p)
  \end{align*}
  are continuous at $p_0$.

  If moreover $g(p_0) \ne 0$, then
  also the function
  $p \mapsto \frac{f(p)}{g(p)}$
  is continuous at $p_0$.
\end{lemma}
\begin{proof}
  \uses{lem:continuous_complex, lem:complex_limit_properties}
  \ldots
\end{proof}

\begin{lemma}[Continuity characterization]
  \label{lem:continuity_characterization}
  \uses{def:continuous, def:open_set, def:closed_set}
  % \lean{}
  % \leanok
  Let $X$ and $Y$ be metric spaces, and let $f \colon X \to Y$ be a function.
  Then the following are equivalent:
  \begin{itemize}
    \item $f$ is a continuous function;
    \item for every open set $V \subset Y$, the preimage
      $f^{-1}[V] = \big\{ x \in X \, \big| \, f(x) \in V \big\}$
      is an open set in $X$;
    \item for every closed set $A \subset Y$, the preimage
      $f^{-1}[A] = \big\{ x \in X \, \big| \, f(x) \in A \big\}$
      is a closed set in $X$.
  \end{itemize}
\end{lemma}
\begin{proof}
  % \uses{}
  % \leanok
  See \textbf{MS-C1541 Metric Spaces}.
\end{proof}

\begin{lemma}[Composition of continuous functions]
  \label{lem:composition_continuous}
  \uses{def:continuous}
  % \lean{}
  % \leanok
  Let $X$, $Y$, and $Z$ be metric spaces, and
  let $f \colon X \to Y$ and $g \colon Y \to Z$ be functions.
  If $f$ is continuous at $x_0 \in X$ and $g$ is continuous at $f(x_0) \in Y$,
  then the composition $g \circ f \colon X \to Z$ is continuous at $x_0$.

  (The composition $g \circ f$ is defined by
  the formula $(g \circ f)(x) = g \big( f(x) \big)$.)
\end{lemma}
\begin{proof}
  \uses{lem:continuity_characterization}
  % \leanok
  \ldots
\end{proof}

\begin{corollary}[Real multivariate polynomials are continuous]
  \label{cor:multivariate_polynomial_continuous}
  \uses{def:continuous}
  % \lean{}
  % \leanok
  Let $N \in \bN$ be a natural number, and let
  $c_{n,m} \in \bR$ be real numbers for $n,m \in \{0,1,\ldots,N\}$.
  Then the function $p \colon \bC \to \bR$ defined by
  \begin{align*}
    p(x + \ii y) = \sum_{m=0}^N \sum_{n=0}^N c_{m,n} \, x^m \, y^n
  \end{align*}
  is continuous.
\end{corollary}
\begin{proof}
  \uses{lem:composition_continuous, lem:complex_continuity_properties, cor:coordinate_continuous}
  % \leanok
  See \textbf{MS-C1541 Metric Spaces}.
\end{proof}

\begin{definition}[Uniform continuity]
  \label{def:uniformly_continuous}
  \uses{def:metric_space}
  %\lean{}
  % \leanok
  Let $X$ and $Y$ be metric spaces.
  A function $f \colon X \to Y$ is \term{uniformly continuous} if
  for every $\eps > 0$ there exists a $\delta > 0$ such that for
  any $p_0 \in X$ and $p \in \BallRC{\delta}{p_0}$
  we have $f(p) \in \BallRC{\eps}{f(p_0)}$.
\end{definition}

\begin{lemma}[Uniform continuity implies continuity]
  \label{lem:uniformly_continuous_implies_continuous}
  \uses{def:continuous, def:uniformly_continuous}
  %\lean{}
  % \leanok
  If a function $f \colon X \to Y$ is uniformly continuous, then it is continuous.
\end{lemma}
\begin{proof}
  See \textbf{MS-C1541 Metric Spaces}.

  (The easy proof is also a good exercise.)
\end{proof}


\section{Connectedness and path-connectedness}

\begin{definition}[Connectedness]
  \label{def:connected}
  \uses{def:metric_space}
  %\lean{}
  % \leanok
  %\ldots
  A set $A \subset X$ in a metric space $X$
  is \term{disconnected} if there exists
  a continuous surjective function
  $f \colon A \to \set{0,1}$ onto the two-element discrete
  set $\set{0,1}$.
  Otherwise $A$ is \term{connected}
  (then every continuous function $A \to \set{0,1}$ must be
  either constant $0$ or constant $1$).

  (The usual definition in topology textbooks reads
  slightly differently, but it is
  equivalent to the one we chose here by
  Lemma~\ref{lem:continuity_characterization}.)
\end{definition}

\begin{definition}[Path-connectedness]
  \label{def:path_connected}
  \uses{def:path}
  %\lean{}
  % \leanok
  \ldots
\end{definition}

\begin{lemma}[Path-connectedness implies connectedness]
  \label{lem:path_connected_implies_connected}
  \uses{def:connected, def:path_connected}
  %\lean{}
  % \leanok
  If $X$ is path-connected, then it is connected.
\end{lemma}
\begin{proof}
  See \textbf{MS-C1541 Metric Spaces}.
\end{proof}

\begin{lemma}[Open connected sets are path-connected]
  \label{lem:open_connected_iff_path_connected}
  \uses{def:connected, def:path_connected, def:open_set, lem:metric_on_C}
  %\lean{}
  % \leanok
  Suppose that $U \subset \bC$ is an open subset of the complex plane.
  Then $U$ is connected if and only if it is path-connected.
\end{lemma}
\begin{proof}
  \uses{lem:path_connected_implies_connected}
  See \textbf{MS-C1541 Metric Spaces}.
\end{proof}

\section{Compactness}

\begin{definition}[Compactness]
  \label{def:compact}
  \uses{def:limit}
  %\lean{}
  % \leanok
  \ldots
\end{definition}

\begin{theorem}[Bolzano-Weierstrass theorem]
  \label{thm:bolzano_weierstrass}
  \uses{def:compact, def:closed_set, lem:metric_on_C, def:bounded}
  %\lean{}
  % \leanok
  %Let $X$ be a metric space.
  A subset $B \subset \bR$ of the real line is compact
  if an only if it is closed and bounded.

  A subset $A \subset \bC$ of the complex plane is compact
  if an only if it is closed and bounded.
\end{theorem}
\begin{proof}
  % \leanok
  See \textbf{MS-C1541 Metric Spaces}.
\end{proof}

\begin{theorem}[Boundedness of continuous functions on compacts]
  \label{thm:continuous_on_compact_implies_bounded}
  \uses{def:compact, def:bounded, def:continuous}
  % \lean{}
  % \leanok
  Suppose that $X$ is compact.
  Then every continuous function $f \colon X \to \bR$
  is bounded.
\end{theorem}
\begin{proof}
  % \uses{}
  % \leanok
  \ldots
\end{proof}

\begin{lemma}[On a compact domain continuity implies uniform continuity]
  \label{lem:on_compact_continuous_implies_uniformly_continuous}
  \uses{def:continuous, def:uniformly_continuous, def:compact}
  %\lean{}
  % \leanok
  If $X$ is compact and a function $f \colon X \to Y$ is continuous, then it is uniformly continuous.
\end{lemma}
\begin{proof}
  See \textbf{MS-C1541 Metric Spaces}.
\end{proof}

\begin{theorem}[Cantor's intersection theorem]
  \label{thm:intersect_nested_compacts}
  \uses{def:compact}
  %\lean{}
  % \leanok
  \ldots
\end{theorem}

\section{Convexity}
