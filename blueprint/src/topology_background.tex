\section{Metrics and related concepts}

\begin{definition}[Metric]
  \label{def:metric_space}
  %\uses{}
  %\lean{}
  % \leanok
  A \term{metric} on a set $X$ is a function
  $d \colon X \times X \to [0,\infty)$ such that
  for all $x_1, x_2, x_3 \in X$ we have
  \begin{align*}
    d(x_1,x_3) \le \; & d(x_1,x_2) + d(x_2,x_3)
            & \text{(triangle inequality)} \\
    d(x_1,x_2) = \; & d(x_2,x_1)
            & \text{(symmetricity)} \\
    d(x_1,_2y) = \; & 0 \text{ if and only if } x_1 = x_2 .
            & \text{(separation of points)}
  \end{align*}
  The set $X$ equipped with the metric $d$ on
  it is called a \term{metric space}.
\end{definition}

\begin{lemma}[Metric in the complex plane]
  \label{lem:metric_on_C}
  \uses{def:metric_space, def:absolute_value}
  % \lean{}
  % \leanok
  The formula
  \begin{align*}
    d(z,w) = |z - w|
    \qquad \text{ for } z,w \in \bC
  \end{align*}
  defines a metric on the complex plane $\bC$.

  (Thus $\bC$ becomes a metric space. Also any subset of $\bC$,
  in particular $\bR \subset \bC$, becomes a metric space when equipped
  with the metric given by the above formula restricted to the subset.)
\end{lemma}
\begin{proof}
  \uses{lem:absolute_value_properties}
  % \leanok
  \ldots
\end{proof}

\begin{definition}[Ball (disk)]
  \label{def:ball}
  \uses{def:metric_space}
  %\lean{}
  % \leanok
  Let $X$ be a metric space with metric $d \colon X \times X \to [0,\infty)$.
  Let $x_0 \in X$ be a point and let $r>0$.

  The set
  \begin{align*}
    \BallRC{r}{x_0} = \set{x \in X \; \Big| \; d(x,x_0) < r}
  \end{align*}
  is called an \term{open ball} in $X$, centered at $x_0$, and with radius $r$.

  The set
  \begin{align*}
    \CBallRC{r}{x_0} = \set{x \in X \; \Big| \; d(x,x_0) \le r}
  \end{align*}
  is called a \term{closed ball} in $X$, centered at $x_0$, and with radius $r$.

  The set
  \begin{align*}
    \SphereRC{r}{x_0} = \set{x \in X \; \Big| \; d(x,x_0) = r}
  \end{align*}
  is called a \term{sphere} in $X$, centered at $x_0$, and with radius $r$.

  (In the case of the complex plane $\bC$, the term \term{disk}
  may be used instead of the general term ball, and the term \term{circle}
  may be used instead of the general term sphere.)
\end{definition}

\begin{definition}[Open set]
  \label{def:open_set}
  \uses{def:metric_space, def:ball}
  % \lean{}
  % \leanok
  Let $X$ be a metric space with metric $d \colon X \times X \to [0,\infty)$.
  A subset $U \subset X$ is said to be an \term{open set} if
  for every $x \in U$ there exists an $r > 0$ such that
  $\Ball{r}{x} \subset U$.
\end{definition}

\begin{definition}[Closed set]
  \label{def:closed_set}
  \uses{def:open_set}
  % \lean{}
  % \leanok
  Let $X$ be a metric space with metric $d \colon X \times X \to [0,\infty)$.
  A subset $F \subset X$ is said to be a \term{closed set} if
  the complement $X \setminus F \subset X$ is an open set.
\end{definition}

\begin{definition}[Boundedness]
  \label{def:bounded}
  \uses{def:metric_space}
  % \lean{}
  % \leanok
  Let $X$ be a metric space with metric $d \colon X \times X \to [0,\infty)$.

  A subset $A \subset X$ is said to be bounded if there exists
  a number $M>0$ such that $d(x,y) \le M$ for all $x,y \in A$.
  (If $X$ is nonempty, an equivalent definition would be that
  $A$ is bounded if it is a subset of some ball in $X$.)

  A function $f \colon Z \to X$ with values in a metric space~$X$
  is \term{bounded} if the set $f[Z] \subset X$ of its values is a
  bounded subset of~$X$.

  (In the case $X = \bC$ we have the following further characterizations:
  A subset $A \subset \bC$ is bounded if and only if there exists an $R>0$
  such that $|z| \le R$ for all $z \in A$. A function $f \colon Z \to \bC$
  is bounded if and only if there exists an $R>0$ such that $|f(z)| \le R$
  for all $z \in Z$.)
\end{definition}



\section{Limits}

\begin{definition}[Limit]
  \label{def:limit}
  \uses{def:metric_space}
  %\lean{}
  % \leanok
  %Let $X$ be a metric space with metric $d \colon X \times X \to [0,\infty)$.
  \ldots
\end{definition}

\begin{definition}[Cauchy sequence]
  \label{def:cauchy_sequence}
  \uses{def:metric_space}
  %\lean{}
  % \leanok
  %Let $X$ be a metric space with metric $d \colon X \times X \to [0,\infty)$.
  \ldots
\end{definition}

\begin{lemma}[Every real Cauchy sequence converges]
  \label{lem:real_complete}
  \uses{def:cauchy_sequence, def:limit}
  %\lean{}
  % \leanok
  If a real number sequence $(x_n)_{n \in \bN}$ is Cauchy,
  then it converges to a limit $\lim_{n \to \infty} x_n \in \bR$.

  (This property is known as \term{completeness} of the metric space $\bR$.)
\end{lemma}
\begin{proof}
  %\uses{def:cauchy_sequence}
  % \leanok
  See \textbf{MS-C1541 Metric Spaces}.
\end{proof}

\begin{lemma}[Every complex Cauchy sequence converges]
  \label{lem:complex_complete}
  \uses{def:cauchy_sequence, def:limit, lem:metric_on_C}
  %\lean{}
  % \leanok
  If a complex number sequence $(z_n)_{n \in \bN}$ is Cauchy,
  then it converges to a limit $\lim_{n \to \infty} z_n \in \bC$.

  (This property is known as \term{completeness} of the metric space $\bC$.)
\end{lemma}
\begin{proof}
  \uses{lem:real_complete}
  % \leanok
  See \textbf{MS-C1541 Metric Spaces}.

  (Idea: This follows from Lemma~\ref{lem:real_complete} by
  considering real and imaginary parts separately and picking
  a subsequence of a subsequence.)
\end{proof}



\section{Continuity}

\begin{definition}[Continuity]
  \label{def:continuous}
  \uses{def:metric_space}
  %\lean{}
  % \leanok
  Let $X$ and $Y$ be metric spaces.
  A functions $f \colon X \to Y$ is \term{continuous} if\ldots
\end{definition}

\begin{definition}[Uniform continuity]
  \label{def:uniformly_continuous}
  \uses{def:metric_space}
  %\lean{}
  % \leanok
  Let $X$ and $Y$ be metric spaces.
  A functions $f \colon X \to Y$ is \term{uniformly continuous} if\ldots
\end{definition}

\begin{lemma}[Uniform continuity implies continuity]
  \label{lem:uniformly_continuous_implies_continuous}
  \uses{def:continuous, def:uniformly_continuous}
  %\lean{}
  % \leanok
  If a function $f \colon X \to Y$ is uniformly continuous, then it is continuous.
\end{lemma}
\begin{proof}
  See \textbf{MS-C1541 Metric Spaces}.

  (The very easy proof is also a good exercise.)
\end{proof}



\section{Connectedness and path-connectedness}

\begin{definition}[Connectedness]
  \label{def:connected}
  \uses{def:metric_space}
  %\lean{}
  % \leanok
  \ldots
\end{definition}

\begin{definition}[Path-connectedness]
  \label{def:path_connected}
  \uses{def:path}
  %\lean{}
  % \leanok
  \ldots
\end{definition}

\begin{lemma}[Path-connectedness implies connectedness]
  \label{lem:path_connected_implies_connected}
  \uses{def:connected, def:path_connected}
  %\lean{}
  % \leanok
  If $X$ is path-connected, then it is connected.
\end{lemma}
\begin{proof}
  See \textbf{MS-C1541 Metric Spaces}.
\end{proof}

\begin{lemma}[Open connected sets are path-connected]
  \label{lem:open_connected_iff_path_connected}
  \uses{def:connected, def:path_connected, def:open_set, lem:metric_on_C}
  %\lean{}
  % \leanok
  Suppose that $U \subset \bC$ is an open subset of the complex plane.
  Then $U$ is connected if and only if it is path-connected.
\end{lemma}
\begin{proof}
  \uses{lem:path_connected_implies_connected}
  See \textbf{MS-C1541 Metric Spaces}.
\end{proof}

\section{Compactness}

\begin{definition}[Compactness]
  \label{def:compact}
  \uses{def:limit}
  %\lean{}
  % \leanok
  \ldots
\end{definition}

\begin{theorem}[Bolzano-Weierstrass theorem]
  \label{thm:bolzano_weierstrass}
  \uses{def:compact, def:closed_set, lem:metric_on_C, def:bounded}
  %\lean{}
  % \leanok
  %Let $X$ be a metric space.
  A subset $B \subset \bR$ of the real line is compact
  if an only if it is closed and bounded.

  A subset $A \subset \bC$ of the complex plane is compact
  if an only if it is closed and bounded.
\end{theorem}
\begin{proof}
  % \leanok
  See \textbf{MS-C1541 Metric Spaces}.
\end{proof}

\begin{theorem}[Boundedness of continuous functions on compacts]
  \label{thm:continuous_on_compact_implies_bounded}
  \uses{def:compact, def:bounded, def:continuous}
  % \lean{}
  % \leanok
  Suppose that $X$ is compact.
  Then every continuous function $f \colon X \to \bR$
  is bounded.
\end{theorem}
\begin{proof}
  % \uses{}
  % \leanok
  \ldots
\end{proof}

\begin{lemma}[On a compact domain continuity implies uniform continuity]
  \label{lem:on_compact_continuous_implies_uniformly_continuous}
  \uses{def:continuous, def:uniformly_continuous, def:compact}
  %\lean{}
  % \leanok
  If $X$ is compact and a function $f \colon X \to Y$ is continuous, then it is uniformly continuous.
\end{lemma}
\begin{proof}
  See \textbf{MS-C1541 Metric Spaces}.
\end{proof}

\begin{theorem}[Cantor's intersection theorem]
  \label{thm:intersect_nested_compacts}
  \uses{def:compact}
  %\lean{}
  % \leanok
  \ldots
\end{theorem}

\section{Convexity}
