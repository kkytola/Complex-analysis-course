\section{Metrics and related concepts}

\begin{definition}[Ball (disk)]
  \label{def:ball}
  %\uses{}
  %\lean{}
  % \leanok
  A \term{ball} is \ldots
\end{definition}

\section{Limits}

\begin{definition}[Limit]
  \label{def:limit}
  %\uses{}
  %\lean{}
  % \leanok
  Let $X$ be a metric space.
  \ldots
\end{definition}

\begin{definition}[Cauchy sequence]
  \label{def:cauchy_sequence}
  %\uses{}
  %\lean{}
  % \leanok
  %Let $X$ be a metric space.
  \ldots
\end{definition}

\begin{lemma}[Every real Cauchy sequence converges]
  \label{lem:real_complete}
  \uses{def:cauchy_sequence, def:limit}
  %\lean{}
  % \leanok
  If a real number sequence $(x_n)_{n \in \bN}$ is Cauchy,
  then it converges to a limit $\lim_{n \to \infty} x_n \in \bR$.
\end{lemma}
\begin{proof}
  %\uses{def:cauchy_sequence}
  % \leanok
  See \textbf{MS-C1541 Metric Spaces}.
\end{proof}

\begin{lemma}[Every complex Cauchy sequence converges]
  \label{lem:complex_complete}
  \uses{def:cauchy_sequence, def:limit, def:complex_numbers}
  %\lean{}
  % \leanok
  %Let $X$ be a metric space.
  \ldots
\end{lemma}
\begin{proof}
  \uses{lem:real_complete}
  % \leanok
  See \textbf{MS-C1541 Metric Spaces}.

  (Idea: This follows from Lemma~\ref{lem:real_complete} by
  considering real and imaginary parts separately and picking
  a subsequence of a subsequence.)
\end{proof}

\section{Continuity}

\begin{definition}[Continuity]
  \label{def:continuous}
  %\uses{}
  %\lean{}
  % \leanok
  Let $X$ and $Y$ be metric spaces.
  A functions $f \colon X \to Y$ is \term{continuous} if\ldots
\end{definition}

\begin{definition}[Uniform continuity]
  \label{def:uniformly_continuous}
  %\uses{}
  %\lean{}
  % \leanok
  Let $X$ and $Y$ be metric spaces.
  A functions $f \colon X \to Y$ is \term{uniformly continuous} if\ldots
\end{definition}

\begin{lemma}[Uniform continuity implies continuity]
  \label{lem:uniformly_continuous_implies_continuous}
  \uses{def:continuous, def:uniformly_continuous}
  \lean{}
  % \leanok
  If a function $f \colon X \to Y$ is uniformly continuous, then it is continuous.
\end{lemma}
\begin{proof}
  See \textbf{MS-C1541 Metric Spaces}.

  (The very easy proof is also a good exercise.)
\end{proof}

\section{Connectedness and path-connectedness}

\begin{definition}[Connectedness]
  \label{def:connected}
  %\uses{}
  %\lean{}
  % \leanok
  \ldots
\end{definition}

\begin{definition}[Path-connectedness]
  \label{def:path_connected}
  \uses{def:path}
  %\lean{}
  % \leanok
  \ldots
\end{definition}

\begin{lemma}[Path-connectedness implies connectedness]
  \label{lem:path_connected_implies_connected}
  \uses{def:connected, def:path_connected}
  \lean{}
  % \leanok
  If $X$ is path-connected, then it is connected.
\end{lemma}
\begin{proof}
  See \textbf{MS-C1541 Metric Spaces}.
\end{proof}

\begin{lemma}[Open connected sets are path-connected]
  \label{lem:connected_open_implies_path_connected}
  \uses{def:connected, def:path_connected}
  \lean{}
  % \leanok
  If a subset $A \subset \bC$ of the complex plane is open
  and connected, then it is path-connected.
\end{lemma}
\begin{proof}
  See \textbf{MS-C1541 Metric Spaces}.
\end{proof}

\section{Compactness}

\begin{definition}[Compactness]
  \label{def:compact}
  \uses{def:limit}
  %\lean{}
  % \leanok
  \ldots
\end{definition}

\begin{theorem}[Bolzano-Weierstrass theorem]
  \label{thm:bolzano_weierstrass}
  \uses{def:compact}
  %\lean{}
  % \leanok
  %Let $X$ be a metric space.
  A subset $B \subset \bR$ of the real line is compact
  if an only if it is closed and bounded.

  A subset $A \subset \bC$ of the complex plane is compact
  if an only if it is closed and bounded.
\end{theorem}
\begin{proof}
  % \leanok
  See \textbf{MS-C1541 Metric Spaces}.
\end{proof}

\begin{lemma}[On a compact domain continuity implies uniform continuity]
  \label{lem:on_compact_continuous_implies_uniformly_continuous}
  \uses{def:continuous, def:uniformly_continuous, def:compact}
  \lean{}
  % \leanok
  If $X$ is compact and a function $f \colon X \to Y$ is continuous, then it is uniformly continuous.
\end{lemma}
\begin{proof}
  See \textbf{MS-C1541 Metric Spaces}.
\end{proof}

\begin{theorem}[Cantor's intersection theorem]
  \label{thm:intersect_nested_compacts}
  \uses{def:compact}
  \lean{}
  % \leanok
  \ldots
\end{theorem}

\section{Convexity}
