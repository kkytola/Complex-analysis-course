\section{Complex-valued integrals}

\begin{definition}[Integral of a complex-valued function]
  \label{def:complex_valued_integral}
  \uses{def:riemann_integral, lem:continuous_complex, lem:continuous_implies_riemann_integrable}
  % \lean{}
  % \leanok
  Let $f \colon [a,b] \to \bC$ be a complex-valued continuous function
  defined on a closed interval $[a,b] \subset \bR$. We define the
  \term{integral} of $f$ as
  \begin{align*}
    \int_a^b f(t) \, \ud t
      \; = \; \int_a^b \re \big( f(t) \big) \, \ud t
              \; + \; \ii \int_a^b \im \big( f(t) \big) \, \ud t .
  \end{align*}

  (Note that on the right hand side we have just Riemann integrals
  of the continuous real-valued functions $t \mapsto \re \big( f(t) \big)$
  and $t \mapsto \im \big( f(t) \big)$.)
\end{definition}

\begin{lemma}[Complex linearity of complex-valued integrals]
  \label{lem:linearity_of_complex_integral}
  \uses{def:complex_valued_integral, def:linear}
  % \lean{}
  % \leanok
  If $f, g \colon [a,b] \to \bC$ are complex-valued continuous
  functions defined on a closed interval $[a,b] \subset \bR$,
  then the integral of their sum is
  \begin{align*}
    \int_a^b \big( f(t) + g(t) \big) \, \ud t
      \; = \; \int_a^b f(t) \, \ud t + \int_a^b g(t) \, \ud t .
  \end{align*}

  If $f \colon [a,b] \to \bC$ is a complex-valued continuous
  function defined on a closed interval $[a,b] \subset \bR$,
  and $\lambda \in \bC$ is a complex number, then the integral of
  the scalar multiple of $f$ is
  \begin{align*}
    \int_a^b \lambda \, f(t) \, \ud t
      \; = \; \lambda \int_a^b f(t) \, \ud t .
  \end{align*}
\end{lemma}
\begin{proof}
  % \uses{}
  % \leanok
  \ldots
\end{proof}

\begin{lemma}[Fundamental theorem of calculus for complex-valued integrals]
  \label{lem:ftc_for_complex_valued_integrals}
  \uses{def:complex_valued_integral}
  % \lean{}
  % \leanok
  Suppose that $f \colon [a,b] \to \bC$ is a continuously differentiable
  complex-valued function on a closed interval $[a,b] \subset \bR$.
  Denote its derivative by $\dot{f}(t) = \frac{\ud}{\ud t} f(t)$.
  Then for the integral of the derivative of $f$ we have
  \begin{align*}
    \int_a^b \dot{f}(t) \, \ud t \; = \; f(b) - f(a) .
  \end{align*}
\end{lemma}
\begin{proof}
  % \uses{}
  % \leanok
  \ldots
\end{proof}



\section{Paths in the complex plane}

\begin{definition}[Path \Palka{Sec.~IV.1.1}]
  \label{def:path}
  \uses{def:continuous}
  %\lean{}
  % \leanok
  A \term{path} in the complex plane is a continuous function $\gamma \colon [a,b] \to \bC$
  from a closed interval $[a,b] \subset \bR$ to $\bC$.

  When $A \subset \bC$ is a subset of the complex plane, we say that $\gamma$ is a path in $A$
  if $\gamma(t) \in A$ for all $t \in [a,b]$.

  If the starting point and the end point of the path~$\gamma$ are the same,
  $\gamma(a) = \gamma(b)$, then we say that $\gamma$ is a \term{closed path}.
\end{definition}

\begin{definition}[Smooth path \Palka{Sec.~IV.1.2}]
  \label{def:smooth_path}
  \uses{def:path, def:continuous}
  %\lean{}
  % \leanok
  A path $\gamma \colon [a,b] \to \bC$ is \term{smooth}
  if it is continuously differentiable, i.e., the derivative
  \begin{align*}
    \dot{\gamma}(t) = \frac{\ud}{\ud t} \gamma(t)
  \end{align*}
  with respect to the parameter $t$
  exists for all $t \in [a,b]$ (one-sided derivative at the end points),
  and defines a continuous complex-valued function $t \mapsto \dot{\gamma}(t)$
  on $[a,b]$.
\end{definition}

\begin{definition}[Contour / piecewise smooth path \Palka{Sec.~IV.1.2}]
  \label{def:contour}
  \uses{def:smooth_path, def:continuous}
  %\lean{}
  % \leanok
  A \term{contour} (also called a \term{piecewise smooth path})
  is a continuous function $\gamma \colon [a,b] \to \bC$
  such that for some finite subdivision $a = t_0 < t_1 < \ldots < t_n = b$,
  the restrictions $\gamma|_{[t_{j-1},t_j]}$
  to the subintervals $[t_{j-1},t_j] \subset [a,b]$
  are smooth paths for each $j = 1, \ldots, n$.

  If the starting point and the end point of the contour~$\gamma$ are the same,
  $\gamma(a) = \gamma(b)$, then we say that $\gamma$ is a \term{closed contour}.
\end{definition}

\begin{definition}[Reverse path \Palka{Sec.~IV.1.4}]
  \label{def:reverse_path}
  \uses{def:path}
  %\lean{}
  % \leanok
  Given a path $\gamma \colon [a,b] \to \bC$,
  the \term{reverse path} $\opp{\gamma} \colon [a,b] \to \bC$
  is the path defined by
  \begin{align*}
    \opp{\gamma}(t) = \gamma(a+b-t) \qquad \text{ for } t \in [a,b] .
  \end{align*}
\end{definition}

\begin{definition}[Concatenation of paths \Palka{Sec.~IV.1.4}]
  \label{def:path_concatenation}
  \uses{def:path}
  %\lean{}
  % \leanok
  Given path $\gamma \colon [a,b] \to \bC$ and
  $\eta \colon [c,d] \to \bC$ with $\gamma(b) = \eta(c)$
  (the starting point of $\eta$ coincides with the end point of $\gamma$),
  the \term{concatenation} of $\gamma$ and $\eta$ is the path
  $\gamma \concat \eta \colon [a,b+d-c] \to \bC$ defined by
  \begin{align*}
    (\gamma \concat \eta)(t) =
    \begin{cases}
      \gamma(t) & \text{ for } t \in [a,b] , \\
      \eta(c+t-b)) & \text{ for } t \in [b,b+d-c] .
    \end{cases}
  \end{align*}

  (The slightly cumbersome formula in the second case is due to the fact
  that we need to attach the two parameter intervals of lengths
  $b-a$ and $d-c$ to each other, and we have, somewhat arbitrarily,
  chosen to glue them to form the interval $[a,b+d-c]$.)
\end{definition}

\begin{definition}[Reparametrization of paths \Palka{Sec.~IV.1.5}]
  \label{def:path_reparametrization}
  \uses{def:path, lem:continuous_bijection_homeomorphism, def:contour}
  %\lean{}
  % \leanok
  Given a path $\gamma \colon [a,b] \to \bC$ and a continuous increasing
  bijection $\phi \colon [c,d] \to [a,b]$, we define the \term{reparametrization}
  of $\gamma$ by $\phi$ as the path
  \begin{align*}
    \gamma \circ \phi \, \colon \, [c,d] \, \to \; & \bC \\
    t \, \mapsto \; & \gamma(\phi(t)) .
  \end{align*}

  Note that
  \begin{itemize}
    \item $\phi^{-1} \colon [a,b] \to [c,d]$ is also a continuous increasing
      bijection (a continuous bijection from the
      compact $[a,b]$ is automatically a homeomorphism; see ) and
      reparametrization can be undone by rereparametrizing by $\phi^{-1}$;
      \item If both $\gamma$ and the reparametrization function $\phi$ are
      smooth (continuously differentiable), then the reparametrized
      path $\gamma \circ \phi$ is also smooth;
      \item If both $\gamma$ and the reparametrization function $\phi$ are
      piecewise smooth, then the reparametrized path $\gamma \circ \phi$
      is also piecewise smooth, i.e., a contour.
  \end{itemize}
\end{definition}



\section{Integrals along paths}

\begin{definition}[Contour integral along a smooth path \Palka{Sec.~IV.2.1}]
  \label{def:smooth_contour_integral}
  \uses{def:complex_valued_integral, def:smooth_path}
  %\lean{}
  % \leanok
  Let $f : A \to \bC$ be a continuous function defined on a subset $A \subset \bC$.
  %and write $u = \re f, v = \im f \colon [a,b] \to \bR$
  %for its real and imaginary parts so that $f(z) = u(z) + \ii v(z)$ for $z \in A$.
  Let $\gamma \colon [a,b] \to A$ be a smooth path in~$A$.
  We define the \term{integral of $f$ along $\gamma$} as
  \begin{align*}
    \int_\gamma f(z) \, dz
    \; = \; \int_a^b f \big( \gamma(t) \big) \, \dot{\gamma}(t) \; \ud t .
    %= \; \int_a^b u \big( \gamma(t) \big) \gamma'(t) \, \ud t
    %     + \ii \int_a^b v \big( \gamma(t) \big) \gamma'(t) \, \ud t .
  \end{align*}

  (Here $\dot{\gamma}(t) = \frac{\ud}{\ud t} \gamma(t)$ denotes the derivative of
  the smooth path $\gamma$ with respect to its parameter $t$.)
\end{definition}

Sometimes it is appropriate to integrate functions with respect to the
arc length in the following sense.
\begin{definition}[Arc length integral along a smooth path \Palka{Sec.~IV.2.1}]
  \label{def:smooth_arc_length_integral}
  \uses{def:complex_valued_integral, lem:continuous_implies_riemann_integrable, def:smooth_path}
  %\lean{}
  % \leanok
  Let $f : A \to \bC$ be a continuous function defined on a subset $A \subset \bC$.
  %and write $u = \re f, v = \im f \colon [a,b] \to \bR$
  %for its real and imaginary parts so that $f(z) = u(z) + \ii v(z)$ for $z \in A$.
  Let $\gamma \colon [a,b] \to A$ be a smooth path in~$A$.
  We define the \term{integral of $f$ with respect to the arc length of $\gamma$} as
  \begin{align*}
    \int_\gamma f(z) \, |dz|
    \; = \; \int_a^b f \big( \gamma(t) \big) \, |\dot{\gamma}(t)| \; \ud t .
    %= \; \int_a^b u \big( \gamma(t) \big) \gamma'(t) \, \ud t
    %     + \ii \int_a^b v \big( \gamma(t) \big) \gamma'(t) \, \ud t .
  \end{align*}

  (Here $\dot{\gamma}(t) = \frac{\ud}{\ud t} \gamma(t) \in \bC$ denotes the
  derivative of the smooth path $\gamma$ with respect to its parameter $t$,
  and $|\dot{\gamma}(t)| \ge 0$ denotes the absolute value of this derivative.)
\end{definition}

In order to extend the definition of contour integrals to piecewise smooth paths,
we note that the definition behaves additively under path concatenation.

\begin{lemma}[Contour integrals and smooth path concatenation \Palka{Lem~IV.2.1(iv)}]
  \label{lem:integral_concatenation}
  \uses{def:smooth_contour_integral, def:smooth_arc_length_integral, def:path_concatenation}
  %\lean{}
  % \leanok
  If a smooth path $\gamma$ in $A$ is a concatenation of smooth
  paths $\eta_1, \ldots, \eta_n$, and
  $f : A \to \bC$ is a continuous function defined on $A \subset \bC$,
  then we have
  \begin{align*}
    \int_\gamma f(z) \, \ud z \; = \; \sum_{j=1}^n \int_{\eta_j} f(z) \, \ud z
  \end{align*}
  and
  \begin{align*}
    \int_\gamma f(z) \, |\ud z| \; = \; \sum_{j=1}^n \int_{\eta_j} f(z) \, |\ud z| .
  \end{align*}
\end{lemma}
\begin{proof}
  %\uses{}
  % \leanok
  %Straightforward from a corresponding property of Riemann integrals.
  \ldots
\end{proof}

By virtue of the above, the following gives a well-defined meaning
to integrals along piecewise smooth paths.
\begin{definition}[Contour integral \Palka{Sec.~IV.2.1}]
  \label{def:contour_integral}
  \uses{def:smooth_contour_integral, def:contour, lem:integral_concatenation}
  %\lean{}
  % \leanok
  Let $f : A \to \bC$ be a continuous function defined on a subset $A \subset \bC$.
  Let $\gamma \colon [a,b] \to A$ be a piecewise smooth path in~$A$,
  which is a concatenation of smooth paths $\eta_1, \ldots, \eta_n$.
  We define the \term{integral of $f$ along $\gamma$} as
  \begin{align*}
    \int_\gamma f(z) \, dz \; = \; \sum_{j=1}^n \int_{\eta_j} f(z) \, dz .
  \end{align*}
\end{definition}

\begin{definition}[Arc-length integral]
  \label{def:arc_length_integral}
  \uses{def:smooth_arc_length_integral, def:contour, lem:integral_concatenation}
  %\lean{}
  % \leanok
  Let $f : A \to \bC$ be a continuous function defined on a subset $A \subset \bC$.
  Let $\gamma \colon [a,b] \to A$ be a piecewise smooth path (i.e., a contour) in~$A$,
  which is a concatenation of smooth paths $\eta_1, \ldots, \eta_n$.
  We define the \term{integral of $f$ along $\gamma$} as
  \begin{align*}
    \int_\gamma f(z) \, dz \; = \; \sum_{j=1}^n \int_{\eta_j} f(z) \, dz .
  \end{align*}
\end{definition}

\begin{definition}[Length of a path or a contour]
  \label{def:length}
  \uses{def:arc_length_integral}
  % \lean{}
  % \leanok
  Let $\gamma \colon [a,b] \to \bC$ be a piecewise smooth path (i.e., a contour)
  in $\bC$. The \term{length} $\ell(\gamma)$ of $\gamma$ is defined as
  \begin{align*}
    \ell(\gamma) = \int_\gamma |\ud z| .
  \end{align*}
\end{definition}

\begin{lemma}[Reparametrization invariance of integrals \Palka{Lem~IV.2.1(v)}]
  \label{lem:reparametrization_invariance_of_integral}
  \uses{def:contour_integral, def:arc_length_integral, def:path_reparametrization}
  %\lean{}
  % \leanok
  Let $\gamma$ be a piecewise smooth path in~$A$,
  and let $\widetilde{\gamma}$ be obtained from $\gamma$ by an
  orientation-preserving reparametrization. Then for any continuous
  function $f : A \to \bC$ we have
  \begin{align*}
    \int_{\widetilde{\gamma}} f(z) \, dz \; = \; \int_{\gamma} f(z) \, dz
  \end{align*}
  and
  \begin{align*}
    \int_{\widetilde{\gamma}} f(z) \, |dz| \; = \; \int_{\gamma} f(z) \, |dz| .
  \end{align*}
\end{lemma}

\begin{lemma}[Contour integrals and path reversal \Palka{Lem~IV.2.1(iii)}]
  \label{lem:integral_reversal}
  \uses{def:contour_integral, def:arc_length_integral, def:reverse_path}
  %\lean{}
  % \leanok
  If $f : A \to \bC$ is a continuous function defined on $A \subset \bC$,
  and $\gamma$ is a piecewise path in $A$, then for the contour integral
  and the arc length integral behave as follows under path reversal:
  then we have
  \begin{align*}
    \int_{\opp{\gamma}} f(z) \, \ud z \; = \; - \int_{\gamma} f(z) \, \ud z
  \end{align*}
  and
  \begin{align*}
    \int_{\opp{\gamma}} f(z) \, |\ud z| \; = \; \int_{\gamma} f(z) \, |\ud z|
  \end{align*}
\end{lemma}
\begin{proof}
  %\uses{}
  % \leanok
  %Straightforward from a corresponding property of Riemann integrals.
  \ldots
\end{proof}

\begin{lemma}[Linearity of integrals \Palka{Lem~IV.2.1(i-ii)}]
  \label{lem:integral_linearity}
  \uses{def:contour_integral, def:arc_length_integral, def:linear}

  Let $A \subset \bC$ be a subset of the complex plane let and $\gamma \colon [a,b] \to A$
  be a contour in~$A$.

  If $f, g \colon A \to \bC$ are continuous
  functions defined on $A$, then the contour integral
  and the arc length integral of their sum are
  \begin{align*}
    \int_\gamma \big( f(z) + g(z) \big) \, \ud z
      \; = \; & \int_\gamma f(z) \, \ud z + \int_\gamma g(z) \, \ud z \\
    \int_\gamma \big( f(z) + g(z) \big) \, |\ud z|
      \; = \; & \int_\gamma f(z) \, |\ud z| + \int_\gamma g(z) \, |\ud z| .
  \end{align*}

  If $f \colon A \to \bC$ is a complex-valued continuous
  function defined on $A$,
  and $\lambda \in \bC$ is a complex number, then the contour
  integral and the arc length integral of the scalar multiple of $f$ are
  \begin{align*}
    \int_\gamma \lambda \, f(z) \, \ud z
      \; = \; & \lambda \int_\gamma f(z) \, \ud z \\
    \int_\gamma \lambda \, f(z) \, |\ud z|
      \; = \; & \lambda \int_\gamma f(z) \, |\ud z| .
  \end{align*}
  %
  %%\lean{}
  %% \leanok
  %\begin{align*}
  %  \int_\gamma \Big( f_1(z) + f_2(z) \Big) \, \ud z
  %  \; = \; \int_\gamma f_1(z) \, \ud z + \int_\gamma f_2(z) \, \ud z
  %\end{align*}
  %\begin{align*}
  %  \int_\gamma c \, f(z) \, \ud z \; = \; c \int_\gamma f(z) \, \ud z .
  %\end{align*}
\end{lemma}
\begin{proof}
  \uses{lem:linearity_of_complex_integral}
  % \leanok
  \ldots
\end{proof}

\begin{lemma}[Triangle inequality for contour integrals \Palka{Lem~IV.2.1(vi)}]
  \label{lem:integral_triangle_inequality}
  \uses{def:contour_integral, def:arc_length_integral}
  %\lean{}
  % \leanok
  Let $f : A \to \bC$ be a continuous function defined on $A \subset \bC$,
  and let $\gamma$ be a contour in $A$. Then we have
  \begin{align*}
    \left| \int_\gamma f(z) \, \ud z \right| \; \le \; \int_\gamma |f(z)| \, |dz| .
  \end{align*}
\end{lemma}
\begin{proof}
  %\uses{}
  % \leanok
  \ldots
\end{proof}

\begin{corollary}[An a priori bound for contour integrals]
  \label{lem:contour_integral_bound}
  \uses{def:contour_integral, def:length}
  % \lean{}
  % \leanok
  Let $f : A \to \bC$ be a continuous function defined on $A \subset \bC$,
  and let $\gamma$ be a contour in $A$.
  Assume that $|f(z)| \le M$ for all points $z$ on the contour $\gamma$.
  Then we have
  \begin{align*}
    \left| \int_\gamma f(z) \, \ud z \right| \; \le \; M \, \ell(\gamma) ,
  \end{align*}
  where $\ell(\gamma) = \int_\gamma |\ud z|$ denotes the length of
  the contour $\gamma$.
\end{corollary}
\begin{proof}
  \uses{lem:integral_triangle_inequality}
  % \leanok
  \ldots
\end{proof}



\section{Primitives}

\begin{definition}[Primitive of a function \Palka{Sec.~IV.2.3}]
  \label{def:primitive}
  \uses{def:analytic_function}
  %\lean{}
  % \leanok
  Let $f : U \to \bC$ be a function defined on an open subset $U \subset \bC$.
  A \term{primitive} of $f$ is a function $F : U \to \bC$ such that
  $F$ is analytic (i.e., complex differentiable) on $U$,
  \begin{align*}
    F'(z) = f(z) \qquad \text{ for all } z \in U .
  \end{align*}
\end{definition}

\begin{theorem}[Fundamental theorem of calculus for contour integrals \Palka{Thm~IV.2.2}]
  \label{thm:ftc_for_contour_integrals}
  \uses{def:primitive, def:contour_integral}
  %\lean{}
  % \leanok
  Suppose that $f \colon U \to \bC$ is a continuous function
  on an open set $U \subset \bC$, and that $f$ has a
  primitive $F \colon U \to \bC$.
  Then for any contour $\gamma \colon [a,b] \to U$ we have
  \begin{align*}
    \int_\gamma f(z) \, dz \; = \; F \big( \gamma(b) \big) - F \big( \gamma(a) \big) .
  \end{align*}
  In particular for any closed contour $\gamma$ in $U$, we have
  \begin{align*}
    \oint_\gamma f(z) \, dz \; = \; 0 .
  \end{align*}
\end{theorem}
\begin{proof}
  \uses{lem:ftc_for_complex_valued_integrals}
  % \leanok
  \ldots
\end{proof}

\begin{lemma}[Existence of primitives for monomials]
  \label{lem:monomial_primitive}
  \uses{def:primitive, def:polynomial, def:rational_function}
  % \lean{}
  % \leanok
  For $n \in \set{0,1,2,\ldots}$, the monomial
  function $f(z) = z^n$ has a primitive
  $F(z) = \frac{1}{n+1} z^{n+1} + c$ (with $c \in \bC$ arbitrary)
  in the whole complex plane~$\bC$.

  For $n \in \set{-2,-3,-4,\ldots}$, the monomial
  function $f(z) = z^n$ has a primitive
  $F(z) = \frac{1}{n+1} z^{n+1} + c$ (with $c \in \bC$ arbitrary)
  in the punctured complex plane~$\bC \setminus \set{0}$.

  The monomial function $f(z) = z^{-1} = \frac{1}{z}$ does not have
  a primitive in the punctured complex plane~$\bC \setminus \set{0}$.
\end{lemma}
\begin{proof}
  \uses{thm:ftc_for_contour_integrals}
  % \leanok
  \ldots
\end{proof}

\begin{theorem}[Characterization of the existence of primitives]
  \label{thm:primitive_characterization}
  \uses{def:primitive}
  % \lean{}
  % \leanok
  Let $f \colon U \to \bC$ be a continuous function on an open
  set $U \subset \bC$. Then the following conditions are equivalent:
  \begin{itemize}
    \item[(a)] $f$ has a primitive on $U$;
    \item[(b)] the contour integrals $\int_\gamma f(z) \, \ud z$
      of $f$ along contours $\gamma$ in $U$ only depend on the
      starting point and the end point of $\gamma$;
    \item[(c)] for all closed contours $\gamma$ in $U$ we have
      $\oint_\gamma f(z) \, \ud z = 0$.
  \end{itemize}
\end{theorem}
\begin{proof}
  \uses{thm:ftc_for_contour_integrals}
  % \leanok
  \ldots
\end{proof}
