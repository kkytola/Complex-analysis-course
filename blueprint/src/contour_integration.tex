\section{Paths in the complex plane}

\begin{definition}[Path \Palka{Sec.~IV.1.1}]
  \label{def:path}
  \uses{def:continuous}
  %\lean{}
  % \leanok
  A \term{path} in the complex plane is a continuous function $\gamma \colon [a,b] \to \bC$
  from a closed interval $[a,b] \subset \bR$ to $\bC$.

  When $A \subset \bC$ is a subset of the complex plane, we say that $\gamma$ is a path in $A$
  if $\gamma(t) \in A$ for all $t \in [a,b]$.
\end{definition}

\begin{definition}[Smooth path \Palka{Sec.~IV.1.2}]
  \label{def:smooth_path}
  \uses{def:path, def:continuous}
  %\lean{}
  % \leanok
  A path $\gamma \colon [a,b] \to \bC$ is \term{smooth}
  if it is continuously differentiable, i.e., the derivative
  \begin{align*}
    \dot{\gamma}(t) = \frac{\ud}{\ud t} \gamma(t)
  \end{align*}
  with respect to the parameter $t$
  exists for all $t \in [a,b]$ (one-sided derivative at the end points),
  and defines a continuous complex-valued function $t \mapsto \dot{\gamma}(t)$
  on $[a,b]$.
\end{definition}

\begin{definition}[Contour (piecewise smooth path) \Palka{Sec.~IV.1.2}]
  \label{def:piecewise_smooth_path}
  \uses{def:smooth_path, def:continuous}
  %\lean{}
  % \leanok
  A \term{contour} (also called a \term{piecewise smooth path})
  is a continuous function $\gamma \colon [a,b] \to \bC$
  such that for some finite subdivision $a = t_0 < t_1 < \ldots < t_n = b$,
  the restrictions $\gamma|_{[t_{j-1},t_j]}$, $j = 1, \ldots, n$ are smooth paths.
\end{definition}

\begin{definition}[Reverse path \Palka{Sec.~IV.1.4}]
  \label{def:reverse_path}
  \uses{def:path}
  %\lean{}
  % \leanok
  Given a path $\gamma \colon [a,b] \to \bC$,
  the \term{reverse path} $\opp{\gamma} \colon [a,b] \to \bC$
  is the path defined by
  \begin{align*}
    \opp{\gamma}(t) = \gamma(a+b-t) \qquad \text{ for } t \in [a,b] .
  \end{align*}
\end{definition}

\begin{definition}[Concatenation of paths \Palka{Sec.~IV.1.4}]
  \label{def:path_concatenation}
  \uses{def:path}
  %\lean{}
  % \leanok
  Given path $\gamma \colon [a,b] \to \bC$ and
  $\eta \colon [c,d] \to \bC$ with $\gamma(b) = \eta(c)$
  (the starting point of $\eta$ coincides with the end point of $\gamma$),
  the \term{concatenation} of $\gamma$ and $\eta$ is the path
  $\gamma \concat \eta \colon [0,(b-a)+(d-c)] \to \bC$ defined by
  \begin{align*}
    (\gamma \concat \eta)(t) =
    \begin{cases}
      \gamma(a+t) & \text{ for } t \in [0,b-a] , \\
      \eta(c+t-(b-a)) & \text{ for } t \in [b-a,(b-a)+(d-c)] .
    \end{cases}
  \end{align*}
\end{definition}

\begin{definition}[Reparametrization of paths \Palka{Sec.~IV.1.5}]
  \label{def:path_reparametrization}
  \uses{def:path, def:compact}
  %\lean{}
  % \leanok
  Given a path $\gamma \colon [a,b] \to \bC$ and a continuous increasing
  bijection $\phi \colon [c,d] \to [a,b]$, we define the \term{reparametrization}
  of $\gamma$ by $\phi$ as the path
  \begin{align*}
    \gamma \circ \phi \colon \, [c,d] \, \to \; & \bC
    t \, \mapsto \; & \gamma(\phi(t)) .
  \end{align*}

  Note that
  \begin{itemize}
    \item $\phi^{-1} \colon [a,b] \to [c,d]$ is also a continuous increasing
      bijection (a continuous bijection from the
      compact $[a,b]$ is automatically a homeomorphism; see ) and
      reparametrization can be undone by rereparametrizing by $\phi^{-1}$;
    \item X
  \end{itemize}
\end{definition}

\begin{definition}[Smooth and piecewise smooth reparametrizations \Palka{Sec.~IV.1.5}]
  \label{def:piecewise_smooth_reparametrization}
  \uses{def:path_reparametrization}
  %\lean{}
  % \leanok
  \ldots
\end{definition}



\section{Integrals along paths}

\begin{definition}[Contour integral along a smooth path \Palka{Sec.~IV.2.1}]
  \label{def:smooth_contour_integral}
  \uses{def:riemann_integral, lem:continuous_implies_riemann_integrable, def:piecewise_smooth_path}
  %\lean{}
  % \leanok
  Let $f : A \to \bC$ be a continuous function defined on a subset $A \subset \bC$.
  %and write $u = \re f, v = \im f \colon [a,b] \to \bR$
  %for its real and imaginary parts so that $f(z) = u(z) + \ii v(z)$ for $z \in A$.
  Let $\gamma \colon [a,b] \to A$ be a smooth path in~$A$.
  We define the \term{integral of $f$ along $\gamma$} as
  \begin{align*}
    \int_\gamma f(z) \, dz
    \; = \; \int_a^b f \big( \gamma(t) \big) \, \gamma'(t) \; \ud t .
    %= \; \int_a^b u \big( \gamma(t) \big) \gamma'(t) \, \ud t
    %     + \ii \int_a^b v \big( \gamma(t) \big) \gamma'(t) \, \ud t .
  \end{align*}
\end{definition}

Sometimes it is appropriate to integrate functions with respect to the
arc-length in the following sense.
\begin{definition}[Arc-length integral along a smooth path \Palka{Sec.~IV.2.1}]
  \label{def:smooth_arc_length_integral}
  \uses{def:riemann_integral, lem:continuous_implies_riemann_integrable, def:piecewise_smooth_path}
  %\lean{}
  % \leanok
  Let $f : A \to \bC$ be a continuous function defined on a subset $A \subset \bC$.
  %and write $u = \re f, v = \im f \colon [a,b] \to \bR$
  %for its real and imaginary parts so that $f(z) = u(z) + \ii v(z)$ for $z \in A$.
  Let $\gamma \colon [a,b] \to A$ be a smooth path in~$A$.
  We define the \term{integral of $f$ with respect to the arc length of $\gamma$} as
  \begin{align*}
    \int_\gamma f(z) \, |dz|
    \; = \; \int_a^b f \big( \gamma(t) \big) \, |\gamma'(t)| \; \ud t .
    %= \; \int_a^b u \big( \gamma(t) \big) \gamma'(t) \, \ud t
    %     + \ii \int_a^b v \big( \gamma(t) \big) \gamma'(t) \, \ud t .
  \end{align*}
\end{definition}

In order to extend the definition of contour integrals to piecewise smooth paths,
we note that the definition behaves additively under path concatenation.

\begin{lemma}[Contour integrals and smooth path concatenation \Palka{Lem~IV.2.1(iv)}]
  \label{lem:integral_concatenation}
  \uses{def:smooth_contour_integral, def:smooth_arc_length_integral, def:path_concatenation}
  %\lean{}
  % \leanok
  If a smooth path $\gamma$ in $A$ is a concatenation of smooth
  paths $\eta_1, \ldots, \eta_n$, and
  $f : A \to \bC$ is a continuous function defined on $A \subset \bC$,
  then we have
  \begin{align*}
    \int_\gamma f(z) \, \ud z \; = \; \sum_{j=1}^n \int_{\eta_j} f(z) \, \ud z
  \end{align*}
  and
  \begin{align*}
    \int_\gamma f(z) \, |\ud z| \; = \; \sum_{j=1}^n \int_{\eta_j} f(z) \, |\ud z| .
  \end{align*}
\end{lemma}
\begin{proof}
  %\uses{}
  % \leanok
  %Straightforward from a corresponding property of Riemann integrals.
  \ldots
\end{proof}

By virtue of the above, the following gives a well-defined meaning
to integrals along piecewise smooth paths.
\begin{definition}[Contour integral along a path \Palka{Sec.~IV.2.1}]
  \label{def:contour_integral}
  \uses{def:smooth_contour_integral, def:piecewise_smooth_path, lem:integral_concatenation}
  %\lean{}
  % \leanok
  Let $f : A \to \bC$ be a continuous function defined on a subset $A \subset \bC$.
  Let $\gamma \colon [a,b] \to A$ be a piecewise smooth path in~$A$,
  which is a concatenation of smooth paths $\eta_1, \ldots, \eta_n$.
  We define the \term{integral of $f$ along $\gamma$} as
  \begin{align*}
    \int_\gamma f(z) \, dz \; = \; \sum_{j=1}^n \int_{\eta_j} f(z) \, dz .
  \end{align*}
\end{definition}
\begin{definition}[Arc-length integral along a path]
  \label{def:arc_length_integral}
  \uses{def:smooth_arc_length_integral, def:piecewise_smooth_path, lem:integral_concatenation}
  %\lean{}
  % \leanok
  Let $f : A \to \bC$ be a continuous function defined on a subset $A \subset \bC$.
  Let $\gamma \colon [a,b] \to A$ be a piecewise smooth path in~$A$,
  which is a concatenation of smooth paths $\eta_1, \ldots, \eta_n$.
  We define the \term{integral of $f$ along $\gamma$} as
  \begin{align*}
    \int_\gamma f(z) \, dz \; = \; \sum_{j=1}^n \int_{\eta_j} f(z) \, dz .
  \end{align*}
\end{definition}

\begin{lemma}[Reparametrization invariance of integrals \Palka{Lem~IV.2.1(v)}]
  \label{lem:reparametrization_invariance_of_integral}
  \uses{def:contour_integral, def:arc_length_integral, def:piecewise_smooth_reparametrization}
  %\lean{}
  % \leanok
  Let $\gamma$ be a piecewise smooth path in~$A$,
  and let $\widetilde{\gamma}$ be obtained from $\gamma$ by an
  orientation-preserving reparametrization. Then for any continuous
  function $f : A \to \bC$ we have
  \begin{align*}
    \int_{\widetilde{\gamma}} f(z) \, dz \; = \; \int_{\gamma} f(z) \, dz
  \end{align*}
  and
  \begin{align*}
    \int_{\widetilde{\gamma}} f(z) \, |dz| \; = \; \int_{\gamma} f(z) \, |dz| .
  \end{align*}
\end{lemma}

\begin{lemma}[Contour integrals and path reversal \Palka{Lem~IV.2.1(iii)}]
  \label{lem:integral_reversal}
  \uses{def:contour_integral, def:arc_length_integral, def:reverse_path}
  %\lean{}
  % \leanok
  If $f : A \to \bC$ is a continuous function defined on $A \subset \bC$,
  and $\gamma$ is a piecewise path in $A$, then for the contour integral
  and the arc length integral behave as follows under path reversal:
  then we have
  \begin{align*}
    \int_{\opp{\gamma}} f(z) \, \ud z \; = \; - \int_{\gamma} f(z) \, \ud z
  \end{align*}
  and
  \begin{align*}
    \int_{\opp{\gamma}} f(z) \, |\ud z| \; = \; \int_{\gamma} f(z) \, |\ud z|
  \end{align*}
\end{lemma}
\begin{proof}
  %\uses{}
  % \leanok
  %Straightforward from a corresponding property of Riemann integrals.
  \ldots
\end{proof}

\begin{lemma}[Linearity of integrals \Palka{Lem~IV.2.1(i-ii)}]
  \label{lem:integral_linearity}
  \uses{def:contour_integral, def:arc_length_integral, def:linear}
  %\lean{}
  % \leanok
  \begin{align*}
    \int_\gamma \Big( f_1(z) + f_2(z) \Big) \, \ud z
    \; = \; \int_\gamma f_1(z) \, \ud z + \int_\gamma f_2(z) \, \ud z
  \end{align*}
  \begin{align*}
    \int_\gamma c \, f(z) \, \ud z \; = \; c \int_\gamma f(z) \, \ud z .
  \end{align*}
\end{lemma}
\begin{proof}
  %\uses{}
  % \leanok
  \ldots
\end{proof}

\begin{lemma}[Triangle inequality for contour integrals \Palka{Lem~IV.2.1(vi)}]
  \label{lem:integral_triangle_inequality}
  \uses{def:contour_integral, def:arc_length_integral}
  %\lean{}
  % \leanok
  \begin{align*}
    \left| \int_\gamma f(z) \, \ud z \right| \; \le \; \int_\gamma |f(z)| \, |dz| .
  \end{align*}
\end{lemma}
\begin{proof}
  %\uses{}
  % \leanok
  \ldots
\end{proof}



\section{Primitives}

\begin{definition}[Primitive of a function \Palka{Sec.~IV.2.3}]
  \label{def:primitive}
  \uses{def:analytic_function}
  %\lean{}
  % \leanok
  Let $f : U \to \bC$ be a function defined on an open subset $U \subset \bC$.
  A \term{primitive} of $f$ is a function $F : U \to \bC$ such that
  $F$ is holomorphic (i.e., complex differentiable) on $U$,
  \begin{align*}
    F'(z) = f(z) \qquad \text{ for all } z \in U .
  \end{align*}
\end{definition}

\begin{theorem}[Fundamental theorem of calculus for contour integrals \Palka{Thm~IV.2.2}]
  \label{thm:ftc_for_contour_integrals}
  \uses{def:primitive, def:contour_integral}
  %\lean{}
  % \leanok
  Suppose that $f \colon U \to \bC$ is a continuous function
  on an open set $U \subset \bC$, and that $f$ has a
  primitive $F \colon U \to \bC$.
  Then for any piecewise smooth path $\gamma \colon [a,b] \to U$ we have
  \begin{align*}
    \int_\gamma f(z) \, dz \; = \; F \big( \gamma(b) \big) - F \big( \gamma(a) \big) .
  \end{align*}
  In particular for any closed path $\gamma$ in $U$, we have
  \begin{align*}
    \oint_\gamma f(z) \, dz \; = \; 0 .
  \end{align*}
\end{theorem}
