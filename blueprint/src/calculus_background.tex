\section{Riemann integral}

For the purposes of this course, it suffices to know the Riemann
integral. (Those who already know Lebesgue integration theory can
substitute that more general notion of integral everywhere.)

\begin{definition}[Riemann integral]
  \label{def:riemann_integral}
  %\uses{}
  %\lean{}
  % \leanok
  \ldots
\end{definition}

\begin{lemma}[Riemann integrability of continuous functions]
  \label{lem:continuous_implies_riemann_integrable}
  \uses{def:riemann_integral}
  %\lean{}
  % \leanok
  Any continuous function $f \colon [a,b] \to \bR$ is Riemann
  integrable on $[a,b]$.
\end{lemma}
\begin{proof}
  \uses{lem:on_compact_continuous_implies_uniformly_continuous,
  thm:bolzano_weierstrass, thm:continuous_on_compact_implies_bounded}
  % \leanok
  See \textbf{MS-C1541 Metric Spaces}.
\end{proof}



\section{Trigonometry}

\begin{lemma}[Trigonometric angle sum identities]
  \label{lem:trigonometric_angle_sum}
  %\uses{}
  % \lean{}
  % \leanok
  Let $\alpha, \beta \in \bR$. Then we have
  \begin{align*}
    \cos (\alpha + \beta) \; = \; \cos (\alpha) \, \cos (\beta) - \sin (\alpha) \, \sin (\beta) \\
    \sin (\alpha + \beta) \; = \; \cos (\alpha) \, \sin (\beta) + \sin (\alpha) \, \cos (\beta) .
  \end{align*}
\end{lemma}
\begin{proof}
  % \uses{}
  % \leanok
  \ldots
\end{proof}
