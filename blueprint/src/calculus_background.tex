\section{Differentiability}

\begin{definition}[Real differentiability]
  \label{def:differential}
  \uses{def:limit, def:linear, def:interior}
  %\lean{}
  % \leanok
  Let $m,n \in \bN$, and
  let $f \colon U \to \bR^m$ be a function
  defined on a subset $U \subset \bR^n$.
  A linear map $L \colon \bR^n \to \bR^m$
  is said to be a \term{differential} of $f$
  at $p_0 \in U$ if
  \begin{align*}
    f(p) = f(p_0) + L (p-p_0) + E(p-p_0)
  \end{align*}
  where the error term $E$ is small near $p_0$ in the sense that
  \begin{align*}
    \lim_{p \to p_0} \frac{\| E(p-p_0) \|}{\|p-p_0\|} = 0 .
  \end{align*}
  We say that $f$ is \term{differentiable} at $p_0$ if such
  a linear map $L$ exists.

  It is easy to check that the differential $L$ of $f$ at $p_0$
  is unique if $p_0$ is an interior point of $U$;
  we then denote it by $L = \ud f (p_0)$.
\end{definition}

\begin{lemma}[Differentiability implies continuity]
  \label{lem:differentiable_implies_continuous}
  \uses{def:differential, def:continuous}
  % \lean{}
  % \leanok
  If a function $f \colon U \to \bR^m$
  defined on a subset $U \subset \bR^n$ is differentiable
  at $p_0 \in U$, then it is continuous at $p_0$.
\end{lemma}
\begin{proof}
  %\uses{}
  % \leanok
  \ldots
\end{proof}

\begin{lemma}[Jacobian matrix of the differential]
  \label{lem:jacobian_matrix}
  \uses{def:differential}
  % \lean{}
  % \leanok
  If a function $f \colon U \to \bR^m$
  defined on a subset $U \subset \bR^n$ is differentiable
  at an interior point $p_0$ of $U$, then it has
  all first order partial derivatives at $p_0$, and the
  matrix representation of the differential $\ud f (p_0)$
  in the standard bases of $\bR^m$ and $\bR^n$ is
  \begin{align*}
    \ud f (p_0)
      %= \Big[ \frac{\partial f}{\partial x_1}(p_0) \; \cdots
      %          \; \frac{\partial f}{\partial x_n}(p_0) \Big]
      = \left[ \begin{array}{ccc}
        \frac{\partial f_1}{\partial x_1}(p_0) & \cdots & \frac{\partial f_1}{\partial x_n}(p_0) \\
        \vdots & \ddots & \vdots \\
        \frac{\partial f_m}{\partial x_1}(p_0) & \cdots & \frac{\partial f_m}{\partial x_n}(p_0)
        \end{array} \right]
      \in \bR^{m \times n} ,
  \end{align*}
  where $f_1, \ldots, f_m \colon U \to \bR$ denote component functions of $f$.
\end{lemma}
\begin{proof}
  %\uses{}
  % \leanok
  \ldots
\end{proof}

\begin{lemma}[Vanishing partial derivatives implies locally constant]
  \label{lem:vanishing_partial_derivatives}
  \uses{def:differential, def:open_set, def:connected}
  % \lean{}
  % \leanok
  Suppose that $f \colon U \to \bR^m$ is a function
  defined on an open and connected subset $U \subset \bR^n$ of $\bR^n$
  whose first order partial derivatives exist and are zero at all points
  of $U$. Then $f$ is a constant function.
\end{lemma}
\begin{proof}
  \uses{lem:open_connected_iff_path_connected}
  % \leanok
  \ldots
\end{proof}


\section{Riemann integral}

For the purposes of this course, it suffices to know the Riemann
integral. (Those who already know Lebesgue integration theory can
substitute that more general notion of integral everywhere.)

\begin{definition}[Riemann integral]
  \label{def:riemann_integral}
  %\uses{}
  %\lean{}
  % \leanok
  \ldots
\end{definition}

\begin{lemma}[Riemann integrability of continuous functions]
  \label{lem:continuous_implies_riemann_integrable}
  \uses{def:riemann_integral}
  %\lean{}
  % \leanok
  Any continuous function $f \colon [a,b] \to \bR$ is Riemann
  integrable on $[a,b]$.
\end{lemma}
\begin{proof}
  \uses{lem:on_compact_continuous_implies_uniformly_continuous,
  thm:bolzano_weierstrass, thm:continuous_on_compact_implies_bounded}
  % \leanok
  See \textbf{MS-C1541 Metric Spaces}.
\end{proof}



\section{Trigonometry}

\begin{lemma}[Trigonometric angle sum identities]
  \label{lem:trigonometric_angle_sum}
  %\uses{}
  % \lean{}
  % \leanok
  Let $\alpha, \beta \in \bR$. Then we have
  \begin{align*}
    \cos (\alpha + \beta) \; = \; \cos (\alpha) \, \cos (\beta) - \sin (\alpha) \, \sin (\beta) \\
    \sin (\alpha + \beta) \; = \; \cos (\alpha) \, \sin (\beta) + \sin (\alpha) \, \cos (\beta) .
  \end{align*}
\end{lemma}
\begin{proof}
  % \uses{}
  % \leanok
  \ldots
\end{proof}
