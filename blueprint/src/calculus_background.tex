\section{Differentiability}

\begin{definition}[Real differentiability]
  \label{def:differential}
  \uses{def:limit, def:linear, def:interior}
  %\lean{}
  % \leanok
  Let $m,n \in \bN$, and
  let $f \colon U \to \bR^m$ be a function
  defined on a subset $U \subset \bR^n$.
  A linear map $L \colon \bR^n \to \bR^m$
  is said to be a \term{differential} of $f$
  at $p_0 \in U$ if
  \begin{align*}
    f(p) = f(p_0) + L (p-p_0) + E(p-p_0)
  \end{align*}
  where the error term $E$ is small near $p_0$ in the sense that
  \begin{align*}
    \lim_{p \to p_0} \frac{\| E(p-p_0) \|}{\|p-p_0\|} = 0 .
  \end{align*}
  We say that $f$ is \term{differentiable} at $p_0$ if such
  a linear map $L$ exists.

  It is easy to check that the differential $L$ of $f$ at $p_0$
  is unique if $p_0$ is an interior point of $U$;
  we then denote it by $L = \ud f (p_0)$.
\end{definition}

\begin{lemma}[Differentiability implies continuity]
  \label{lem:differentiable_implies_continuous}
  \uses{def:differential, def:continuous}
  % \lean{}
  % \leanok
  If a function $f \colon U \to \bR^m$
  defined on a subset $U \subset \bR^n$ is differentiable
  at $p_0 \in U$, then it is continuous at $p_0$.
\end{lemma}
\begin{proof}
  %\uses{}
  % \leanok
  \ldots
\end{proof}

\begin{lemma}[Jacobian matrix of the differential]
  \label{lem:jacobian_matrix}
  \uses{def:differential}
  % \lean{}
  % \leanok
  If a function $f \colon U \to \bR^m$
  defined on a subset $U \subset \bR^n$ is differentiable
  at an interior point $p_0$ of $U$, then it has
  all first order partial derivatives at $p_0$, and the
  matrix representation of the differential $\ud f (p_0)$
  in the standard bases of $\bR^m$ and $\bR^n$ is
  \begin{align*}
    \ud f (p_0)
      %= \Big[ \frac{\partial f}{\partial x_1}(p_0) \; \cdots
      %          \; \frac{\partial f}{\partial x_n}(p_0) \Big]
      = \left[ \begin{array}{ccc}
        \frac{\partial f_1}{\partial x_1}(p_0) & \cdots & \frac{\partial f_1}{\partial x_n}(p_0) \\
        \vdots & \ddots & \vdots \\
        \frac{\partial f_m}{\partial x_1}(p_0) & \cdots & \frac{\partial f_m}{\partial x_n}(p_0)
        \end{array} \right]
      \in \bR^{m \times n} ,
  \end{align*}
  where $f_1, \ldots, f_m \colon U \to \bR$ denote component functions of $f$.
\end{lemma}
\begin{proof}
  %\uses{}
  % \leanok
  \ldots
\end{proof}

\begin{lemma}[Vanishing partial derivatives implies locally constant]
  \label{lem:vanishing_partial_derivatives}
  \uses{def:differential, def:open_set, def:connected}
  % \lean{}
  % \leanok
  Suppose that $f \colon U \to \bR^m$ is a function
  defined on an open and connected subset $U \subset \bR^n$ of $\bR^n$
  whose first order partial derivatives exist and are zero at all points
  of $U$. Then $f$ is a constant function.
\end{lemma}
\begin{proof}
  \uses{lem:open_connected_iff_path_connected}
  % \leanok
  \ldots
\end{proof}


\section{Riemann integral}

For the purposes of this course, it suffices to know the Riemann
integral. (Those who already know Lebesgue integration theory can
substitute that more general notion of integral everywhere.)

\begin{definition}[Riemann integral]
  \label{def:riemann_integral}
  %\uses{}
  %\lean{}
  % \leanok
  \ldots
\end{definition}

\begin{lemma}[Riemann integrability of continuous functions]
  \label{lem:continuous_implies_riemann_integrable}
  \uses{def:riemann_integral}
  %\lean{}
  % \leanok
  Any continuous function $f \colon [a,b] \to \bR$ is Riemann
  integrable on $[a,b]$.
\end{lemma}
\begin{proof}
  \uses{lem:on_compact_continuous_implies_uniformly_continuous,
  thm:bolzano_weierstrass, thm:continuous_on_compact_implies_bounded}
  % \leanok
  See \MetSpCourse{}.
\end{proof}



\section{Trigonometry}

\begin{lemma}[Trigonometric angle sum identities]
  \label{lem:trigonometric_angle_sum}
  %\uses{}
  % \lean{}
  % \leanok
  Let $\alpha, \beta \in \bR$. Then we have
  \begin{align*}
    \cos (\alpha + \beta) \; = \; \cos (\alpha) \, \cos (\beta) - \sin (\alpha) \, \sin (\beta) \\
    \sin (\alpha + \beta) \; = \; \cos (\alpha) \, \sin (\beta) + \sin (\alpha) \, \cos (\beta) .
  \end{align*}
\end{lemma}
\begin{proof}
  % \uses{}
  % \leanok
  \ldots
\end{proof}



\section{Supremum, infimum, limit superior, and limit inferior}

\begin{definition}[Supremum]
  \label{def:supremum}
  %\uses{}
  % \lean{}
  % \leanok
  The \term{supremum}, or the \term{least upper bound}, of a set $A \subset \bR$
  is the smallest real number $s$ such that $a \le s$ for all $a \in A$,
  and is denoted by $s = \sup A$.

  By the completeness axiom of real numbers,
  every nonempty set ($A \ne \emptyset$) of real numbers which is bounded
  from above (for some $u \in \bR$ we have $a \le u$ for all $a \in A$)
  has a supremum $\sup A \in \bR$. We adopt the notational conventions
  that $\sup \emptyset = -\infty$, and that $\sup A = +\infty$ if $A$ is not
  bounded from above.

  For convenience, we also adopt some flexibility in the
  notation:
  % to avoid the need to literally write down the set
  % of real numbers when this is merely cumbersome.
  for example the supremum of values of a real-valued
  function on a set $D$ is denoted by
  \begin{align*}
    \sup_{x \in D} f(x) \; := \; \sup \big\{ f(x) \; \big| \; x \in D \big\}
  \end{align*}
  and the supremum of values in the tail of a real-number
  sequence $(x_n)$ starting from index $m$ is denoted by
  \begin{align*}
    \sup_{n \ge m} x_n \; := \; \sup \big\{ x_n \; \big| \; m \ge n \big\} \, .
  \end{align*}
\end{definition}

\begin{definition}[Infimum]
  \label{def:infimum}
  %\uses{}
  % \lean{}
  % \leanok
  The \term{infimum}, or the \term{greatest lower bound}, of a set $A \subset \bR$
  is the greatest real number $i$ such that $a \ge i$ for all $a \in A$,
  and is denoted by $i = \inf A$.

  By the completeness axiom of real numbers,
  every nonempty set ($A \ne \emptyset$) of real numbers which is bounded
  from below (for some $\ell \in \bR$ we have $a \ge \ell$ for all $a \in A$)
  has an infimum $\inf A \in \bR$. We adopt the notational conventions
  that $\inf \emptyset = +\infty$, and that $\inf A = -\infty$ if $A$ is not
  bounded from below.

  For convenience, we also adopt some flexibility in the
  notation for infimums of function values or sequence
  values, similarly as with supremums.
\end{definition}

\begin{definition}[Limit superior]
  \label{def:limsup}
  \uses{def:supremum}
  % \lean{}
  % \leanok
  Let $(x_n)_{n \in \bN}$ be a sequence of real numbers.
  Then the \term{limit superior} of the sequence is defined as
  \begin{align*}
    \limsup_{n \to \infty} x_n \; := \; \lim_{m \to \infty} \Big( \sup_{n \ge m} x_n \Big) .
  \end{align*}

  With the following conventions, the limit superior of a sequence always exists
  as either a real number or one of the symbols $\pm\infty$.
  If the sequence is not bounded from above, then by
  conventions regarding the supremum, we have
  $\sup_{n \ge m} x_n = +\infty$ for every $m$, so we correspondingly
  set $\limsup_{n \to \infty} x_n = +\infty$.
  Otherwise the sequence $(\sup_{n \ge m} x_n)_{m \in \bN}$
  is a decreasing sequence of real numbers, so either it is
  bounded from below and converges to
  $\lim_{m \to \infty} \big( \sup_{n \ge m} x_n \big) \, = \,
  \inf_{m \in \bN} \big( \sup_{n \ge m} x_n \big) \in \bR$,
  or it is not bounded from below and we set
  $\limsup_{n \to \infty} x_n \, = \,
  \inf_{m \in \bN} \big( \sup_{n \ge m} x_n \big) = -\infty$.
\end{definition}

\begin{definition}[Limit inferior]
  \label{def:liminf}
  \uses{def:infimum}
  % \lean{}
  % \leanok
  Let $(x_n)_{n \in \bN}$ be a sequence of real numbers.
  Then the \term{limit inferior} of the sequence is defined as
  \begin{align*}
    \liminf_{n \to \infty} x_n \; := \; \lim_{m \to \infty} \Big( \inf_{n \ge m} x_n \Big) .
  \end{align*}

  With the following conventions, the limit inferior of a sequence always exists
  as either a real number or one of the symbols $\pm \infty$.
  If the sequence is not bounded from below, then by
  conventions regarding the infimum, we have
  $\inf_{n \ge m} x_n = -\infty$ for every $m$, so we correspondingly
  set $\liminf_{n \to \infty} x_n = -\infty$.
  Otherwise the sequence $(\inf_{n \ge m} x_n)_{m \in \bN}$
  is an increasing sequence of real numbers, so either it is
  bounded from above and converges to
  $\lim_{m \to \infty} \big( \inf_{n \ge m} x_n \big) \, = \,
  \sup_{m \in \bN} \big( \inf_{n \ge m} x_n \big) \in \bR$,
  or it is not bounded from above and we set
  $\liminf_{n \to \infty} x_n \, = \,
  \sup_{m \in \bN} \big( \inf_{n \ge m} x_n \big) = +\infty$.
\end{definition}

\begin{lemma}[Limit with limsup and liminf]
  \label{lem:limit_with_limsup_liminf}
  \uses{def:limsup, def:liminf, def:limit}
  % \lean{}
  % \leanok
  Let $(x_n)_{n \in \bN}$ be a sequence of real numbers, and let $x \in \bR$.
  Then the following are equivalent:
  \begin{itemize}
    \item The limit $\lim_{n \to \infty} x_n$ exists and equals $x$.
    \item We have both $\limsup_{n \to \infty} x_n = x$ and
      $\liminf_{n \to \infty} x_n = x$.
  \end{itemize}

  (With the usual conventions of $\pm\infty$ as possible limits of real-number sequences,
  the above equivalence of conditions also extends to the cases $x = \pm \infty$.)
\end{lemma}
\begin{proof}
  % \uses{}
  % \leanok
  \ldots
\end{proof}
