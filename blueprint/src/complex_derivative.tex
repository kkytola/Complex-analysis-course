\section{Real linear maps versus complex linear maps}

The right abstract way of understanding the differential of a function is as a linear
approximation to a function locally. The key difference between real analysis and
complex analysis is whether one uses real linear maps or complex linear maps.

\begin{definition}[Linear map]
  \label{def:linear}
  \uses{}
  %\lean{}
  % \leanok
  Let $\bK$ be a field (for example $\bK = \bR$ or $\bK = \bC$), and let
  $V$ and $W$ be vector spaces over $\bK$. A function $L : V \to W$ is
  said to be \term{$\bK$-linear} if
  \begin{align*}
    L(v_1 + v_2) = \; & L(v_1) + L(v_2) & & \text{ for all } v_1, v_2 \in V, \\
    L(c v) = \; & c \, L(v) & & \text{ for all } v \in V, c \in \bK .
  \end{align*}
  Such a function $L$ is also called a \term{$\bK$-linear map} (or a
  \term{$\bK$-linear transformation}) between the spaces $V$ and $W$.
\end{definition}

The complex plane $\bC \cong \bR^2$ can be seen either as a 2-dimensional real vector space
or as a $1$-dimensional complex vector space. In particular, it makes sense to talk about
both $\bR$-linear maps $\bC \to \bC$ and $\bC$-linear maps $\bC \to \bC$.

More generally, any complex vector space can be seen as a real vector space
(of twice the same dimension), and any complex linear map becomes a real linear map.
The converse is not true! Let us elaborate on this in a simple example which
will soon be seen to pertain to the difference of complex differentiability
and real differentiability.

\textbf{Remark:} Identifying $\bC = \bR^2$ (and choosing basis vectors $1, \ii \in \bC$
for $\bC$ seen as a 2-dimensional vector space), we see that an
$\bR$-linear map $L : \bC \to \bC$ can be encoded in a $2 \times 2$ matrix with real entries,
\begin{align*}
  M = \left[ \begin{array}{cc} a & b \\ c & d \end{array} \right] \in \bR^{2 \times 2}
\end{align*}
in such a way that
\begin{align*}
  L (x + \ii y) = (a x + b y) + \ii (c x + d y) .
\end{align*}

\textbf{Remark:} A $\bC$-linear map $L : \bC \to \bC$ can be encoded in a single complex number
$\lambda \in \bC$ (or more pedantically, in a $1 \times 1$ matrix $[\lambda] \in \bC^{1 \times 1}$),
in such a way that
\begin{align*}
  L z = \lambda z .
\end{align*}
We can write $\lambda = \alpha + \ii \beta$, with
$\alpha = \re(\lambda), \beta = \im(\lambda) \in \bR$. We can also write $z = x + \ii y$
and obtain the expression
\begin{align*}
  L (x + \ii y) = (\alpha + \ii \beta) (x + \ii y)
    = (\alpha x - \beta y) + \ii (\beta x + \alpha y) .
\end{align*}
In other words, seen as a real-linear map, the complex multiplication by $\lambda$
corresponds to the matrix
\begin{align*}
  M = \left[ \begin{array}{cc} \alpha & -\beta \\ \beta & \alpha \end{array} \right] .
\end{align*}
This clearly shows that not every real-linear map $\bC \to \bC$ is complex linear.
It also gives an explicit set of equations for the entries of
the matrix of a real-linear map characterizing complex-linearity, which turn out to be
very closely related to the Cauchy-Riemann equations.

\begin{lemma}[Complex linear versus real linear maps of $\bC$]
  \label{lem:complex_linear_vs_real_linear}
  \uses{def:linear, thm:complex_number_field}
  %\lean{}
  % \leanok
  Let $L : \bC \to \bC$ be a $\bR$-linear map represented in the basis $1,\ii$ by
  the matrix $M = \left[ \begin{array}{cc} a & b \\ c & d \end{array} \right] \in \bR^{2 \times 2}$.
  Then the following are equivalent:
  \begin{itemize}
    \item $L$ is $\bC$-linear;
    \item $b = - c$ and $a = d$.
  \end{itemize}
\end{lemma}
\begin{proof}
  %\uses{}
  % \leanok
  Clear from the above discussion.
\end{proof}



\section{Complex derivative}

\begin{definition}[Complex derivative \Palka{Sec.~III.1.1}]
  \label{def:complex_derivative}
  \uses{thm:complex_number_field, def:limit}
  %\lean{}
  % \leanok
  A function $f \colon A \to \bC$ is said to be
  \term{complex differentiable} at an interior
  point $z_0 \in A \subset \bC$ of its domain of definition
  if the limit
  \begin{align*}
    f'(z_0) := \lim_{z \to z_0} \frac{f(z) - f(z_0)}{z - z_0}
  \end{align*}
  exists. The value $f'(z_0) \in \bC$ is then called the
  \term{(complex) derivative} of $f$ at $z_0$.
\end{definition}

\begin{lemma}[Local linear approximation]
  \label{lem:local_linear_approximation}
  \uses{def:complex_derivative}
  %\lean{}
  % \leanok
  If a function $f \colon A \to \bC$ has complex derivative
  $f'(z_0) = \lambda \in \bC$ at a point $z_0 \in A$, then
  we can write a linear approximation
  \begin{align*}
    f(z) = f(z_0) + (z - z_0) \, \lambda + \epsilon(z) ,
  \end{align*}
  where the error term $\epsilon$ is small near $z_0$ in the sense that
  $\lim_{z \to z_0} \frac{\epsilon(z)}{|z-z_0|} = 0$.
\end{lemma}
\begin{proof}
  %\uses{}
  % \leanok
  \ldots
\end{proof}

\begin{lemma}[Complex differentiability implies continuity \Palka{Sec.~III.1.1}]
  \label{lem:complex_differentiable_implies_continuous}
  \uses{def:continuous, def:complex_derivative}
  %\lean{}
  % \leanok
  If a function $f \colon A \to \bC$ has a complex derivative $f'(z_0)$
  at a point $z_0 \in A$, then it is continuous at $z_0$.
\end{lemma}
\begin{proof}
  \uses{lem:differentiable_implies_continuous}
  % \leanok
  \ldots
\end{proof}


\subsection{Differentiation rules}

\begin{lemma}[Linearity of the derivative \Palka{Sec.~III.1.2~(3.4)}]
  \label{lem:derivative_linearity}
  \uses{def:complex_derivative, def:linear}
  %\lean{}
  % \leanok
  If two functions $f, g \colon A \to \bC$ have complex derivatives
  $f'(z_0), g'(z_0)$ at a point $z_0 \in A$, then the sum function $f + g$
  has a complex derivative at $z_0$ given by
  \begin{align*}
    (f + g)'(z_0) = f'(z_0) + g'(z_0) .
  \end{align*}
  If a function $f \colon A \to \bC$ is has a complex derivative
  $f'(z_0)$ at a point $z_0 \in A$ and $c \in \bC$ is a complex number,
  then the function $c f$ has complex derivative
  \begin{align*}
    (c f)'(z_0) = c \, f'(z_0)
  \end{align*}
  at $z_0$.
\end{lemma}
\begin{proof}
  \uses{lem:complex_limit_properties}
  % \leanok
  \ldots
\end{proof}

\begin{lemma}[Leibniz rule \Palka{Sec.~III.1.2~(3.4)}]
  \label{lem:leibniz_rule}
  \uses{def:complex_derivative}
  %\lean{}
  % \leanok
  If two functions $f, g \colon A \to \bC$ have complex derivatives
  $f'(z_0), g'(z_0)$ at a point $z_0 \in A$, then the product function $f g$
  has complex derivative
  \begin{align*}
    (f g)'(z_0) = f'(z_0) \, g(z_0) + f(z_0) \, g'(z_0)
  \end{align*}
  at $z_0$.
\end{lemma}
\begin{proof}
  %\uses{}
  % \leanok
  \ldots
\end{proof}

\begin{lemma}[Derivative of a quotient \Palka{Sec.~III.1.2~(3.4)}]
  \label{lem:quotient_derivative}
  \uses{def:complex_derivative}
  %\lean{}
  % \leanok
  If two functions $f, g \colon A \to \bC$ have complex derivatives
  $f'(z_0), g'(z_0)$ at a point $z_0 \in A$ and $g(z_0) \ne 0$, then
  the quotient function $f / g$ has complex derivative
  \begin{align*}
    \left( \frac{f}{g} \right)'(z_0)
      = \frac{f'(z_0) \, g(z_0) - f(z_0) \, g'(z_0)}{g(z_0)^2} .
  \end{align*}
  at $z_0$.
\end{lemma}
\begin{proof}
  %\uses{}
  % \leanok
  \ldots
\end{proof}

\begin{lemma}[Chain rule \Palka{Thm~III.1.1}]
  \label{lem:chain_rule}
  \uses{def:complex_derivative}
  %\lean{}
  % \leanok
  If $f \colon A \to B \subset \bC$ is differentiable at $z_0 \in A$
  and $g \colon B \to \bC$ is differentiable at $f(z_0) \in B$,
  then the composition $g \circ f \colon A \to \bC$ is differentiable at $z_0$,
  with derivative
  \begin{align*}
    (g \circ f)'(z_0) = f'(z_0) \; g'\big( f(z_0) \big) .
  \end{align*}
\end{lemma}
\begin{proof}
  %\uses{}
  % \leanok
  \ldots
\end{proof}

\begin{lemma}[Derivative of inverse \Palka{Thm~III.4.1}]
  \label{lem:inverse_derivative}
  \uses{def:complex_derivative}
  %\lean{}
  % \leanok
  \ldots (with $w_0 = f(z_0)$)
  \begin{align*}
    (f^{-1})'(w_0) = \frac{1}{f'(z_0)} .
  \end{align*}
\end{lemma}
\begin{proof}
  \uses{lem:chain_rule}
  % \leanok
  \ldots
\end{proof}


\subsection{Analytic functions}

\begin{definition}[Analytic function \Palka{Sec.~III.1.3}]
  \label{def:analytic_function}
  \uses{def:complex_derivative, def:open_set}
  %\lean{}
  % \leanok
  A function $f \colon U \to \bC$ defined on an open set $U \subset \bC$ is
  said to be \term{analytic} (or \term{holomorphic}) if it is complex
  differentiable at every point $z_0 \in U$.
\end{definition}

\begin{lemma}[Polynomials are analytic]
  \label{lem:polynomial_analytic}
  \uses{def:polynomial, def:analytic_function}
  %\lean{}
  % \leanok
  Every polynomial function $p \colon \bC \to \bC$ is analytic.
\end{lemma}
\begin{proof}
  \uses{lem:derivative_linearity, lem:leibniz_rule}
  % \leanok
  \ldots
\end{proof}

\begin{lemma}[Rational functions are analytic]
  \label{lem:rational_function_analytic}
  \uses{def:rational_function, def:analytic_function}
  %\lean{}
  % \leanok
  Every rational function $f \colon U \to \bC$ is analytic
  on its domain of definition $U \subset \bC$.
\end{lemma}
\begin{proof}
  \uses{lem:polynomial_analytic, lem:quotient_derivative}
  % \leanok
  \ldots
\end{proof}



\section{Cauchy-Riemann equations}

\begin{theorem}[Cauchy-Riemann equations \Palka{Thm~III.2.2}]
  \label{thm:cauchy_riemann}
  \uses{def:analytic_function}
  %\lean{}
  % \leanok
  Let $f \colon U \to \bC$ be a function defined on an open set $U \subset \bC$,
  and let $u \colon U \to \bR$ and $v \colon U \to \bR$ be its
  real and imaginary parts, viewed as real functions of two real variables,
  \begin{align*}
    u(x,y) = \re \Big( f (x + \ii y) \Big)
    \quad \text{and} \quad
    v(x,y) = \im \Big( f (x + \ii y) \Big)
  \end{align*}
  so that $f = u + \ii \, v$.
  Suppose furthermore that $u$ and $v$ are continuously real-differentiable on $U$.

  Then the following are equivalent:
  \begin{itemize}
    \item The partial derivatives of $u$ and $v$ satisfy
          the \term{Cauchy-Riemann equations}
          \begin{align*}
            %\pder{x} u(x,y) = \pder{y} v(x,y)
            %\pderof{u}{x} (x,y) = \pderof{v}{y} (x,y)
            \pderof{u}{x} = \pderof{v}{y}
            \quad \text{and} \quad
            %\pder{y} u(x,y) = -\pder{x} v(x,y) .
            %\pderof{u}{y} (x,y) = - \pderof{v}{x} (x,y) .
            \pderof{u}{y} = - \pderof{v}{x}
          \end{align*}
          in $U$.
    \item The function $f$ is analytic.
  \end{itemize}
  %Then we have:
%
  %\textbf{(i):} If $f$ is analytic, then $u$ and $v$ are real-differentiable
  %at every point $(x,y) \in U$ and their partial derivatives satisfy
  %the \term{Cauchy-Riemann equations}
  %\begin{align*}
  %  %\pder{x} u(x,y) = \pder{y} v(x,y)
  %  %\pderof{u}{x} (x,y) = \pderof{v}{y} (x,y)
  %  \pderof{u}{x} = \pderof{v}{y}
  %  \quad \text{and} \quad
  %  %\pder{y} u(x,y) = -\pder{x} v(x,y) .
  %  %\pderof{u}{y} (x,y) = - \pderof{v}{x} (x,y) .
  %  \pderof{u}{y} = - \pderof{v}{x} .
  %\end{align*}
%
  %\textbf{(ii):} If $u$ and $v$ are continuously real-differentiable
  %in all of $U$ and their partial derivatives satisfy
  %the Cauchy-Riemann equations above, then $f$ is analytic.
\end{theorem}
\begin{proof}
  \uses{lem:complex_linear_vs_real_linear, lem:local_linear_approximation}
  % \leanok
  \ldots
\end{proof}

\begin{lemma}[The complex exponential is analytic]
  \label{lem:exp_analytic}
  \uses{def:complex_exp, def:analytic_function}
  %\lean{}
  % \leanok
  The complex exponential function $\exp \colon \bC \to \bC$ is analytic.
\end{lemma}
\begin{proof}
  \uses{thm:cauchy_riemann}
  % \leanok
  \ldots
\end{proof}

\begin{lemma}[Branches of $n$th root functions are analytic]
  \label{lem:root_function_analytic}
  \uses{def:analytic_function, def:principal_complex_nth_root}
  % \lean{}
  % \leanok
  The principal branch of the $n$th root function
  $z \mapsto \sqrt[n]{z}$ is analytic on its domain
  $\bC \setminus (-\infty,0]$.

  (Different branch choices can be made to obtain analyticity on
  other domains, but for $n \ge 2$,
  no branch of $\sqrt[n]{z}$ can be made analytic on all of $\bC$.)
\end{lemma}
\begin{proof}
  \uses{lem:inverse_derivative}
  % \leanok
  \ldots
\end{proof}



\subsection{Consequences of Cauchy-Riemann equations}

\begin{lemma}[Analytic functions of vanishing derivative]
  \label{lem:zero_derivative_implies_constant}
  \uses{def:analytic_function}
  %\lean{}
  % \leanok
  Suppose that $f \colon D \to \bC$ is a analytic function
  on a domain $D \subset \bC$ (i.e. a connected open set)
  such that $f'(z) = 0$ for all $z \in D$.
  Then $f$ is a constant function.
\end{lemma}
\begin{proof}
  \uses{thm:cauchy_riemann, lem:vanishing_partial_derivatives_implies_constant}
  % \leanok
  \ldots
\end{proof}

\begin{theorem}[Criteria for constantness of a analytic function]
  \label{thm:constantness_criteria}
  \uses{def:analytic_function}
  %\lean{}
  % \leanok
  Suppose that $f \colon D \to \bC$ is a analytic function
  on a domain $D \subset \bC$ (i.e. a connected open set).
  If any of the functions $u = \re(f) \colon D \to \bR$,
  $v = \im(f) \colon D \to \bR$, $|f| \colon D \to \bR$,
  is constant on $D$, then $f$ is itself a constant function.
\end{theorem}
\begin{proof}
  \uses{thm:cauchy_riemann, lem:zero_derivative_implies_constant}
  % \leanok
  \ldots
\end{proof}
